\chapter{45E100 Fast Ethernet Controller}

\section{Overview}

The 45E100 is a new and simple Fast Ethernet controller that has been
designed specially for the MEGA65 and for 8-bit computers generally.
In addition to supporting 100Mbit Fast Ethernet, it is radically
different from other ethernet controllers, such as the RR-NET.

The 45E100 includes dual receive buffers, allowing one frame to be
processed while another is receiving.  It includes automatic CRC32
checking on reception, and automatic CRC32 generation for transmit, considerably
reducing the burden on the processor and allowing for very simple programs.  It also supports true
full-duplex operation at 100Mbit per second, allowing for total bi-directional
throughput exceeding 100Mbit per second.  The MAC address is software
configurable, and promiscuous mode is supported, as are individual control
of the reception of broadcast and multi-cast ethernet frames.  The 45E100 also
supports both transmit and receive interrupts, allowing greatly improved
real-world performance. When especially low latency is required, it is also possible
to immediately abort the transmission of the current ethernet frame, so that a
higher-priority frame can be immediately sent.
These features combine to enable sub-millisecond round trip latencies,
which can be of particular value for interactive applications, such as multi-player network
games.

\subsection{Differences to the RR-NET and similar solutions}

The RR-NET and other ethernet controllers for the Commodore(tm) line
of 8-bit home computers generally use an ethernet controller that was
designed for 16-bit PCs, but that also supports a so-called ``8-bit mode,''
which suffers from a number of disadvantages. The primary disadvantages
are the lack of working interrupts, and procesor intensive access to
the ethernet frame buffers.  The lack of interrupts forces programs to
use polling to check for the arrival of new ethernet frames.  This,
together with the complexities of accessing the buffers results in an
ethernet interface that is very slow, and whose real-world throughput
is considerably less than its theoretical 10Mbits per second.  Even
a Commodore 64 with REU cannot achieve speeds above several tens of
kilobytes per second.

In contrast, the 45E100 supports both RX (ethernet frame received) interrupts and TX
(ready to transmit) interrupts, freeing the processor from having to poll
the device. Because the 45E100 supports RX interrupts, there is no need for large
numbers of receive buffers, which is why the 45E100 requires only two RX buffers
to achieve very high levels of performance.

Further, the 45E100 supports direct memory mapping of the
ethernet frame buffers, allowing for much more efficient access, including
by DMA.  Using the MEGA65's integrated DMA controller it is quite possible
to achieve transfer rates of several mega-bytes per second -- some 100x
faster than the RR-NET.

\subsection{Theory of Operation: Receiving Frames}

The 45E100 is simple to operate: To begin receiving ethernet frames, the programmer
needs only to clear the RST bit (bit 0 of register \$D6E0) to release the
ethernet controller from reset.  It will then auto-negotiate connection at
the highest available speed, typically 100Mbit, full-duplex.  The RXBLKD bit (bit 6 of \$D6E0)
should then be checked, and if set, the RXBM (bit 2 of register \$D6E1) bit should be toggled to switch
the active and mapped receive buffers, so that the 45E100 knows that the
program no longer needs the contents of the previously mapped buffer, and can
safely begin receiving an ethernet frame into that buffer.

When the 45E100 receives an ethernet frame, it will assert RXBLKD to indicate that the receive
buffer has been filled with an ethernet frame.  No further ethernet frames will be received until
RXBLKD is cleared again, as described above. This is because the 45E100 has only two receive buffers
for ethernet frames: one of which is mapped visible to the processor, and the other which is visible
to the 45E100's ethernet engine at any point in time.  Toggling RXBM allows toggling between which
of the two buffers is mapped and which is ready to receive an ethernet frame.  The buffers are 2KiB
bytes each.  The first two bytes are used to indicate the length of the received frame, and four
are consumed by the ethernet CRC32 code, yielding an effective Maximum Transport Unit (MTU) length
of 2,042 bytes.  The ethernet frame data begins at byte offset 2 in the receive buffer, with the
frame length written LSB-first in the first two bytes.

Because of the very rapid rate at which Fast Ethernet frames can be received, a programmer should use the
receive interrupt feature, enabled by setting RXQEN (bit 7 of \$D6E1).  Polling is possible as an alternative, but
is not recommended with the 45E100, because at the 100Mbit Fast Ethernet speed, packets can arrive
as often as every 10 microseconds.  Fortunately, at the MEGA65's 40MHz full speed mode, and using
the 20MiB per second DMA copy functionality, it is possible to keep up with such high data rates.

\subsection{Theory of Operation: Sending Frames}

Sending frames is similarly simple: The program must simply load the frame to be transmitted into
the transmit buffer, write its length into TXSZLSB and TXSZMSB registers, and then write \$01 into
the COMMAND register.  The frame will then begin to transmit, as soon as the transmitter is idle.
There is no need to calculate and attach an ethernet CRC32 field, as the 45E100 does this automatically.

Unlike for the receiver, there is only one frame buffer for the transmitter (this may be changed in
a future revision). This means that you cannot prepare the next frame until the previous frame has
already been sent.  This slightly reduces the maximum data throughput, in return for a very simple
architecture.

Also, note that the transmit buffer is write-only from the processor bus interface. This means that
you cannot directly read the contents of the transmit buffer, but must load values ``blind''.  Finally,
the 45E100 allows you to send ethernet

\subsection{Advanced Features}

In addition to operating as a simple and efficient ethernet frame transceiver, the 45E100
includes a number of advanced features, described here.

\subsubsection{Broadcast and Multicast Traffic and Promiscuous Mode}

The 45E100 supports filtering based on the destination Ethernet address, i.e., MAC address.
By default, only frames where the destination Ethernet address matches the ethernet address
programmed into the MACADDR1 -- MACADDR6 registers will be received. However, if the MCST bit
is set, then multicast ethernet frames will also be received. Similarly, setting the BCST bit
will allow all broadcast frames, i.e., with MAC address ff:ff:ff:ff:ff:ff, to be received.
Finally, if the NOPROM bit is cleared, the 45E100 disables the filter entirely, and will
receive all valid ethernet frames.

\subsubsection{Debugging and Diagnosis Features}

The 45E100 also supports several features to assist in the diagnosis of ethernet problems.
First, if the NOCRC bit is set, then even ethernet frames that have invalid CRC32 values
will be received. This can help debug faulty ethernet devices on a network.

If the STRM bit is set, the ethernet transmitter transmits a continuous stream of debugging frames
supplied via a special high-bandwidth logging interface. By default, the 45E100 emits a
stream of approximately 2,200 byte ethernet frames that contain compressed video provided
by a VIC-IV or compatible video controller that supports the MEGA65 video-over-ethernet
interface.  By writing a custom decoder for this stream of ethernet frames, it is possible
to create a remote display of the MEGA65 via ethernet. Such a remote display can be used,
for example, to facilitate digital capture of the display of a MEGA65.

The size and content of the debugging frames can be controlled by writing special values to the 
COMMAND register.  Writing \$F1 allows the selection of frames that are ~1,200 bytes long.
While this reduces the performance of the debugging and streaming features, it allows the reception
of these frames on systems whose ethernet controllers cannot be configured to receive frames
of ~2,200 bytes.

If the STRM bit is set and bit 2 of \$D6E1 is also set, a compressed log of instructions executed by
the 45gs02 CPU will instead be streamed, if a compatible processor is connected to this interface.
This mechanism includes back-pressure, and will cause the 45gs02 processor to slowdown,
so that the instruction data can be emitted.  This typically limits the speed of the connected
45gs02 processor to around 5MHz, depending on the particular instruction mix.

Note also that
the status of bit 2 of \$D6E1 cannot currently be read directly. This may be correcteed in a future
revision.

Finally, if the video streaming functionality is enabled, this also enables reception of synthetic
keyboard events via ethernet.  These are delivered to the MEGA65's Keyboard Complex Interface Adapter
(KCIA), allowing full remote interaction with a MEGA65 via its ethernet interface.  This feature is
primarily intended for development.

\section{Memory Mapped Registers}

The 45E100 Fast Ethernet controller is a MEGA65-specific feature.
It is therefore only available in the MEGA65 IO context.
This is enabled by writing \$53 and then \$47 to VIC-IV register \$D02F.
If programming in BASIC, this can be done with:

\begin{screenoutput}
POKE53295,ASC("G"):POKE53295,ASC("S")
\end{screenoutput}

The 45E100 Fast Ethernet controller has the following registers

\setlength{\tabcolsep}{3pt}
\begin{longtable}{|L{1.2cm}|L{1.1cm}|c|c|c|c|c|c|c|c|}
\hline
{\bf{HEX}} & {\bf{DEC}} & {\bf{DB7}} & {\bf{DB6}} & {\bf{DB5}} & {\bf{DB4}} & {\bf{DB3}} & {\bf{DB2}} & {\bf{DB1}} & {\bf{DB0}} \\
\hline
\endfirsthead
\multicolumn{3}{l@{}}{\ldots continued}\\
\hline
{\bf{HEX}} & {\bf{DEC}} & {\bf{DB7}} & {\bf{DB6}} & {\bf{DB5}} & {\bf{DB4}} & {\bf{DB3}} & {\bf{DB2}} & {\bf{DB1}} & {\bf{DB0}} \\
\endhead
\multicolumn{3}{l@{}}{continued \ldots}\\
\endfoot
\hline
\endlastfoot
\small  D6E0 & \small 55008 & \small -- & \small RXBLKD & \small -- & \small KEYEN & \small DRXDV & \multicolumn{2}{c|}{\small DRXD}& \small RST \\
\hline
\small  D6E1 & \small 55009 & \small RXQEN & \small TXQEN & \small RXQ & \small TXQ & \small STRM & \small RXBU & \small RXBM & \small -- \\
\hline
\small  D6E2 & \small 55010 & \multicolumn{8}{c|}{\small TXSZLSB}\\
\hline
\small  D6E3 & \small 55011 & \multicolumn{8}{c|}{\small TXSZMSB}\\
\hline
\small  D6E4 & \small 55012 & \multicolumn{8}{c|}{\small COMMAND}\\
\hline
\small  D6E5 & \small 55013 & \multicolumn{2}{c|}{\small --}& \small MCST & \small BCST & \multicolumn{2}{c|}{\small TXPH}& \small NOCRC & \small NOPROM \\
\hline
\small  D6E6 & \small 55014 & \multicolumn{3}{c|}{\small MIIMPHY}& \multicolumn{5}{c|}{\small MIIMREG}\\
\hline
\small  D6E7 & \small 55015 & \multicolumn{8}{c|}{\small MIIMVLSB}\\
\hline
\small  D6E8 & \small 55016 & \multicolumn{8}{c|}{\small MIIMVMSB}\\
\hline
\small  D6E9 & \small 55017 & \multicolumn{8}{c|}{\small MACADDR1}\\
\hline
\small  D6EA & \small 55018 & \multicolumn{8}{c|}{\small MACADDR2}\\
\hline
\small  D6EB & \small 55019 & \multicolumn{8}{c|}{\small MACADDR3}\\
\hline
\small  D6EC & \small 55020 & \multicolumn{8}{c|}{\small MACADDR4}\\
\hline
\small  D6ED & \small 55021 & \multicolumn{8}{c|}{\small MACADDR5}\\
\hline
\small  D6EE & \small 55022 & \multicolumn{8}{c|}{\small MACADDR6}\\
\hline
\end{longtable}
\begin{itemize}
\item{\bf{BCST}} Accept broadcast frames
\item{\bf{COMMAND}} Ethernet command register (write only)
\item{\bf{DRXD}} Read ethernet RX bits currently on the wire
\item{\bf{DRXDV}} Read ethernet RX data valid (debug)
\item{\bf{KEYEN}} Allow remote keyboard input via magic ethernet frames
\item{\bf{MACADDR1}} Ethernet MAC address
\item{\bf{MACADDR2}} Ethernet MAC address
\item{\bf{MACADDR3}} Ethernet MAC address
\item{\bf{MACADDR4}} Ethernet MAC address
\item{\bf{MACADDR5}} Ethernet MAC address
\item{\bf{MACADDR6}} Ethernet MAC address
\item{\bf{MCST}} Accept multicast frames
\item{\bf{MIIMPHY}} Ethernet MIIM PHY number (use 0 for Nexys4, 1 for MEGA65 r1 PCBs)
\item{\bf{MIIMREG}} Ethernet MIIM register number
\item{\bf{MIIMVLSB}} Ethernet MIIM register value (LSB)
\item{\bf{MIIMVMSB}} Ethernet MIIM register value (MSB)
\item{\bf{NOCRC}} Disable CRC check for received packets
\item{\bf{NOPROM}} Ethernet disable promiscuous mode
\item{\bf{RST}} Write 0 to hold ethernet controller under reset
\item{\bf{RXBLKD}} Indicate if ethernet RX is blocked until RX buffers rotated
\item{\bf{RXBM}} Set which RX buffer is memory mapped
\item{\bf{RXBU}} Indicate which RX buffer was most recently used
\item{\bf{RXQ}} Ethernet RX IRQ status
\item{\bf{RXQEN}} Enable ethernet RX IRQ
\item{\bf{STRM}} Enable streaming of CPU instruction stream or VIC-IV display on ethernet
\item{\bf{TXPH}} Ethernet TX clock phase adjust
\item{\bf{TXQ}} Ethernet TX IRQ status
\item{\bf{TXQEN}} Enable ethernet TX IRQ
\item{\bf{TXSZLSB}} TX Packet size (low byte)
\item{\bf{TXSZMSB}} TX Packet size (high byte)
\end{itemize}


\subsection{COMMAND register values}

The following values can be written to the COMMAND register to perform the described functions.
In normal operation only the STARTTX command is required, e.g., by performing the following POKE:

\begin{screenoutput}
POKE55012,1
\end{screenoutput}

\begin{longtable}{|L{1.2cm}|L{1.1cm}|C{2cm}|L{6cm}|}
\hline
{\bf{HEX}} & {\bf{DEC}} & {\bf{Signal}} & {\bf{Description}} \\
\hline
\endfirsthead
\multicolumn{3}{l@{}}{\ldots continued}\\
\hline
{\bf{HEX}} & {\bf{DEC}} & {\bf{Signal}} & {\bf{Description}} \\
\hline
\endhead
\multicolumn{3}{l@{}}{continued \ldots}\\
\endfoot
\hline
\endlastfoot
\small  0000 & \small 0 & STOPTX & Immediately stop transmitting the current ethernet frame.  Will cause a partially sent frame to be received, most likely resulting in the loss of that frame.   \\
\hline
\small  0001 & \small 1 & STARTTX & Transmit packet \\
\hline
\small  00D0 & \small 208 & RXNORMAL & Disable the effects of RXONLYONE \\
\hline
\small  00D4 & \small 212 & DEBUGVIC & Select VIC-IV debug stream via ethernet when \$D6E1.3 is set \\
\hline
\small  00DC & \small 220 & DEBUGCPU & Select CPU debug stream via ethernet when \$D6E1.3 is set \\
\hline
\small  00DE & \small 222 & RXONLYONE & Receive exactly one ethernet frame only, and keep all signals states (for debugging ethernet sub-system) \\
\hline
\small  00F1 & \small 241 & FRAME1K & Select ~1KiB frames for video/cpu debug stream frames (for receivers that do not support MTUs of greater than 2KiB) \\
\hline
\small  00F2 & \small 242 & FRAME2K & Select ~2KiB frames for video/cpu debug stream frames, for optimal performance. \\
\hline
\end{longtable}


\section{Example Programs}

Example programs for the ethernet controller exist in imperfect for in the MEGA65 Core repository on github in the src/tests and src/examples directories.

