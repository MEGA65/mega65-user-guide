\chapter{45E100 Fast Ethernet Controller}

\section{Overview}

The 45E100 is a new and simple Fast Ethernet controller that has been
designed specially for the MEGA65 and for 8-bit computers generally.
In addition to supporting 100Mbit Fast Ethernet, it is radically
different from other Ethernet controllers, such as the RR-NET.

The 45E100 includes four receive buffers, allowing upto three frames to be
received while another is being processed, or to allow less frequent processing of interrupts.
These receive buffers can be memory mapped, and also directly accessed using the MEGA65's DMA controller.
Together with automatic CRC32 checking on reception, and automatic CRC32 generation for transmit, these features considerably
reduce the burden on the processor, and make it much simpler to write ethernet-enabled programs.

The 45E100 also supports true
full-duplex operation at 100Mbit per second, allowing for total bi-directional
throughput exceeding 100Mbit per second.  The MAC address is software
configurable, and promiscuous mode is supported, as are individual control
of the reception of broadcast and multi-cast Ethernet frames.

The 45E100 also
supports both transmit and receive interrupts, allowing greatly improved
real-world performance. When especially low latency is required, it is also possible
to immediately abort the transmission of the current Ethernet frame, so that a
higher-priority frame can be immediately sent.
These features combine to enable sub-millisecond round trip latencies,
which can be of particular value for interactive applications, such as multi-player network
games.

\subsection{Differences to the RR-NET and similar solutions}

The RR-NET and other Ethernet controllers for the Commodore\texttrademark \ line
of 8-bit home computers generally use an Ethernet controller that was
designed for 16-bit PCs, but that also supports a so-called ``8-bit mode,''
which suffers from a number of disadvantages. These disadvantages
include the lack of working interrupts, as well as processor intensive access to
the Ethernet frame buffers.  The lack of interrupts forces programs to
use polling to check for the arrival of new Ethernet frames.  This,
together with the complexities of accessing the buffers results in an
Ethernet interface that is very slow, and whose real-world throughput
is considerably less than its theoretical 10Mbits per second.  Even
a Commodore 64 with REU cannot achieve speeds above several tens of
kilobytes per second.

In contrast, the 45E100 supports both RX (Ethernet frame received) interrupts and TX
(ready to transmit) interrupts, freeing the processor from having to poll
the device. Because the 45E100 supports RX interrupts, there is no need for large
numbers of receive buffers, which is why the 45E100 requires only two RX buffers
to achieve very high levels of performance.

Further, the 45E100 supports direct memory mapping of the
Ethernet frame buffers, allowing for much more efficient access, including
by DMA.  Using the MEGA65's integrated DMA controller it is quite possible
to achieve transfer rates of several mega-bytes per second -- some 100x
faster than the RR-NET.

\subsection{Theory of Operation: Receiving Frames}

The 45E100 is simple to operate: To begin receiving Ethernet frames, the programmer
needs only to clear the RST and TXRST bits (bit 0 of register \$D6E0) to ensure that the
Ethernet controller is reset, and then set these bits to 1, to release the controller from the reset state.
It will then auto-negotiate connection at the highest available speed, typically 100Mbit, full-duplex.

If you wish to simply poll for the arrival of ethernet frames, check the RXQ bit (bit 5 of \$D6E1).
If it is set, then there is at least one frame that has been received.  To access the next frame that
has been received, write \$01 to \$D6E1, and then \$03 to \$D6E1.  This will rotate the ring of receive
buffers, to make the next received frame accessible by the processor. The receive buffer that was previously
accessible by the processor is marked free, and the 45E100 will use it to receive another ethernet frame when required.

Because the 45E100 has four receive buffers, it is possible that to process multiple frames in succession by
following this procedure. If all receive buffers contain received frames, and the processor has not accepted them, then
the RXBLKD signal will be asserted, so that the processor knows that it if any more frames are received, they will be lost.
Programmers should take care to avoid this situation.  As the 45E100 supports receive interrupts, this is generally easy
to manage -- but don't underestimate how often ethernet frames can arrive on a 100mbit Fast Ethernet connection:
If a sender sends a continuous stream of minimum-length ethernet frames, they can arrive every 6 microseconds or so!
While, it is unlikely that you will have to deal with such a high rate of packet reception, you should anticipate the need
to process packets at least every milli-second. In particular, a once-per-frame CIA or raster IRQ may cause some packets
to be lost, more than three arrive in a 16 -- 20 ms video frame.  The RXBLKD signal can be used to determine if this situation
is likely to have occurred. But note that it indicates only when all receive buffers are occupied, not if any further frames
arrived while there were no free receive buffers.

The receive buffers are 2KiB bytes each, and can each hold only one received ethernet frame at a time.
This is different to some ethernet controllers that use their total receive buffer memory as a simple ring buffer.
The reason for this is to keep the mechanism for programmers as simple as possible. By having the fixed buffers, it
means that the controller can memory map the received ethernet frames in exactly the same location each time, making
it possible to write much simpler receiver programs, because the location of the received ethernet frames can be
assumed to be constant.

The structure of a receive buffer containing an ethernet frame is quite simple:
The first two bytes indicate the length of the received frame. The frame then follows immediately.
The effective Maximum Transport Unit (MTU) length is 2,042 bytes, as the last four bytes are occupied by the
CRC32 checksum of the received ethernet frame.
The layout of the receive buffers is thus as follows:

\begin{longtable}{|L{1.0cm}|L{1.1cm}|C{1.3cm}|L{6.9cm}|}
\hline
{\bf{HEX}} & {\bf{DEC}} & {\bf{Length}} & {\bf{Description}} \\
\hline
\endfirsthead
\multicolumn{3}{l@{}}{\ldots continued}\\
\hline
{\bf{HEX}} & {\bf{DEC}} & {\bf{Length}} & {\bf{Description}} \\
\hline
\endhead
\multicolumn{3}{l@{}}{continued \ldots}\\
\endfoot
\hline
\endlastfoot
\small  0000 & \small 0 & 1 & The low byte of the length of the received ethernet frame. \\
\hline
\small  0001 & \small 1 & 1 & The lower four bits contain the upper bits of the length of the received ethernet frame.  Bit 4 is set if the received ethernet frame is a multi-cast frame. Bit 5 if it is a broadcast frame. Bit 6 is set if the frame's destination address matches the 45E100's programmed MAC address. Bit 7 is set if the CRC32 check for the received frame failed, i.e., that the frame is either truncated or was corrupted in transit. \\
\hline
\small  0002 -- 07FB & \small 2 -- 2,043 & 2,042 & The received frame. Frames shorter than 2,042 bytes will begin at offset 2. \\
\hline
\small  07FC -- 07FF & \small 2,044 -- 2,047 & 4 & Reserved space for holding the CRC32 code during reception.  The CRC32 code is, however, always located directly after the received frame, and thus will only occupy this space if the received frame is more than 2,038 bytes long. '' \\
\hline
\end{longtable}

Because of the very rapid rate at which Fast Ethernet frames can be received, a programmer should use the
receive interrupt feature, enabled by setting RXQEN (bit 7 of \$D6E1).  Polling is possible as an alternative, but
is not recommended with the 45E100, because at the 100Mbit Fast Ethernet speed, packets can arrive
as often as every 5 microseconds.  Fortunately, at the MEGA65's 40MHz full speed mode, and using
the 20MiB per second DMA copy functionality, it is possible to keep up with such high data rates.

\subsection{Accessing the Ethernet Frame Buffers}

Unlike on the RR-NET, the 45E100's ethernet frame buffers are able to be memory mapped, allowing rapid access via DMA
or through assembly language programs.  It is also possible to access the buffers from BASIC with some care.

The frame buffers can either be accessed from their natural location in the MEGA65's extended address space at address
\$FFDE800 -- \$FFDEFFF, or they can be mapped into the normal C64/C65 \$D000 I/O address space.  Care must be
taken as mapping the ethernet frame buffers into the \$D000 I/O address space causes all other I/O devices to unavailable
during this time.  Therefore CIA-based interrupts MUST be disabled before doing so, whether using BASIC or machine code.  Therefore
when programming in assembly language or machine code, it is recommended to use the natural location, and to access this
memory area using one of the three mechanisms for accessing extended address space, which are described in \bookvref{sec:extended-memory}.

The method of
disabling interrupts differs depending on the context in which a program is being written. For programs being written using C64-mode's BASIC 2, the following will work:

\begin{screenoutput}
  POKE56333,127: REM DISABLE CIA TIMER IRQS
\end{screenoutput}

While for MEGA65's BASIC 65, the following must instead be used, because a VIC-III raster interrupt is used instead of a CIA-based timer interrupt:

\begin{screenoutput}
POKE53274,0: REM DISABLE VIC-II/III/IV RASTER IRQS
\end{screenoutput}

Once this has been done, the I/O context for the ethernet controller can be activated by writing \$45 (69 in decimal, equal to the character 'E' in PETSCII)) and \$54 (84 in decimal, equal to the character 'T' in PETSCII) into the VIC-IV's KEY register (\$D02F, 53295 in decimal), for example:

\begin{screenoutput}
POKE53295,ASC("E"):POKE53295,ASC("T")
\end{screenoutput}

At this point, the ethernet RX buffer can be read beginning at location \$D000 (53248 in decimal), and the TX buffer can be written to at the same address.  Refer to `Theory of Operation: Receiving Frames' above for further explanation on this.

Once you have finished accessing the ethernet frame buffer, you can restore the normal C64, C65 or MEGA65 I/O context by writing to the VIC-III/IV's KEY register.  In most cases, it will make the most sense to revert to the MEGA65's I/O context by writing \$47 (71 decimal) in and \$53 (83 in decimal) to the KEY register, for example:

\begin{screenoutput}
POKE53295,ASC("G"):POKE53295,ASC("S")
\end{screenoutput}

Finally, you should then re-enable interrupts, which will again depend on whether you are programming from C64 or C65-mode.  For C64-mode:

\begin{screenoutput}
POKE56333,129
\end{screenoutput}

For C65-mode it would be:

\begin{screenoutput}
POKE53274,129
\end{screenoutput}



\subsection{Theory of Operation: Sending Frames}

Sending frames is similarly simple: The program must simply load the frame to be transmitted into
the transmit buffer, write its length into TXSZLSB and TXSZMSB registers, and then write \$01 into
the COMMAND register.  The frame will then begin to transmit, as soon as the transmitter is idle.
There is no need to calculate and attach an ethernet CRC32 field, as the 45E100 does this automatically.

Unlike for the receiver, there is only one frame buffer for the transmitter (this may be changed in
a future revision). This means that you cannot prepare the next frame until the previous frame has
already been sent.  This slightly reduces the maximum data throughput, in return for a very simple
architecture.

Also, note that the transmit buffer is write-only from the processor bus interface. This means that
you cannot directly read the contents of the transmit buffer, but must load values ``blind''.  Finally,
the 45E100 allows you to send ethernet.

\subsection{Advanced Features}

In addition to operating as a simple and efficient ethernet frame transceiver, the 45E100
includes a number of advanced features, described here.

\subsubsection{Broadcast and Multicast Traffic and Promiscuous Mode}

The 45E100 supports filtering based on the destination Ethernet address, i.e., MAC address.
By default, only frames where the destination Ethernet address matches the ethernet address
programmed into the MACADDR1 -- MACADDR6 registers will be received. However, if the MCST bit
is set, then multicast ethernet frames will also be received. Similarly, setting the BCST bit
will allow all broadcast frames, i.e., with MAC address ff:ff:ff:ff:ff:ff, to be received.
Finally, if the NOPROM bit is cleared, the 45E100 disables the filter entirely, and will
receive all valid ethernet frames.

\subsubsection{Debugging and Diagnosis Features}

The 45E100 also supports several features to assist in the diagnosis of ethernet problems.
First, if the NOCRC bit is set, then even ethernet frames that have invalid CRC32 values
will be received. This can help debug faulty ethernet devices on a network.

If the STRM bit is set, the ethernet transmitter transmits a continuous stream of debugging frames
supplied via a special high-bandwidth logging interface. By default, the 45E100 emits a
stream of approximately 2,200 byte ethernet frames that contain compressed video provided
by a VIC-IV or compatible video controller that supports the MEGA65 video-over-ethernet
interface.  By writing a custom decoder for this stream of ethernet frames, it is possible
to create a remote display of the MEGA65 via ethernet. Such a remote display can be used,
for example, to facilitate digital capture of the display of a MEGA65.

The size and content of the debugging frames can be controlled by writing special values to the
COMMAND register.  Writing \$F1 allows the selection of frames that are ~1,200 bytes long.
While this reduces the performance of the debugging and streaming features, it allows the reception
of these frames on systems whose ethernet controllers cannot be configured to receive frames
of ~2,200 bytes.

If the STRM bit is set and bit 2 of \$D6E1 is also set, a compressed log of instructions executed by
the 45gs02 CPU will instead be streamed, if a compatible processor is connected to this interface.
This mechanism includes back-pressure, and will cause the 45gs02 processor to slowdown,
so that the instruction data can be emitted.  This typically limits the speed of the connected
45gs02 processor to around 5MHz, depending on the particular instruction mix.

Note also that
the status of bit 2 of \$D6E1 cannot currently be read directly. This may be corrected in a future
revision.

Finally, if the video streaming functionality is enabled, this also enables reception of synthetic
keyboard events via ethernet.  These are delivered to the MEGA65's Keyboard Complex Interface Adapter
(KCIA), allowing full remote interaction with a MEGA65 via its ethernet interface.  This feature is
primarily intended for development.

\section{Memory Mapped Registers}

The 45E100 Fast Ethernet controller is a MEGA65-specific feature.
It is therefore only available in the MEGA65 I/O context.
This is enabled by writing \$53 and then \$47 to VIC-IV register \$D02F.
If programming in BASIC, this can be done with:

\begin{screenoutput}
POKE53295,ASC("G"):POKE53295,ASC("S")
\end{screenoutput}

The 45E100 Fast Ethernet controller has the following registers

\input{regtable_ETH.MEGA65}

\subsection{COMMAND register values}

The following values can be written to the COMMAND register to perform the described functions.
In normal operation only the STARTTX command is required, for example, by performing the following {\bf POKE}:

\begin{screenoutput}
POKE55012,1
\end{screenoutput}

\input{regtable_ETHCOMMAND.MEGA65}

\section{Example Programs}

Example programs for the ethernet controller exist in imperfect for in the MEGA65 Core repository on github in the src/tests and src/examples directories.

If you wish to use the ethernet controller for TCP/IP traffic, you may wish to examine the port of WeeIP to the MEGA65 at
\url{https://github.com/mega65/mega65-weeip}.  The code that controls the ethernet controller is located in {\tt eth.c}.
