\chapter{45IO27 Multi-Function IO Controller}

\section{Overview}

The 45IO27 is a multi-purpose IO controller that incorporates the functions of the
C65's F011 floppy controller, together with the MEGA65's SD card controller interface,
and a number of other miscellaneous IO functions.

Each of these major functions is covered in a separate section of this chapter.

\section{F011-compatible Floppy Controller}

The MEGA65 computer is one of the very few modern computers that still
includes first-class support for magnetic floppy drives.  It includes
a floppy controller that is backwards compatible with the C65's F011D
floppy drive controller.

However, unlike the F011D, the MEGA65's
floppy disk controller supports HD and ED media, and similar to the
1541 floppy drive, it also supports variable data rates, so that a
determined user could develop disk formats that store more data,
include robust copy protection schemes, or both.

GCR encoding is not currently supported, but may be supported by a
future revision of the controller.  It may also be possible with some
creativity and effort to use the debug register interface to read
double-density GCR formatted media.  This is because there are debug
registers that can be queried to indicate the gap between each
successive magnetic domain -- which is sufficient to decode any disk
format. 

\subsection{Multiple Drive Support}

Like the C65's F011 floppy drive controller, the 45IO27 supports up to 8 drives.
The first two of those drives, drive 0 and drive 1, are assumed to be connected to a
standard 34-pin floppy cable, the same as used in standard PCs, i.e.,
with a twist in the cable to allow the use of two unjumpered drives.

As is described in later sections, it is possible to switch drive 0
and drive 1's position, without having to change cabling. Similarly,
either or both of the first two drives may reference a real floppy
drive, a D81 disk image stored on an attached SD card, or redirected
to the floppy drive virtualisation service, so that the sector
accesses can be handled by a connected computer, e.g., as part of a
comfortable and efficient cross-development environment.

The remaining six drives are supported only in conjunction with a
future C1565-compatible external drive port.

\subsection{Buffered Sector Operations}
\label{subsec:reading-from-disks-and-buffer-management}

The 45IO27 support two main modes of reading sectors from a
disk: byte-by-byte, and via a memory-mapped sector buffer.

The byte-by-byte mechanism consists of having a loop wait for the DRQ
signal to be asserted, and then reading the byte of data from the DATA
register (\$D087).

The memory-mapped sector buffer method consists of waiting for the
BUSY flag to clear, indicating that the entire sector has been read,
and then directly accessing the sector buffer located at \$FFD6C00 --
\$FFD6DFF. Care should be taken to ensure that the BUFSEL signal (bit 7 of \$D689) is
cleared, so that the floppy sector buffer is visible, rather than the
SD card sector buffer for programs other than the Hypervisor. This
is because only the Hypervisor has access to the full 4KiB SD
controller buffer space: Normal programs see either the floppy
sector buffer or the SD card sector buffer repeated 8 times between
\$FFD6000 and \$FFD6FFF.

Alternatively, the sector buffer can be mapped at \$DE00 -- \$DFFF,
i.e., in the 4KB IO area, by writing the \$81 to the SD command
register at \$D680.  This will hide any IO peripherals that are
otherwise using this area, e.g., from cartridges, or REU emulation.
This function can be disabled again by writing \$82 to the SD command
register. As with the normal sector buffer memory mapping at
\$FFD6xxx, the BUFSEL signal (bit 7 of \$D689) affects whether the FDC
or the SD card sector buffer is visible, for software not running in
Hypervisor mode.  Note that if you use the Matrix Mode / serial monitor
interface to inspect the contents of the sector buffer, that this occurs
in the Hypervisor context, and so the BUFSEL signal will be ignored,
and the full 4KiB buffer will be visible.

The memory-mapped sector buffer has the advantage that it can be
accessed via DMA, allowing for very efficient copies.  Also, it allows
for loading a sector to occur in the background, while your program
gets on with more interesting things in the meantime.

\subsection{Reading Sectors from a Disk}

There are several steps that you must follow in order to successfully read
a sector from a disk. If you follow these instructions, your code will work
with both physical disks, as well as D81 disk images that exist on the SD
card:

\begin{itemize}
\item First, enable the motor and select the appropriate drive. The F011
  supports upto 8 physical drives, although it is rare for more than two
  to be physically connected.  To enable the motor, write \$60 to \$D080.
  You should then write a SPINUP command (\$20) to \$D081, and wait for
  the BUSY flag (bit 7 of \$D082) to clear.  The drive is now spinning
  at speed, and ready to service requests.
\item Next, select the correct side of the disk by either setting or clearing
  the SIDE1 flag (bit 3 of \$D080).  This takes effect immediately.
\item Third, use the step-in and step-out commands (writing \$10 and \$18 to
  \$D081) as required to move the head to the correct track. Again, after each
  command, you should wait for the BUSY flag (bit 7 of \$D082) to clear, before
  issuing the next command.

  Note that you can check if the head is at track 0
  by checking the TRACK0 flag, but there is no fool-proof way to know if you
  are on any other specific track.  You can use the registers at \$D6A3 --
  \$D6A5 to see the track, sector and side value from the last sector header
  which passed under the head to make an informed guess as to which track
  is currently selected. Note that this only works for real disks, as disk
  images do not spin under the read head. Also note that it is possible for
  tracks to contain sectors which purposely or accidently have incorrect
  track numbers in the sector headers.
  
  \item Fourth, you need to load the desired track, sector and
side number into the TRACK, SECTOR and SIDE registers (\$D084, \$D085
and \$D086, respectively). The FDC is now primed ready to read a sector.

\item Fifth, you should write an appropriate read command
value into \$D081.  This will normally be \$40 (64).  You then wait
for the RDREQ signal (\$D083, bit 7) to go high, to indicate that the
sector has been found. You then either wait for each occassion when
DRQ goes high, and read byte-by-byte in such a loop, or wait for the
BUSY flag to clear and the DRQ and EQ flags to go high, which indicates
that the complete sector has been read into the buffer.

\end{itemize}

\subsection{Track Auto-Tune Function Deprecated}

The 45IO27 also includes a track ``auto-tune'' function, which is enabled
by clearing bit 7 of \$D696.  That function reads the sector headers to
determine which track the head is currently over, and steps the head in or
out to try to get to the correct track. Auto-tune is enabled by default.

\subsection{Sector Skew and Target Any Mode}

It is also worth noting that the TARGANY signal can be asserted to
tell the floppy controller to simply read the next sector that passes
under the head.  This applies only when using real floppy disks, where
it offers the considerable advantage of letting you read the sectors
in the order in which they exist on the disk. This allows you to read
a track at once, without having to wait for the index hole to pass by,
or having to know which sector will next pass under the head.

For example,
the C65 DOS formats disks using a skew factor of 7, while PCs may use
a different skew-factor. If you don't know the skew factor of the disk,
you may schedule the reading of the sectors on the track in a sub-optimal
order. This can result in transfer rates as low as 5 sectors per second,
compared with the optimal case of 50 sectors per second.
Thus with either correct sector order, or using the target any mode,
it is possible to read approximately two full tracks per second,
i.e., two sides $\times$ two tracks, or approximately 20KiB/second on
DD disks, or double that on HD disks, at around 40KiB/second.  This
compares very favourably with the C65 DOS loading speed, which is
typically nearer 1KiB/sec in C64 mode.

\subsection{Disk Layout and 1581 Logical Sectors}

The 1581 disk format is unusual in that the physical sectors on the
disk are a different size of the size of the data blocks that it
presents to the user.  Specifically, the disks use 512 byte sectors,
while the 1581 (and C65) DOS present 256 byte data blocks.
Two blocks are stored in each physical sector.  Also, the physical
track numbers are from 0 to 79, while the logical track numbers of the
DOS are 1 to 80.  Physical sectors are also numbered from 1 to 10,
while logical block numbers begin are 0 to 39.

This means that if you want to find a 1581 logical sector, you need to
know which physical sector it will be found in.  To determine the
physical sector that contains a block, you first subtract one from the
track number, and then divide the sector number by two.  Logical
sectors 0 to 19 of each track are located in physical sectors 1 to 10
on the first side of the disk.  Logical sectors 20 to 39 are 
located in physical sectors 1 to 10 on the reverse side of the disk.  

Thus we can map a some logical track and sector $t$,$s$ to the
physical track, side and sector as follows:

$track = t - 1$

$sector = (s/2)+1, IFF s < 20, ELSE = ((s-20)/2) + 1$

$side = 0 IFF sector < 20$

It is also worth noting that the 45IO27 is capable of reading from
tracks beyond track 80, provided that the disk drive is capable of
this.  Almost all 3.5 inch floppy drives are capable of reading at
least one extra track, as historically manufacturers of floppy disks
stored information about the disk on the 81st track.  In our
experience almost all drives will also be able to access an 82nd
track.

\subsection{FD2000 Disks}

The CMD(tm) FD2000(tm) high-density 3.5'' disk drives for Commodore(tm) computers
use an unusual disk layout that is quite different from PCs: They use 10 sectors,
the same as on 720KB double-density (DD) disks, but double the {\em sector size}
from 512 bytes to 1,024 bytes.  The 45IO27 does not currently support these
larger sectors. At least read-only support is planned to be added via a core update
in the future.

However, the 45IO27 {\em does} already support high-density disks and drives, with much
higher capacities than the FD2000 was able to support.

\subsection{High-Density and Variable-Density Disks}

The 45IO27 supports variable data rates, allowing the use of HD drives and media,
with a flexible approach to disk formats to support user experimentation, and the
easy manipulation of high-capacity software distribution formats.

You are really only limited by your imagination, available time, and
the limited number of people who are still interested in inserting a
floppy disk into their computer!

The standard high-density (HD) disk format is ``1.44MB'', using 18 sectors per track over 80 tracks.
This results in 80 tracks $\times$ 18 sectors $\times$ 2 sides = 2,880 sectors. As each sector
is 512 bytes, this corresponds to 1,440 KiB.  This leads us into the interesting wonderland of
``floppy disk marketing megabytes,'' a phenomena which long predates SD card and hard drive
manufacturers using 1,000,000 byte megabytes.

Curiously for floppy disks, the 1,024,000 byte ``megabyte'' was used, i.e., ``1MB'' = 1 KiB $\times$ 1 KB,
that is a strange hybrid of binary and decimal conventions.  Perhaps it was because the previous standard
was 720KB, and they thought people would thing it odd if double 720KB was 1.41MiB, and complain about the
missing kilo-bytes.  We will continue to use the 1,024KB = 1,000KiB floppy disk marketing mega-byte for
consistency with this historical inconsistency.

However, HD floppy disks are fundamentally capable of holding much more than 1.44MB. For example, the FD2000
stored 1.6MB by using double-sized sectors to squeeze the equivalent of 20 sectors per track, and the Amiga
went further by using track-at-once writing to fit 22 sectors per track. Both these formats used a constant
data rate over all tracks, and thus a constant number of sectors per track.

However, the circumference of the tracks on a 3.5'' floppy disk vary quite a lot: The inner track has a diameter
of around 2.5cm, while the outside track is 1.6$\times$ longer. 
The 1.44MB disk format is designed so that
the data is reliably stored on those shorter inner tracks.
This means that we should be able to fit 160\%
more data on the outer-most track compared with the inner-most track, subject to a number of terms and conditions
imposed by The Laws of Physics, the design of floppy drive electronics, the quality of media being used and
various other annoying things.
Because of this variability and uncertainty, the MEGA65's floppy controller supports fully variable data rate
on a track-by-track basis.

\subsection{Track Information Blocks}

To support variable data rates, the 45GS27 supports the use of
{\em Track Information Blocks}\index{Track Information Block} (TIBs)\index{TIB} that contain information on the
data rate and encoding used on the track. This allows users to experiment with various densities on various tracks,
and yet have the disks function automatically for buffered sector operations.

The Track Information Block is automatically created when using the automatic track format function, but must be
manually created if using unbuffered formatting. The TIB itself consists of the following data:

\begin{enumerate}
\item 3$\times$ \$A1 Sync bytes (written with clock byte \$FB)
\item \$65 MEGA65 Track Information Block marker (written with clock byte \$FF, as are all following bytes in the block)
\item The track number
\item The data rate divisor, in the same format as \$D6A2, i.e., data rate = 40.5MHz / value.
\item Track encoding information: Bit 7 = Track-at-once flag, 1 = no inter-sector gaps (Amiga style), 0 = with inter-sector gaps (normal), Bit 6 = data encoding, 0 = MFM, 1=RLL2,7. Other bits are reserved, and should be 0 when written.
\item Sector count, i.e., number of sectors on the track.
\item CRC byte 1, using the normal floppy disk CRC algorithm.
\item CRC byte 2, using the normal floppy disk CRC algorithm.
\end{enumerate}

The Track Information Block is always written a the data rate for a 720KB Double-Density disk,
so that they can be present on any disk.  Writing the Track Information Block and start-of-track
gaps at the DD data rate also ensures that at very high data rates, the head still has sufficient
time to switch to write mode, thus avoiding one of the many problems that arise when writing data
at very high data rates.

If formatting disks unbuffered, it is the programmer's responsibility to switch the data rate after
having written the Track Information Block, and several more bytes to allow the floppy encoding
pipeline to flush out the last byte of the Track Information Block.
This is all automatically managed if using the automatic track formatting function.

The inclusion of the TIB allows users to play and explore the possibilities of different data rates
on different drives and media, while still being automatically readable in all MEGA65s, because the
TIB allows the controller to switch to the correct data rate and encoding.  It is likely that over
time somewhat standardised formats will develop, quite likely in the range of 2MB to 3.5MB -- thus
approaching the capacity of ED media in ED drives, without the need for those drives or media.

\subsection{Formatting Disks}

Formatting disks is now possible with the 45IO27, either unbuffered or fully-automatic.
To format a track issue one of the following commands to \$D081:

\begin{itemize}
\item \$A0 -- Automatic format, with inter-sector gaps, and write pre-compensation disabled.
\item \$A1 -- Manual format, write-precompensation disabled.
\item \$A4 -- Automatic format, with inter-sector gaps, and write pre-compensation enable.
\item \$A5 -- Manual format, write-precompensation enabled.
\item \$A8 -- Automatic format, Amiga-style track-at-once, and write pre-compensation disabled.
\item \$AC -- Automatic format, Amiga-style track-at-once, and write pre-compensation enable.
\end{itemize}  
  
Manual formatting is not recommended, unless mastering track-at-once formatted disks for software
distribution, because of the relative complexity of doing so. Also, at the higher data rates,
bytes have to be delivered to the floppy controller as often as every 20 cycles, which requires
considerable care when writing the format routine.
For more information on manual formatting tracks, refer to the C64 Specifications Manual or the
C65 ROM DOS source code, for examples of manual formatting.

The automatic modes, in contrast, format a track with a single command, and are thus much easier to use,
and are recommended for general use.  Write pre-compensation should normally be enabled, as it is
required at higher data rates, and does not cause problems at lower data rates.

\subsection{Write Pre-Compensation}

Write pre-compensation is a family of algorithms used when writing high data-rate signals to floppy disks.
It is used to anticipate and cancel out the predictable component of timing variation of magnetic recording.
There are a variety of sources of this timing variation, which have been the subject of PhD theses, and a
lot of proprietary research by hard drive manufacturers.  What is important for us to understand is that adjacent
pulses (really magnetic inversions) get pushed together, if they are surrounded by longer pulses, or tend to
spread apart if surrounded by shorter pulses.

There are also other fascinatingly complex and difficult to predict factors, that cause things such as the ``negative
shift of mid-length pulses'', ``inverse F-distribution of pulse arrival times'' and goodness knows what else.  But
we shall leave those to the hard drive manufacturers.  We limit ourselves to the data pattern induced effect described
in the previous paragraph.

The 45GS27 supports two tunable coefficients for small and large corrections to this,
which are used with an internal look-up table.  However, this is all automatically handled if you enable write pre-compensation.
This allows data rates that much more closely approach the expected limit of HD media, although due to the other horrors of
magnetic media recording alluded to above, the actual limit is not reached.

\subsection{Buffered Sector Writing}

The 45IO27 can write to disk images that are located on the SD card,
or when using virtualised disk access.

To write a sector, you follow a similar process to reading, except
that you write \$84 to the command byte instead of \$40.  The \$80
indicates a write, and the \$04 activates write-precompensation.  This
is important when writing to real floppy disks, especially HD and ED
disks.  Write-precompensation causes bits to be written slightly early
or slightly late, using an algorithm that models how the magnetic
domains on a disk tend to move after being written.

If you do not wish
to use the sector buffer, but instead provide each byte one at a time
during the write operation, you must add \$01 to the command code.
However, this is not recommended on the MEGA65, because when writing
to the SD card or using virtualised disk images the entire sector
operation can happen instantaneously from the perspective of your
program.  This means that it is not possible to supply data reliably
when in this mode.  Thus apart from being less convenient, it is also
less reliable. 

Once a write operation has been triggered, the DRQ signal indicates
when you should provide the next byte if performing a byte-by-byte
write. Otherwise, it is assumed that you will have pre-filled the
sector buffer with the complete 512 bytes of data required.

To write to disks that contain Track Information Blocks\index{Track Information Block}\index{TIB},
you should first wait for the TIB to be read when changing tracks. This is done by waiting
for \$D6A9 (sectors per track from the TIB) to contain a non-zero value.

\subsection{F011 Floppy Controller Registers}

The following are the set of F011 compatibility registers of the 45IO47.
Note that registers related to the use of SD card based storage are found in the corresponding section below.

\input{regtable_FDC.C65}

The following registers apply to the 45IO27 only, i.e., are MEGA65 specific:

\input{regtable_FDC.MEGA65}

\section{SD Card Controller and F011 Virtualisation Functions}

For those situations where you do not wish to use real floppy disks,
the 45IO27 supports two complementary alternative modes:

\begin{itemize}
\item SD card Based Disk Image Access.
\item Virtualised Disk Image Access.
\end{itemize}

This is in addition to providing direct access to a dual-bus SD card
interface.

\subsection{SD card Based Disk Image Access}

The 45IO27 is both a floppy drive and SD card controller.
This enables it to transparently allow access to D81 disk images
stored on the SD card.  Further, because the controller is combined,
it is possible to still have the floppy drive step and spin as though
it were being used, providing considerable atmosphere and sense of
realism, even when using disk images.

The 45IO27 supports both 800KB standard D81 disk images, as well as
64MiB ``MEGA Images''.  While an operating system may impose
restrictions based on the name of a file, the 45IO27 is blind to these
requirements. Instead, it requires only that a {\tt contiguous} 800KiB
or 64MiB of the SD card is used to contain a disk image.

When a disk image is enabled, the corresponding set of sectors on the
SD card are effectively placed under user control, and the operating
system is no longer able to prevent the reading or writing of any of
those sectors.  Thus you should never enable access to an image that
is shorter than the required size, as it will otherwise allow the user
to unwittingly or maliciously access and/or modify data that is not
part of the image file.

For the same reason, only the hypervisor can change the sector number
where a disk image starts (the D?STARTSEC? signals), or allow the use
of disk images instead of the real floppy drive (USEREAL0 and USEREAL1
signals).  Once the Hypervisor has set the start sector of a disk
image, and cleared the USEREAL0 or USEREAL1 signal, the user can still
controll whether an access will go to the real floppy drive or to the
disk image by respectively clearing or setting the appropriate
signal.  For drive 0, this is D0IMG, and for drive 1, it is D1IMG.

There are also signals to control whether a disk image is an 800KiB
D81 image or a 64MiB MEGA Disk image, and whether a disk image is
present, and whether it is write protected. These are all located in
the \$D68B register. Because of the ability of manipulation of these
registers to corrupt or improperly access data, these signals are all
read-only, except from within the hypervisor.

The following table lists the registers that are used to control
access to disk images resident on the SD card:

\input{regtable_SDFDC.MEGA65}

\subsection{F011 Virtualisation}

In addition to allowing automatic read and write access to SD card
based D81 images, it is possible to connect a program to the serial
monitor interface that provides and accepts data as though it were the
floppy disk.

This is commonly used in a cross-development
environment, where you wish to frequently modify a disk image that is
used by a program you are developing -- without the need to
continually push new versions of the disk image on the MEGA65's
SD card first. It also has the added benefit that it allows you to
easily visualise which sectors are being read from and written to,
which can help speed up development and debugging of your program.

This function operates together with the MEGA65's Hypervisor by
triggering hyperrupts (that is, interrupts that activate the
Hypervisor).  There is then special code in the Hypervisor that
communicates with the {\tt m65} program via the serial monitor
interface.

If that all sounds rather complex, all you need to know is that to use
this function, you run the {\tt m65} utility with arguments like
{\tt -d image.d81}.  This should automatically establish the link with
the MEGA65.  If the BASIC interprettor stops responding, press the
reset button (not the power switch) on the left side of your MEGA65,
and it should return to the BASIC's {\tt READY.} prompt -- and if your
supplied disk image has a C65 auto-boot function, then it should
automatically start booting.

This function works very well if the host computer runs Linux, and
will allow loading at a speed of around 60KiB per second.  However, it
may be much slower on Windows or Apple OSX-based systems.

Of course to use this, you will also need an interface module and/or
cable to connect your cross-development system to the MEGA65's serial
monitor interface. This is most easily done using a Trenz TE0790-03
JTAG adapter and mini-USB cable.

More information on using this interface and the {\tt m65} tool can be
found in \bookvref{cha:transfer-and-debug-tools}.

\subsection{Dual-Bus SD Card Controller}

The 45IO27 contains a high-speed dual-bus SD card controller.  This
controller operates in SPI x1 mode at a clock speed of 20MHz,
providing a maximum throughput of approximately 2MiB/sec.  The quality
of the SD card makes a signficant difference to performance, with some
cards routinely delivering 1.7MiB/sec, while others 1MiB/sec or
less. Generally speaking, newer cards marketted as being suitable for
video recording perform better.  The controller supports SDHC cards,
and has experimental support for SDXC cards.  Legacy SD cards with a
capacity of 2GiB or less are not supported, as these use a different
addressing mode.

The SD controller itself is very simple to drive: Supply the sector
number in \$D861-\$D684, and then issue a read or write command to the
command register (\$D680).  The SD controller supports only
sector-based buffered operations, using the sector buffer. In
hypervisor mode, the sector buffer is located at \$FFD6E00 --
\$FFD6FFF, while when the computer is in normal operating mode, the SD
card and the floppy controller share a single address for both the
floppy drive and SD card sector buffers. Which buffer is visible at
that address is dictated by the BUFSEL signal. If it is 1, then the SD
card buffer is visible, while if it is 0, then the floppy drive sector
buffer is visible.  See also Sub-section
\vref{subsec:reading-from-disks-and-buffer-management} for further
discussion on the precise behaviour of this buffer with regard to
normal mode versus Hypervisor mode, and how it can also be mapped at
\$DE00. 

\subsubsection{Write Gate}

When writing a sector, you must, however, first open the ``write
gate''. This is a mechanism to prevent accidental corruption of data
on the SD card, as it requires two different values to be written to
the command register (\$D680) in quick succession: You have
approximately 1 milli second after opening the write gate to command
the write, before the write gate effectively closes again,
write-protecting the SD card until the write gate is opened again.
There are two different write gates: One for the master boot record
(sector 0), and the other for all other sectors, both of which are
listed in the command table below. This is designed to provide
additional protection to the very important master boot record sector
against programs accidentally calculating sector 0 as the target for
an ordinary write.

\subsubsection{Fill Mode}

Where you wish to fill sectors with a constant value, the 45IO27
supports a mode for this, so that you do not need to overwrite the
contents of the sector buffer. This is activated by placing the
desired fill value into the FILLVAL register (\$D686), and then
issuing the enable fill mode command (\$83), performing the sector
write operations, and then issuing the disable fill mode command
(\$84). 

\subsubsection{Selecting Among Multiple SD Cards}

The controller supports two SD Card interfaces, and it is possible to
have a card in both at the same time.  However, each card needs to be
reset and commanded separately.  Only one card can be commanded at a
time.  That said, it is possible to reset each card once, and then
switch between the cards to perform individual operations.

To select the first SD Card slot, write \$C0 to the SD Controller
Command Register (\$D680).  To select the second SD Card slot, write
\$C1 instead.

\subsubsection{SD Controller Command Table}

The SD controller supports the following commands that can be written
to the command register at \$D680:

\setlength{\tabcolsep}{3pt}
\begin{longtable}{|L{2.4cm}|L{8cm}|}
\hline
{\bf{Command}} & {\bf{Function}} \\
\hline
\endfirsthead
\multicolumn{2}{l@{}}{\ldots continued}\\
\hline
{\bf{Command}} & {\bf{Function}} \\
\endhead
\multicolumn{2}{l@{}}{continued \ldots}\\
 \endfoot
 \hline
\endlastfoot
\small \$00 (0) & Place SD card under reset (deprecated. Use command
\$10 instead) \\
 \hline
\small \$01 (1) & Release SD card from reset \\
 \hline
\small \$02 (2) & Read a sector from the SD card \\
 \hline
\small \$03 (3) & Write a single sector to the SD card \\
 \hline
\small \$04 (4) & Write the first sector of a multi-sector write to the SD card \\
 \hline
\small \$05 (5) & Write a subsequent sector of a multi-sector write to the SD card \\
 \hline
\small \$06 (6) & Write the final sector of a multi-sector write to the SD card \\
 \hline
\small \$0C (12) & Request flush of SD card write buffers (experimental) \\
 \hline
\small \$0E (14) & Pull SD handshake line low (debug only) \\
 \hline
\small \$0F (15) & Pull SD handshake line high (debug only) \\
 \hline
\small \$10 (16) & Place SD card under reset with flags set (preferred
method) \\
 \hline
\small \$11 (17) & Release SD card from reset (alternate method) \\
 \hline
\small \$40 (64) & Clear the SDHC/SDXC flag, selecting legacy SD card mode (deprecated)  \\
 \hline
\small \$41 (65) & Set the SDHC/SDXC mode flag   \\
 \hline
\small \$44 (68) & End force clearing of SD card state machine error
flag \\
\small \$45 (69) & Begin force clearing of SD card state machine error
flag \\
\small \$4D (77) & Open write-gate to sector 0 (master boot record)
for approximately 1 milli-second \\
 \hline
\small \$57 (87) & Open write-gate for all sectors > 0 for approximately 1 milli-second \\
 \hline
\small \$81 (129) & Enable mapping of the SD/FDC sector buffer at
\$DE00 -- \$DFFF \\
 \hline
\small \$82 (130) & Disable mapping of the SD/FDC sector buffer at
\$DE00 -- \$DFFF \\
 \hline
\small \$83 (131) & Enable SD Card Fill Mode \\
 \hline
\small \$84 (132) & Disable SD Card Fill Mode \\
 \hline
\small \$C0 (192) & Select SD Card Slot 0 \\
 \hline
\small \$C1 (193) & Select SD Card Slot 1 \\
 \hline
   \end{longtable}


Note that the hypervisor can enable or disable direct access to the SD
controller. The hypervisor operating system may provide a mechanism
for requesting permission to access the SD card controller, e.g., for
disk management utilities.

The SD card controller registers are as follows:

\input{regtable_SD.MEGA65}

\section{Touch Panel Interface}

Some MEGA65 variants include an LCD touch panel, primarily the
MEGAphone hand-held version of the MEGA65.  The touch interface
supports the detection of two simultaneous touch events.  Some
variants may also support gesture detection, however, this is still
very experimental.

The touch detection interface that is contained in the 45IO27 is
complemented by the on-screen-keyboard interface of the 4551 UART and
GPIO controller.  Refer to \bookref{sec:uartmisc} for further
information.  Of particular relevance are bit 7 of the registers \$D615 --
\$D617 which allow activating the on-screen keyboard interface,
selecting whether the on-screen keyboard is placed in the upper or
lower portion of the screen, and whether the primary or secondary
on-screen keyboard is displayed.

Direct connections between the 4551 and the 45IO27 combine information
about any currently displayed on-screen keyboard and the touch
interface controller, allowing synthetic keyboard events to be
automatically triggered when the on-screen keyboard portion of the
touch interface is pressed.  This allows the touch interface to be
used to drive the on-screen keyboard without requiring any support
from user programs. This works even when the on-screen keyboard is
moving during activation or transitioning between the top and bottom
of the screen.

As touch interfaces can require calibration, the 45IO27 allows for a
linear transformation of both the X and Y coordinates of a touch
event.  Specifically, there are scale (TCHXSCALE and TCHYSCALE) and
offset registers (TCHXDELTA and TCHYDELTA) that provide for this
transformation. It is also possible to flip the touch screen
coordinates in either or both the X and Y axes.  These calibration
registers also affect the operation of the on-screen keyboard.

It should also be noted that some touch interfaces do not have
constant horizontal or vertical resolution. For example, some panels
have a low horizontal resolution region in the middle of the panel,
which can require some care to accommodate.

To detect the primary touch event, the TOUCH1XLSB, TOUCH1XMSB, TOUCH1YLSB,
TOUCH1YMSB registers can be read.  Similar registers exist for the 2nd
touch event: TOUCH2XLSB, TOUCH2XMSB, TOUCH2YLSB, TOUCH2YMSB. Each
touch event has a signle bit flag that indicates whether the touch
event is currently valid: the EV1 and EV2 bits of the register
\$D6B0.  There are also corresponding bit-fields that indicate whether a
given touch event has been made or released, allowing the detection of
when a finger both makes and breaks contact with the screen.  The
UPDN1 and UPDN2 signals provide this information.  Binary values of 01 and
10, respectively indicate if the finger has been removed or pressed
against the touch panel.  Values of 00 and 11 mean that a finger is
either being held or not being held against the touch panel.

The
primary touch event is also fed into the lightpen\index{light pen} input of the VIC-IV,
and can be detected using the normal light pen registers of the VIC-IV.

The registers for the touch panel interface are as follows:

\input{regtable_TOUCH.MEGA65}

\section{Audio Support Functions}
\index{cross-bar switch, audio}
\index{mixer, audio}
\index{audio mixer}
\index{audio cross-bar switch}
The 45IO27 provides the primary interface into the MEGA65's full
cross-bar audio mixer. This includes the interface for reading or
modifying the mixer co-efficients, as well as accessing the mixer
feedback registers, and setting the 16-bit digital sample values that
are two of the input channels into the audio mixer.

The audio mixer consists of 128 coefficients, each of which is 16
bits.  Each audio output channel, e.g., left speaker, right speaker,
left headphone, right headphone, cellular modem 1 (MEGAphone models
only) and so on, are generated by taking each of the audio input
channels, multiplying them by the appropriate coefficient, and adding
it to the total output of the audio output channel.

Because each
audio output channel has its own set of coefficients that are applied
to all of the audio input channels, this means that it is possible to
produce totally different audio out each audio channel: For example,
it is possible to play your favourite quadrophonic SID music out of
the headphones while rick-rolling passers by with Amiga-style MOD
audio.  This is why the audio mixer is refered to as a {\tt full
  cross-bar} mixer, because there are no restrictions on how you mix
each audio output channel.  In this regard, it is very similar to a
full-function audio desk, allowing different mixing levels for
different speakers.

Because the audio coefficients are 16 bits each, each one is formed
using two successive bytes of the audio co-efficient space.  Changes
to the audio coefficients take effect immediately, so care should be
taken when changing coefficients to avoid audible clicks and pops.
Also, you must allow 32 cycles to elapse before changing the selected
audio coefficient, as otherwise the change may be discarded if the
audio mixer accumulator has not had time to re-visit that coefficient.

The audio sources on the MEGA65 and MEGAphone devices are as follows:

\setlength{\tabcolsep}{3pt}
\begin{longtable}{|L{2.4cm}|L{8cm}|}
\hline
{\bf{Input Channel ID}} & {\bf{Connection}} \\
\hline
\endfirsthead
\multicolumn{2}{l@{}}{\ldots continued}\\
\hline
{\bf{Input Channel ID}} & {\bf{Connection}} \\
\endhead
\multicolumn{2}{l@{}}{continued \ldots}\\
 \endfoot
 \hline
\endlastfoot
\small \$0 (0) & Left SIDs \\
 \hline
\small \$1 (1) & Right SIDs \\
 \hline
\small \$2 (2) & Modem Bay 1 (MEGAphone only) \\
 \hline
\small \$3 (3) & Modem Bay 2 (MEGAphone only) \\
 \hline
\small \$4 (4) & Bluetooth(tm) Left \\
 \hline
\small \$5 (5) & Bluetooth(tm) Right \\
 \hline
\small \$6 (6) & Headphone Interface 1 \\
 \hline
\small \$7 (7) & Headphone Interface 2 \\
 \hline
\small \$8 (8) & Digital audio Left \\
 \hline
\small \$9 (9) & Digital audio Right \\
 \hline
\small \$A (10) & MEMs Microphone 0 (Nexys4 and MEGAphone only) \\
 \hline
\small \$B (11) & MEMs Microphone 1 (MEGAphone only) \\
 \hline
\small \$C (12) & MEMs Microphone 2 (MEGAphone only) \\
 \hline
\small \$D (13) & MEMs Microphone 3 (MEGAphone only) \\
 \hline
\small \$E (14) & Headphone jack microphone (Nexys4 and MEGAphone only) \\
 \hline
\small \$F (15) & OPL-compatible FM audio ({\em shares co-efficient
  with input 14}) \\
 \hline
\end{longtable}

The OPL-compatible FM audio which is on source 15 is controlled by the
coefficient for source 14.  This is because the coefficient for source
15 provides the master volume level for each output.

The audio cross-bar mixer supports the following eight output
channels:

\setlength{\tabcolsep}{3pt}
\begin{longtable}{|L{2.4cm}|L{8cm}|}
\hline
{\bf{Output Channel ID}} & {\bf{Connection}} \\
\hline
\endfirsthead
\multicolumn{2}{l@{}}{\ldots continued}\\
\hline
{\bf{Output Channel ID}} & {\bf{Connection}} \\
\endhead
\multicolumn{2}{l@{}}{continued \ldots}\\
 \endfoot
 \hline
\endlastfoot
\small \$0 (0) & Left Primary Speaker (digital audio on MEGA65 R2/R3,
physical speaker on MEGAphone, headphone jack audio on Nexys4) \\
 \hline
\small \$1 (1) & Right Primary Speaker (digital audio on MEGA65 R2/R3,
physical speaker on MEGAphone, headphone jack audio on Nexys4) \\
 \hline
\small \$2 (2) & Modem Bay 1 audio output (MEGAphone only) \\
 \hline
\small \$3 (3) & Modem Bay 2 audio output (MEGAphone only) \\
 \hline
\small \$4 (4) & Bluetooth Left Audio (MEGAphone only) \\
 \hline
\small \$5 (5) & Bluetooth Right Audio (MEGAphone only) \\
 \hline
\small \$6 (6) & Headphone Left output (MEGA65 R2/R3 and MEGAphone
only.  On Nexys4 boards the primary speaker drives the 3.5mm jack) \\
 \hline
\small \$7 (7) & Headphone Right output (MEGA65 R2/R3 and MEGAphone
only.  On Nexys4 boards the primary speaker drives the 3.5mm jack) \\
 \hline
\end{longtable}

To determine the coefficient register number for a given source and
output, multiply the output number by 32 and multiply the source
number by 2.  This will be the register number for the LSB of the
16-bit coefficient.  The MSB will be the next register. For example,
to set the coefficient of the right SIDs to the 2nd modem bay audio
output, the coefficient would be $32\times 3 + 1 \times 2 = 96 + 2 =
98$.  



XXX - mixer stuff
XXX - mixer feedback registers
XXX - Left and right digi
XXX - CPU register for selecting PWM/PDM

\input{regtable_AUDIO.MEGA65}


\section{Miscellaneous IO Functions}

