\chapter{45GS02 \& 6502 Instruction Sets}

The 45GS02 CPU is able to operate in native mode, where it
is compatible with the CSG 4510, and in 6502 compatibility mode,
where 6502 undocumented instructions, also known as illegal
instructions, are supported for compatibility.

When in 4510 compatibility mode, the 45GS02 also supports a number
of extensions through {\em compound instructions}. These work be prefixing
the desired instruction's opcode with one or more {\em prefix bytes}, which
represent sequences of instructions that should not normally occur.  For example,
two \stw{NEG} instructions in a row acts as a prefix to tell the 45GS02 that the
following instruction will operate on 32 bits of data, instead of the usual 8 bits
of data.  This means that a 45GS02 instruction stream can be readily decoded or disassembled,
without needing to set special instruction length flags, as is the case with the 65816
family of microprocessors. The trade-off is increased execution time, as the 45GS02 must
skip over the prefix bytes.

The remainder of this chapter introduces the addressing modes, instructions, opcodes and
instruction timing data of the 45GS02, beginning with 6502 compatibility mode, before
moving on to 4510 compatibility mode, and the 45GS02 extensions.

\section{Addressing Modes}

The 45GS02 supports 36 different addressing modes, which are explained below.
Many of these are very similar to one another, being variations of the normal 6502
or 65CE02 addressing modes, except that they accept either
32-bit pointers, operate on 32-bits of data, or both.

\subsection{Implied}

In this mode, there are no operands, as the precise function of the instruction is
implied by the instruction itself.  For example, the \screentext{INX} instruction increments
the X Register.

\subsection{Accumulator}

In this mode, the Accumulator is the operand. This is typically used to shift,
rotate or modify the value of the Accumulator Register in some way.  For example,
\screentext{INC A} increments the value in the Accumulator Register.

\subsection{Q Pseudo Register}

In this mode, the Q Pseudo Register is the operand. This is typically used to shift,
rotate or modify the value of the Q Pseudo Register in some way.  For example,
\screentext{ASL Q} shifts the value in the Q Pseudo Register left one bit.

Remember that the Q Pseudo Register is simply the A, X, Y and Z registers acting together
as a pseudo 32-bit register, where A contains the least significant bits, and Z the
most significant bits.  There are some cases where using a Q mode instruction can be
helpful for operating on the four true registers.

\subsection{Immediate Mode}

In this mode, the argument to the instruction is a value that is used directly.
This is indicated by proceeding the value with a \# character. Most assemblers allow
values to be entered in decimal, or in hexadecimal by preceding the value with a \$ sign,
in binary, by preceding the value with a \% sign.  For example, to set the Accumulator
Register to the value 5, you could use the following:

\begin{tcolorbox}[colback=black,coltext=white]
\verbatimfont{\codefont}
\begin{verbatim}
LDA #5
\end{verbatim}
\end{tcolorbox}

The immediate argument is encoded as a single byte following the instruction.  For the above
case, the instruction stream would contain \$A9, the opcode for LDA immediate mode, followed
by \$05, the immediate operand.

\subsection{Immediate Word Mode}

In this mode, the argument is a 16-bit value that is used directly. There is only one instruction
which uses this addressing mode, \screentext{PHW}.  For example, to push the word \$1234
onto the stack, you could use:

\begin{tcolorbox}[colback=black,coltext=white]
\verbatimfont{\codefont}
\begin{verbatim}
PHW #$1234
\end{verbatim}
\end{tcolorbox}

The low byte of the immediate value follows the opcode of the instruction.  The high byte of the
immediate value then follows that.  For the above example, the instruction stream would thus
be \$F4 \$34 \$12.

\subsection{Base Page (Zero-Page) Mode}

In this mode, the argument is an 8-bit address.  The upper 8-bits of the address are taken from
the Base Page Register.  On 6502 processors, there is no Base Page Register, and instead, the
upper 8-bits are always set to zero -- hence the name of this mode on the 6502: Zero-Page. On
the 45GS02, it is possible to move this ``Zero-Page'' to any page in the processor's 64KB view
of memory by setting the Base Page Register using the \screentext{TAB} instruction. Base Page
Mode allows faster access to a 256 region of memory, and uses less instruction bytes to do so.

The argument is encoded as a single byte that immediately follows the instruction opcode. For
example, \screentext{LDA \$12} would read the value stored in location \$12 in the Base Page,
and put it into the Accumulator Register.  The instruction byte stream for this would be
\$85 \$12.

\subsection{Base Page (Zero-Page) Quad Mode}

In this mode, the argument is an 8-bit address.  The upper 8-bits of the address are taken from
the Base Page Register.  On 6502 processors, there is no Base Page Register, and instead, the
upper 8-bits are always set to zero -- hence the name of this mode on the 6502: Zero-Page. On
the 45GS02, it is possible to move this ``Zero-Page'' to any page in the processor's 64KB view
of memory by setting the Base Page Register using the \screentext{TAB} instruction. Base Page
Mode allows faster access to a 256 region of memory, and uses less instruction bytes to do so.

The argument is encoded as a single byte that immediately follows the instruction opcode. For
example, \screentext{LDQ \$12} would read the value stored in locations \$12 -- \$15 in the Base Page,
and put them into the Q Pseudo Register.  

\subsection{Base Page (Zero-Page) X Indexed Mode}

This mode is identical to Base Page Mode, except that the address is formed by taking the
argument, and adding the value of the X Register to it.  In 6502 mode, the result will always
be in the Base Page, that is, any carry due to the addition from the low byte into the high byte
of the address will be ignored.  The encoding for this addressing mode is identical to Base Page
Mode.

\subsection{Base Page (Zero-Page) Quad X Indexed Mode}

This mode is identical to Base Page Quad Mode, except that the address is formed by taking the
argument, and adding the value of the X Register to it.  In 6502 mode, the result will always
be in the Base Page, that is, any carry due to the addition from the low byte into the high byte
of the address will be ignored.  The encoding for this addressing mode is identical to Base Page Quad
Mode.

\subsection{Base Page (Zero-Page) Y Indexed Mode}

This mode is identical to Base Page Mode, except that the address is formed by taking the
argument, and adding the value of the Y Register to it.  In 6502 mode, the result will always
be in the Base Page, that is, any carry due to the addition from the low byte into the high byte
of the address will be ignored.  The encoding for this addressing mode is identical to Base Page
Mode.

\subsection{Base Page (Zero-Page) Base Y Indexed Mode}

This mode is identical to Base Page Quad Mode, except that the address is formed by taking the
argument, and adding the value of the Y Register to it.  In 6502 mode, the result will always
be in the Base Page, that is, any carry due to the addition from the low byte into the high byte
of the address will be ignored.  The encoding for this addressing mode is identical to Base Page Quad
Mode.

\subsection{Base Page (Zero-Page) Z Indexed Mode}

This mode is identical to Base Page Mode, except that the address is formed by taking the
argument, and adding the value of the Z Register to it.  In 6502 mode, the result will always
be in the Base Page, that is, any carry due to the addition from the low byte into the high byte
of the address will be ignored.  The encoding for this addressing mode is identical to Base Page
Mode.

\subsection{Base Page (Zero-Page) Quad Z Indexed Mode}

This mode is identical to Base Page Quad Mode, except that the address is formed by taking the
argument, and adding the value of the Z Register to it.  In 6502 mode, the result will always
be in the Base Page, that is, any carry due to the addition from the low byte into the high byte
of the address will be ignored.  The encoding for this addressing mode is identical to Base Page Quad
Mode.

\subsection{Absolute Mode}

In this mode, the argument is an 16-bit address.  The low 8-bits of the address are taken from
the byte immediately following the instruction opcode. The upper 8-bits are taken from the
byte following that.  For example, the instruction \screentext{LDA \$1234}, would read the
memory location \$1234, and place the read value into the Accumulator Register.  This would
be encoded as \$AD \$34 \$12.

\subsection{Absolute Quad Mode}

In this mode, the argument is an 16-bit address.  The low 8-bits of the address are taken from
the byte immediately following the instruction opcode. The upper 8-bits are taken from the
byte following that.  For example, the instruction \screentext{LDQ \$1234}, would read the
memory locations \$1234 -- \$1237, and place the read values into the Q Pseudo Register.  This would
be encoded as \$42 \$42 \$AD \$34 \$12.

\subsection{Absolute X Indexed Mode}

This mode is identical to Absolute Mode, except that the address is formed by taking the
argument, and adding the value of the X Register to it.  If the indexing causes the address
to cross a page boundary, i.e., if the upper byte of the address changes, this may incur a
1 cycle penalty, depending on the processor mode and speed setting.
The encoding for this addressing mode is identical to Absolute Mode.

\subsection{Absolute Quad X Indexed Mode}

This mode is identical to Absolute Quad Mode, except that the address is formed by taking the
argument, and adding the value of the X Register to it.  If the indexing causes the address
to cross a page boundary, i.e., if the upper byte of the address changes, this may incur a
1 cycle penalty, depending on the processor mode and speed setting.
The encoding for this addressing mode is identical to Absolute Quad Mode.

\subsection{Absolute Y Indexed Mode}

This mode is identical to Absolute Mode, except that the address is formed by taking the
argument, and adding the value of the Y Register to it.  If the indexing causes the address
to cross a page boundary, i.e., if the upper byte of the address changes, this may incur a
1 cycle penalty, depending on the processor mode and speed setting.
The encoding for this addressing mode is identical to Absolute Mode.

\subsection{Absolute Quad Y Indexed Mode}

This mode is identical to Absolute Quad Mode, except that the address is formed by taking the
argument, and adding the value of the Y Register to it.  If the indexing causes the address
to cross a page boundary, i.e., if the upper byte of the address changes, this may incur a
1 cycle penalty, depending on the processor mode and speed setting.
The encoding for this addressing mode is identical to Absolute Quad Mode.

\subsection{Absolute Z Indexed Mode}

This mode is identical to Absolute Mode, except that the address is formed by taking the
argument, and adding the value of the Z Register to it.  If the indexing causes the address
to cross a page boundary, i.e., if the upper byte of the address changes, this may incur a
1 cycle penalty, depending on the processor mode and speed setting.
The encoding for this addressing mode is identical to Absolute Mode.

\subsection{Absolute Quad Z Indexed Mode}

This mode is identical to Absolute Quad Mode, except that the address is formed by taking the
argument, and adding the value of the Z Register to it.  If the indexing causes the address
to cross a page boundary, i.e., if the upper byte of the address changes, this may incur a
1 cycle penalty, depending on the processor mode and speed setting.
The encoding for this addressing mode is identical to Absolute Quad Mode.

\subsection{Absolute Indirect Mode}

In this mode, the 16-bit argument is the address that points to, i.e., contains the
address of actual byte to read.  For example, if memory location \$1234 contains \$78
and memory location \$1235 contains \$56, then \screentext{JMP (\$1234)} would jump
to address \$5678.  The encoding for this addressing mode is identical to Absolute Mode.

\subsection{Absolute Indirect X-Indexed Mode}

In this mode, the 16-bit argument is the address that points to, i.e., contains the
address of actual byte to read. It is identical to Absolute Indirect Mode, except that
 the value of the X Register is added to the pointer address.
For example, if the X Register contains the value \$04, memory location \$1238 contains \$78
and memory location \$1239 contains \$56, then \screentext{JMP (\$1234,X)} would jump
to address \$5678.
The encoding for this addressing mode is identical to Absolute Mode.

\subsection{Base Page Indirect X-Indexed Mode}

This addressing mode is identical to Absolute Indirect X-Indexed Mode, except that the address
of the pointer is formed from the Base Page Register (high byte) and the 8-bit operand (low byte).
The encoding for this addressing mode is identical to Base Page Mode.

\subsection{Base Page Quad Indirect X-Indexed Mode}

This addressing mode is identical to Base PAge Indirect X-Indexed Mode, except that the address
of the pointer is formed from the Base Page Register (high byte) and the 8-bit operand (low byte).
The encoding for this addressing mode is identical to Base Page Quad Mode.

\subsection{Base Page Indirect Y-Indexed Mode}

This addressing mode differs from the X-Indexed Indirect modes, in that the Y Register is
added to the address that is read from the pointer, instead of being added to the pointer.
This is a very useful mode, that is frequently because it effectively provides access to
``the Y-th byte of the memory at the address pointed to by the operand.'' That is, it de-references
a pointer.
The encoding for this addressing mode is identical to Base Page Mode.

\subsection{Base Page Quad Indirect Y-Indexed Mode}

This addressing mode is identical to the Base Page Indirect Y-Indexed Mode, except that
32-bits of data are operated on. The encoding for this addressing mode is identical to
Base Page Mode, except that it is prefixed by \$42, \$42.

\subsection{Base Page Indirect Z-Indexed Mode}

This addressing mode differs from the X-Indexed Indirect modes, in that the Z Register is
added to the address that is read from the pointer, instead of being added to the pointer.
This is a very useful mode, that is frequently because it effectively provides access to
``the Z-th byte of the memory at the address pointed to by the operand.'' That is, it de-references
a pointer.
The encoding for this addressing mode is identical to Base Page Mode.

That is, it is equivalent to the Base Page Indirect Y-Indexed Mode.

\subsection{Base Page Quad Indirect Z-Indexed Mode}

This addressing mode is identical to the Base Page Indirect Z-Indexed Mode, except that
32-bits of data are operated on. The encoding for this addressing mode is identical to
Base Page Mode, except that it is prefixed by \$42, \$42.

\subsection{32-bit Base Page Indirect Z-Indexed Mode}

This mode is formed by preceding a Base Page Indirect Z-Indexed Mode instruction with
the {NOP} instruction (opcode \$EA).  This causes the 45GS02 to read a 32-bit address instead
of a 16-bit address from the Base Page address indicated by its operand.  The Z index is added
to that pointer.  Importantly, the 32-bit address does not refer to the processor's current 64KB
view of memory, but rather to the 45GS02's true 28-bit address space. This allows easy access
to any memory, without requiring the use of complex bank-switching or DMA operations.

For example, if addresses \$12 to \$15 contained the bytes \$20, \$D0, \$FF, \$0D, and the
Z index contained the value \$01, the following instruction sequence would change the screen
colour to blue:

\begin{tcolorbox}[colback=black,coltext=white]
\verbatimfont{\codefont}
\begin{verbatim}
LDA #$06
LDZ #$01
STA [$12],Z
\end{verbatim}
\end{tcolorbox}

\subsection{32-bit Base Page Indirect Quad Z-Indexed Mode}

This addressing mode is identical to the 32-bit Base Page Indirect Z-Indexed Mode,
except that it operates on 32-bits of data at the 32-bit address formed by the argument,
in comparison to 32-bit Base Page Indirect Z-Indexed Mode which operates on only 8 bits
of data.   The encoding of this addressing mode is \$42, \$42, \$EA, followed by the
natural 6502 opcode for the instruction being performed.


\subsection{32-bit Base Page Indirect Mode}

This mode is formed by preceding a Base Page Indirect Z-Indexed Mode instruction with
the {NOP} instruction (opcode \$EA).  This causes the 45GS02 to read a 32-bit address instead
of a 16-bit address from the Base Page address indicated by its operand.
Importantly, the 32-bit address does not refer to the processor's current 64KB
view of memory, but rather to the 45GS02's true 28-bit address space. This allows easy access
to any memory, without requiring the use of complex bank-switching or DMA operations.

For example, if addresses \$12 to \$15 contained the bytes \$20, \$D0, \$FF, \$0D,
the following instruction sequence would change the screen border colour to blue:

\begin{tcolorbox}[colback=black,coltext=white]
\verbatimfont{\codefont}
\begin{verbatim}
LDA #$06
STA [$12]
\end{verbatim}
\end{tcolorbox}

NOTE: The ACME assembler is the only assembler that currently supports this addressing mode.
For other assemblers, you can achieve the same result by using a \stw{NOP} instruction to
prefix the equivalent {\em Base Page Indirect, Indexed by Z} instruction.

\begin{tcolorbox}[colback=black,coltext=white]
\verbatimfont{\codefont}
\begin{verbatim}
LDA #$06
NOP
STA ($12),Z
\end{verbatim}
\end{tcolorbox}


The encoding for this addressing mode is identical to Base Page Mode.

\subsection{32-bit Base Page Indirect Mode}

This addressing mode is identical to the 32-bit Base Page Indirect Mode,
except that it operates on 32-bits of data at the 32-bit address formed by the argument,
in comparison to 32-bit Base Page Indirect Mode which operates on only 8 bits
of data.   The encoding of this addressing mode is \$42, \$42, \$EA, followed by the
natural 6502 opcode for the instruction being performed.

\subsection{Stack Relative Indirect, Indexed by Y}

This addressing mode is similar to Base Page Indirect Y-Indexed Mode,
except that instead of providing the address of the pointer in the
Base Page, the operand indicates the offset in the stack to find the
pointer. This addressing mode effectively de-references a pointer that
has been placed on the stack, e.g., as part of a function call from a
high-level language.  It is encoded identically to the Base Page Mode.

\subsection{Relative Addressing Mode}

In this addressing mode, the operand is an 8-bit signed offset to the
current value of the Program Counter (PC). It is used to allow branches
to encode the nearby address at which execution should proceed if the
branch is taken.

\subsection{Relative Word Addressing Mode}

This addressing mode is identical to Relative Addressing Mode, except that
the address offset is a 16-bit value. This allows a relative branch or jump
to any location in the current 64KB memory view.  This makes it possible
to write software that is fully relocatable, by avoiding the need for absolute
addresses when calling routines.

\section{6502 Instruction Set}

NOTE: The mechanisms for switching from 4510 to 6502 CPU personality
have yet to be finalised.

NOTE: Not all 6502 illegal opcodes are currently implemented.

\subsection{Opcode Map}

\begin{center}
\rotatebox{270}{
\input{6502-opcodes}
}
\end{center}

\subsection{Instruction Timing}

The following table summarises the base instruction timing for 6502 mode.
Please also read the information for 4510 mode, as it discusses a number
of important factors that affect these figures.

\begin{center}
\rotatebox{270}{
\input{6502-cycles}
}
\end{center}

\subsection{Addressing Mode Table}

\begin{center}
  \rotatebox{270}{
    \scriptsize
\input{6502-modes}
}
\end{center}

\subsection{Official And Unintended Instructions}

The 6502 opcode matrix has a size of 16 x 16 = 256 possible opcodes.
Those, that are officially documented, form the set of the
\textcolor{blue}{legal} instructions.
All instructions of this legal set are headed by a blue coloured mnemonic.

The remaining opcodes form the set of the
\textcolor{red}{unintended} instructions
(sometimes called "illegal" instructions).
For the sake of completeness these are documented too.
All instructions of the unintended set are headed by a red coloured mnemonic.

The unintended instructions are implemented in the 6502 mode,
but are not guaranteed to produce exactly the same results as on other
CPU's of the 65xx family. Many of these instructions are known to
be unstable, even running on old hardware.

\addtocontents{toc}{\protect\setcounter{tocdepth}{1}}
\input{instructionset-6502}
\addtocontents{toc}{\protect\setcounter{tocdepth}{5}}


\section{4510 Instruction Set}

\subsection{Opcode Map}

\begin{center}
\rotatebox{270}{
  \input{4510-opcodes}
  }
\end{center}

\subsection{Instruction Timing}

The following table lists the base cycle count for each opcode.
Note that the number of cycles depends on the speed setting of the
processor: Some instructions take more or fewer cycles when the
processor is running at full-speed, or a C65 compatibility 3.5MHz speed,
or at C64 compatibility 1MHz/2MHz speed.  More detailed information on
this is listed under each each instruction's information, but the high-level
view is:

\begin{itemize}
\item When the processor is running at 1MHz, all instructions take at least
  two cycles, and dummy cycles are re-inserted into Read-Modify-Write instructions,
  so that all instructions take exactly the same number of cycles as on a 6502.
\item The Read-Modify-Write instructions and all instructions that read a value from
  memory all require an extra cycle when operating at full speed, to allow signals
  to propagate within the processor.
\item The Read-Modify-Write instructions require an additional cycle if the operand
  is \$D019, as the dummy write is performed in this case.
  This is to improve compatibility with C64 software that frequently uses this
  ``bug'' of the 6502 to more rapidly acknowledge VIC-II interrupts.
\item Page-crossing and branch-taking penalties do not apply when the processor is
  running at full speed.
\item Many instructions require fewer cycles when the processor is running at full
  speed, as generally most non-bus cycles are removed. For example, Pushing and Pulling
  values to and from the stack requires only 2 cycles, instead of the 4 that that the
  6502 requires for these instructions.
\end{itemize}

Note that it is possible that further changes to processor timing will occur.

Similar issues apply to when the processor is in 6502 mode.

\begin{center}
\rotatebox{270}{
\input{4510-cycles}
}
\end{center}

\subsection{Addressing Mode Table}

\begin{center}
  \rotatebox{270}{
    \scriptsize
\input{4510-modes}
}
\end{center}


\addtocontents{toc}{\protect\setcounter{tocdepth}{1}}
\input{instructionset-4510}
\addtocontents{toc}{\protect\setcounter{tocdepth}{5}}

\section{45GS02 Compound Instructions}

As the 4510 has no unallocated opcodes, the 45GS02 uses compound instructions
to implement its extension.  These compound instructions consist of one or
more single byte instructions placed immediately before a conventional
instruction.  These prefixes instruct the 45GS02 to treat the following instruction
differently, as described in \bookvref{cha:cpu}.

% Opcodes can have multiple meanings, so no table

% No instruction timing table for compounds, as it can't
% be correctly drawn.

% Same goes for mode list.


\addtocontents{toc}{\protect\setcounter{tocdepth}{1}}
\input{instructionset-45GS02}
\addtocontents{toc}{\protect\setcounter{tocdepth}{5}}

