\chapter{KERNAL Jump Table}
\label{cha:kernal-jump-table}

\section{Using the KERNAL Jump Table}

The \emph{KERNAL} is the centerpiece of the MEGA65 operating system. It is responsible for setting up the operating system when the computer is switched on or reset, interfacing with peripherals, invoking disk operations, managing input and output streams, and driving key features of the screen editor and BASIC. While it is running, the KERNAL owns the CPU memory map and interrupt handlers, and reserves regions of memory to support its features.

The KERNAL provides an \emph{Application Programming Interface} (API) so that machine code programs can take advantage of its features, such as to read a file from disk, by calling subroutines. A program accesses these subroutines by calling addresses in a \emph{jump table}, a region of ROM code that is guaranteed by the API to always access the same subroutines in the same way, even across different versions of the ROM code. The program uses \texttt{jsr} or \texttt{jmp} (as required by the routine) using the jump table address as an absolute target. Memory at the jump table address contains a \texttt{jmp} instruction to the appropriate internal address.

For example, to write a character to the output stream (the screen terminal, by default), a program sets the accumulator (A) register to the PETSCII code to write, then calls the CHROUT subroutine with the jump table address \$FFD2:

\begin{asmcode}
    lda #$41
    jsr $ffd2
\end{asmcode}

\subsection{Prerequisites of Jump Table Routines}

All KERNAL jump table routines expect the following conditions to be true when called:

\begin{enumerate}
\item The 16-bit address range \$E000 -- \$FFFF must be mapped to the KERNAL ROM at 3.E000 -- 3.FFFF.
\item The 16-bit address range \$0000 -- \$1FFF must be mapped to KERNAL internal variable memory at 0.0000 -- 0.1FFF.
\item The CPU base page register (B) must be \$00.
\end{enumerate}

Some jump table routines have additional prerequisites. See the reference below.

The KERNAL BASIC and SYS memory maps maintain these conditions by default. If a machine code program does not change the MAP or B registers, it can call KERNAL jump table routines without additional set-up.

The KERNAL makes use of other regions of memory. See \bookref{cha:memory}.

Unless otherwise noted, a program must assume that a call to a KERNAL routine overwrites data in all CPU registers and value-related status flags. Implications for the stack and mode-related status flags (interrupts) are noted explicitly.

\subsection{History and Compatibility}

The MEGA65 KERNAL jump table is based on, and compatible with, the Commodore 65 jump table. This API is derived from, but \emph{not} entirely compatible with, the KERNAL APIs of the Commodore 64 and Commodore 128.

This documentation adopts the names of the routines from Commodore 65 documentation where possible, and invents unique names for routines new to the MEGA65. A routine with a name similar to C64 or C128 documentation may \emph{not} be API compatible. Differences are noted in this reference, where known.

The MEGA65's ``GO64'' mode uses the C64 KERNAL. Refer to a C64 programming reference for information on how to use the C64 KERNAL.

For a complete history of the KERNAL jump table across all Commodore computers, see: \url{https://www.pagetable.com/?p=926}

\newpage
\section{MEGA65 KERNAL Jump Table Quick Reference}

The following is a quick reference for the MEGA65 KERNAL jump table, in address order. See the full reference for details and examples.

\begin{longtable}{|L{2cm}|p{2cm}|p{2cm}|L{6cm}|}
\hline
\textbf{Address} & \textbf{Category} & \textbf{Name} & \textbf{Description} \\
\hline
\endfirsthead
\multicolumn{4}{l@{}}{\ldots continued}\\
\hline
\textbf{Address} & \textbf{Category} & \textbf{Name} & \textbf{Description} \\
\hline
\endhead
\multicolumn{4}{l@{}}{continued \ldots}\\
\endfoot
\hline
\endlastfoot

\$FF2F & OS & VERSIONQ & Load the ROM version number into Q \\
\hline
\$FF32 & OS & RESET\_RUN & Reset the computer in various ways \\
\hline
\$FF35 & Editor & CURSOR & Enable or disable the cursor \\
\hline
\$FF38 & Reserved & n/a & RESERVED, DO NOT USE \\
\hline
\$FF3B & Files & SAVEFL & Save to file, with flags \\
\hline
\$FF3E & Reserved & n/a & RESERVED, DO NOT USE \\
\hline
\$FF41 & I/O & GETIO & Read the current input and output devices \\
\hline
\$FF44 & Files & GETLFS & Read file, device, secondary address \\
\hline
\$FF47 & Editor & SYSFLAGS & Read/set keyboard locks \\
\hline
\$FF4A & Editor & ADDKEY & Add a character to the soft keyboard input buffer \\
\hline
\$FF4D & Reserved & SPIN\_SPOUT & Set up fast serial ports \\
\hline
\$FF50 & Files & CLOSE\_ALL & Close all files on a device \\
\hline
\$FF53 & OS & C64MODE & Reset to GO64 mode \\
\hline
\$FF56 & OS & MonitorCall & Invoke monitor \\
\hline
\$FF59 & OS & BOOT\_SYS & Boot an alternate system from disk \\
\hline
\$FF5C & OS & PHOENIX & Call cartridge cold start or disk boot loader \\
\hline
\$FF5F & Files & LKUPLA & Search for logical file number in use \\
\hline
\$FF62 & Files & LKUPSA & Search for secondary address in use \\
\hline
\$FF65 & Editor & SWAPPER & Toggle between 40x25 and 80x25 text modes \\
\hline
\$FF68 & Editor & PFKEY & Program an editor function key \\
\hline
\$FF6B & Files & SETBNK & Set bank for I/O and filename memory \\
\hline
\$FF6E & Memory & JSRFAR & Call subroutine in any bank \\
\hline
\$FF71 & Memory & JMPFAR & Jump to address in any bank \\
\hline
\$FF74 & Memory & LDA\_FAR & Read a byte from an address in any bank \\
\hline
\$FF77 & Memory & STA\_FAR & Store a byte to an address in any bank \\
\hline
\$FF7A & Memory & CMP\_FAR & Compare a byte with an address in any bank \\
\hline
\$FF7D & Tools & PRIMM & Print an inline null-terminated short string \\
\hline
\$FF81 & Editor & CINT & Initialise screen editor \\
\hline
\$FF84 & OS & IOINIT & Initialise I/O devices \\
\hline
\$FF87 & OS & RAMTAS & Initialise RAM and buffers \\
\hline
\$FF8A & OS & RESTOR & Initialise KERNAL vector table \\
\hline
\$FF8D & OS & VECTOR & Read/set KERNAL vector table \\
\hline
\$FF90 & OS & SETMSG & Enable/disable KERNAL messages \\
\hline
\$FF93 & Serial & SECND & Send secondary address to listener \\
\hline
\$FF96 & Serial & TKSA & Send secondary address to talker \\
\hline
\$FF99 & Reserved & MEMTOP & RESERVED, DO NOT USE \\
\hline
\$FF9C & Reserved & MEMBOT & RESERVED, DO NOT USE \\
\hline
\$FF9F & Tools & KEY & Scan keyboard \\
\hline
\$FFA2 & OS & MONEXIT & Monitor's exit to BASIC \\
\hline
\$FFA5 & Serial & ACPTR & Accept a byte from talker \\
\hline
\$FFA8 & Serial & CIOUT & Send a byte to listener \\
\hline
\$FFAB & Serial & UNTLK & Send ``untalk'' command \\
\hline
\$FFAE & Serial & UNLSN & Send ``unlisten'' command \\
\hline
\$FFB1 & Serial & LISTN & Send ``listen'' command \\
\hline
\$FFB4 & Serial & TALK & Send ``talk'' command \\
\hline
\$FFB7 & I/O & READSS & Get status of last I/O operation \\
\hline
\$FFBA & Files & SETLFS & Set file, device, secondary address \\
\hline
\$FFBD & Files & SETNAM & Set filename pointers \\
\hline
\$FFC0 & Files & OPEN & Open logical file \\
\hline
\$FFC3 & Files & CLOSE & Close logical file \\
\hline
\$FFC6 & I/O & CHKIN & Set input channel \\
\hline
\$FFC9 & I/O & CKOUT & Set output channel \\
\hline
\$FFCC & I/O & CLRCH & Restore default channels \\
\hline
\$FFCF & I/O & BASIN & Read a character from input device \\
\hline
\$FFD2 & I/O & BSOUT & Write a character to output device \\
\hline
\$FFD5 & Files & LOAD & Load/verify from file \\
\hline
\$FFD8 & Files & SAVE & Save to file \\
\hline
\$FFDB & Tools & SETTIM & Set CIA1 24-hour clock \\
\hline
\$FFDE & Tools & RDTIM & Read CIA1 24-hour clock \\
\hline
\$FFE1 & Tools & STOP & Report Stop key (see ScanStopKey) \\
\hline
\$FFE4 & I/O & GETIN & Read a character from input device, without waiting \\
\hline
\$FFE7 & Files & CLALL & Close all files and channels \\
\hline
\$FFEA & Tools & ScanStopKey & Scan Stop key \\
\hline
\$FFED & Editor & SCRORG & Get current screen window size \\
\hline
\$FFF0 & Editor & PLOT & Read/set cursor position \\
\hline
\$FFF3 & Reserved & IOBASE & RESERVED, DO NOT USE \\
\hline
\end{longtable}


\newpage
\section{MEGA65 KERNAL Jump Table Reference}
\titleformat*{\subsection}{\normalfont\huge\bfseries\color{blue}}

%%%%%%%%%%%%%%%%%%%%%%%%%%%%%%%%%%%%%%%%%%%%%%%%%%%%%%%%%%%%%%%%%

% ****
% ACPTR
% ****

\newpage
\subsection{ACPTR}
\index{KERNAL Jump Table!ACPTR}
\label{KERNAL Jump Table!ACPTR}
\begin{description}[leftmargin=2cm,style=nextline]
    \item [Address:] JSR \$FFA5
    \item [Description:] Accept a byte from talker
    \item [Outputs:]
        \textbf{A} The accepted byte
    \item [Remarks:]
        This is a low-level serial routine. Most programs will prefer the higher level I/O routines (OPEN et al.).

        After sending the TALK command to a serial device to tell it to provide output, and optionally a secondary address with TKSA, a program calls ACPTR to read data from the device, one byte at a time. See TALK.

        This behavior can be overridden or extended with the IACPTR vector.
    \item [Example:]
        \begin{asmcode}
acptr = $ffa5

    jsr $ffa5
    sta data
        \end{asmcode}
\end{description}


% ****
% ADDKEY
% ****

\newpage
\subsection{ADDKEY}
\index{KERNAL Jump Table!ADDKEY}
\label{KERNAL Jump Table!ADDKEY}
\begin{description}[leftmargin=2cm,style=nextline]
    \item [Address:] JSR \$FF4A
    \item [Description:] Add a character to the keyboard input buffer
    \item [Inputs:]
        \textbf{A} The PETSCII code
    \item [Outputs:]
        \textbf{C flag} 0 on success, 1 on failure
    \item [Remarks:]
        The MEGA65 KERNAL reads keyboard events from three sources: the active function key macro, a keyboard buffer managed by the KERNAL, and the hardware typing event queue. When GETIN reads from the keyboard, it checks each of these sources, in order.

        ADDKEY adds a PETSCII code to the keyboard buffer. This buffer is processed by GETIN when no function key macro is active. It supersedes the hardware typing event queue.

        The keyboard buffer has a fixed size. If the buffer is full, the key will not be added, and the C flag will be set.

        \underline{NOTE}: For the MEGA65, the buffer is consumed by calls to GETIN, and not by the KERNAL IRQ as on a C64. A program must return to the screen editor for GETIN to be called. If a program needs to enqueue more than a few characters, one option is to use an IIRQ vector to feed characters into the buffer gradually.

    \item [Example:]
        \begin{asmcode}
addkey = $ff4a

    ; Cause RUN and a carriage return to be typed
    ; once control returns to the screen editor.
    lda #'R'
    jsr addkey
    bcs err
    lda #'U'
    jsr addkey
    bcs err
    lda #'N'
    jsr addkey
    bcs err
    lda #13
    jsr addkey
    bcs err
        \end{asmcode}

\end{description}


% ****
% BASIN
% ****

\newpage
\subsection{BASIN}
\index{KERNAL Jump Table!BASIN}
\label{KERNAL Jump Table!BASIN}
\begin{description}[leftmargin=2cm,style=nextline]
    \item [Address:] JSR \$FFCF
    \item [Description:] Read a character from input device
    \item [Outputs:]
        On success: \\
        \textbf{C flag} = 0 \\
        \textbf{A} The character read

        On error: \\
        \textbf{C flag} = 1 \\
        \textbf{A} Error code, if any
    \item [Remarks:]
        The default input device is the keyboard. A program can change the input device with OPEN and CHKIN.

        BASIN treats the input device as buffered and line oriented, with each line ending in a carriage return (PETSCII 13).

        If the device is RS-232 or serial and the buffer is empty, BASIN waits for the next character. If the device is in an error state, BASIN skips input and returns a carriage return.

        If the device is the keyboard, the cursor is activated and BASIN waits for the user to type a logical line and press Return. Each call to BASIN returns a character from the line, up to and including the carriage return. This also advances the cursor position to the last character on the logical line---but not after it. (Writing a character at this point, such as with BSOUT, overwrites the last character on the line. It is usually appropriate to write a single carriage return to the screen terminal.)

        If the device is the screen, BASIN behaves as if the user pressed Return with keyboard input, and reads from the current cursor position up to the end of the logical line. The final character of a logical line is a carriage return.

        To attempt to read a character without waiting on an empty buffer, use GETIN.

        This routine can be overridden or extended using the IBASIN vector.

        BASIN is comparable to ``CHRIN'' on the Commodore 64.
    \item [Example:]
       \begin{asmcode}
basin = $ffcf

    ; Read a line into memory, up to
    ; 255 characters
    ldy #0
read_more:
    jsr basin
    sta data,y
    iny
    beq out_of_memory
    cmp #$0d   ; carriage return?
    bne read_more
        \end{asmcode}

\end{description}


% ****
% BOOT\_SYS
% ****

\newpage
\subsection{BOOT{\_}SYS}
\index{KERNAL Jump Table!BOOT\_SYS}
\label{KERNAL Jump Table!BOOT_SYS}
\begin{description}[leftmargin=2cm,style=nextline]
    \item [Address:] JMP \$FF59
    \item [Description:] Boot an alternate system from disk
    \item [Remarks:]
        The Commodore 65 intended a feature where a floppy disk in the internal 3-1/2" disk drive could bootstrap software using up to 512 bytes of machine code in a ``home'' sector, at track 0, sector 1.

        When the BOOT\_SYS KERNAL routine is called, or the user performs the \textbf{BOOT SYS} command, or the user holds the Alt key during start-up, the KERNAL reads the home sector of the disk. If the first byte is \$4C, the KERNAL loads the sector into memory at 0.0400, then invokes it with JMP. If no disk is present or the first byte is not \$4C, no action is taken.

        The boot sector feature is distinct from the BASIC 10 \textbf{BOOT} command that loads and runs a file named \texttt{AUTOBOOT.C65}. It is not compatible with C128-style boot sectors.

        This routine does not return. Only use JMP with this routine.

        \underline{NOTE}: For the MEGA65, this feature is untested.
    \item [Example:]
        \begin{asmcode}
boot_sys = $ff59

    jmp boot_sys
        \end{asmcode}

\end{description}


% ****
% BSOUT
% ****

\newpage
\subsection{BSOUT}
\index{KERNAL Jump Table!BSOUT}
\label{KERNAL Jump Table!BSOUT}
\begin{description}[leftmargin=2cm,style=nextline]
    \item [Address:] JSR \$FFD2
    \item [Description:] Write a character to output device
    \item [Inputs:]
        \textbf{A} Character to output
    \item [Outputs:]
        \textbf{C flag} Set on error \\
        \textbf{A} Error code, if any
    \item [Remarks:]
        The default output device is the screen. A program can change the output device with OPEN and CKOUT.

        BSOUT treats the output device as buffered.

        If the device is RS-232 and the output buffer is full, BSOUT waits until there is room in the buffer.

        If the device is serial, a single character buffer is used with CIOUT and the previously buffered character is sent.

        If the device is the screen, the character is written to screen memory at the current cursor position.

        For the list of possible error codes, see OPEN.

        This routine can be overridden or extended using the IBSOUT vector.

        BSOUT is comparable to ``CHROUT'' on the Commodore 64.
    \item [Example:]
        \begin{asmcode}
bsout = $ffd2

    lda #$0d
    jsr bsout
        \end{asmcode}

\end{description}


% ****
% C64MODE
% ****

\newpage
\subsection{C64MODE}
\index{KERNAL Jump Table!C64MODE}
\label{KERNAL Jump Table!C64MODE}
\begin{description}[leftmargin=2cm,style=nextline]
    \item [Address:] JMP \$FF53
    \item [Description:] Reset to GO64 mode
    \item [Remarks:]
        This switches to the GO64 mode memory map, resets the VIC modes, and jumps to the GO64 start routine.

        This routine does not return. Only use JMP with this routine.
    \item [Example:]
        \begin{asmcode}
c64mode = $ff53

    jmp c64mode
        \end{asmcode}
\end{description}


% ****
% CHKIN
% ****

\newpage
\subsection{CHKIN}
\index{KERNAL Jump Table!CHKIN}
\label{KERNAL Jump Table!CHKIN}
\begin{description}[leftmargin=2cm,style=nextline]
    \item [Address:] JSR \$FFC6
    \item [Description:] Set input channel
    \item [Inputs:]
        \textbf{X} The logical address of the input device
    \item [Outputs:]
        \textbf{C flag} Set on error \\
        \textbf{A} Error code, if any
    \item [Remarks:]
        This changes the input channel to use a given device identified by its logical address. Call CHKIN after a call to OPEN. BASIN and GETIN read from the current input channel.

        The default input device is the keyboard. A program can call CLRCH to restore default channel settings.

        CHKIN performs device-specific tasks, such as sending a TALK command to a serial channel. Setting a non-input device results in an error.

        For the list of possible error codes, see OPEN.

        This routine can be overridden or extended using the ICHKIN vector.
    \item [Example:]
        \begin{asmcode}
chkin = $ffc6

    ldx #8
    jsr chkin
        \end{asmcode}
\end{description}


% ****
% CINT
% ****

\newpage
\subsection{CINT}
\index{KERNAL Jump Table!CINT}
\label{KERNAL Jump Table!CINT}
\begin{description}[leftmargin=2cm,style=nextline]
    \item [Address:] JSR \$FF81
    \item [Description:] Initialise screen editor
    \item [Remarks:]
        CINT resets all properties of the screen editor, including indirect vectors, function key macros, VIC registers, and SID registers. It also clears the screen.

        CINT is typically called along with IOINIT, which resets additional properties of the screen editor.

        Because it updates indirect vectors, it must be called with IRQs disabled.
    \item [Example:]
        \begin{asmcode}
cint = $ff81

    sei
    jsr cint
    cli
        \end{asmcode}
\end{description}


% ****
% CIOUT
% ****

\newpage
\subsection{CIOUT}
\index{KERNAL Jump Table!CIOUT}
\label{KERNAL Jump Table!CIOUT}
\begin{description}[leftmargin=2cm,style=nextline]
    \item [Address:] JSR \$FFA8
    \item [Description:] Send a byte to listener
    \item [Inputs:]
        \textbf{A} The byte to send
    \item [Remarks:]
        This is a low-level serial routine. Most programs will prefer the higher level I/O routines (OPEN et al.).

        After sending the LISTN command to a serial device to tell it to listen for input, and optionally a secondary address with SECND, a program uses CIOUT to send data to the device, one byte at a time.

        This behavior can be overridden or extended with the ICIOUT vector.
    \item [Example:]
        \begin{asmcode}
ciout = $ffa8

    lda data
    jsr ciout
        \end{asmcode}
\end{description}


% ****
% CKOUT
% ****

\newpage
\subsection{CKOUT}
\index{KERNAL Jump Table!CKOUT}
\label{KERNAL Jump Table!CKOUT}
\begin{description}[leftmargin=2cm,style=nextline]
    \item [Address:] JSR \$FFC9
    \item [Description:] Set output channel
    \item [Inputs:]
        \textbf{X} The logical address of the output device
    \item [Outputs:]
        \textbf{C flag} Set on error \\
        \textbf{A} Error code, if any
    \item [Remarks:]
        This changes the output channel to use a given device identified by its logical address. Call CKOUT after a call to OPEN. BSOUT writes to the current output channel.

        The default output device is the screen terminal. A program can call CLRCH to restore default channel settings.

        CKOUT performs device-specific tasks, such as sending a LISTN command to a serial channel. Setting a non-output device results in an error.

        For the list of possible error codes, see OPEN.

        This routine can be overridden or extended using the ICKOUT vector.

        CKOUT is comparable to ``CHKOUT'' on the Commodore 64.
    \item [Example:]
        \begin{asmcode}
ckout = $ffc9

    ldx #8
    jsr chkout
        \end{asmcode}
\end{description}


% ****
% CLALL
% ****

\newpage
\subsection{CLALL}
\index{KERNAL Jump Table!CLALL}
\label{KERNAL Jump Table!CLALL}
\begin{description}[leftmargin=2cm,style=nextline]
    \item [Address:] JSR \$FFE7
    \item [Description:] Close all files and channels
    \item [Remarks:]
        All open logical files are closed, as if closed with CLOSE. This includes performing device-specific closing tasks, as well as restoring the default input and output channels (as if via CLRCH).

        This behavior can be overridden or extended with the ICLALL vector.
    \item [Example:]
        \begin{asmcode}
clall = $ffe7

    jsr clall
        \end{asmcode}
\end{description}


% ****
% CLOSE
% ****

\newpage
\subsection{CLOSE}
\index{KERNAL Jump Table!CLOSE}
\label{KERNAL Jump Table!CLOSE}
\begin{description}[leftmargin=2cm,style=nextline]
    \item [Address:] JSR \$FFC3
    \item [Description:] Close logical file
    \item [Inputs:]
        \textbf{A} The logical file to close \\
        \textbf{C flag} Set to delete logical file without sending close command
    \item [Outputs:]
        \textbf{C flag} Set on error \\
        \textbf{A} Error code, if error
    \item [Remarks:]
        This performs device-specific closing tasks for the given open logical file number.

        If the file is already closed, or the logical file number refers to the keyboard or screen, CLOSE returns without an error.

        If the device is RS-232, this waits for the write buffer to empty before closing.

        If the device is serial, any buffered output character is sent, a ``close'' command is sent if appropriate, and the device is told to UNLSTN.

        As a special case, if the C flag is set, the device is a disk (FA is 8 or greater), and the secondary address is the command channel (15), no ``close'' command is sent, and the logical file is removed from the file list. This is necessary to prevent the disk from closing other opened files when closing the command channel. See OPEN for a complete example.

        \underline{NOTE}: When CLOSE is used on a file that was set as the input or output channel by CHKIN or CKOUT, it does \emph{not} reset the input/output channel, and a subsequent read/write may misbehave. Use CLRCH after CLOSE to reset the input/output channels.

        For the list of possible error codes, see OPEN.

        This behavior can be overridden or extended with the ICLOSE vector.
    \item [Example:]
        \begin{asmcode}
close = $ffc3
clrch = $ffcc

    clc          ; normal close
    lda #$01
    jsr close
    jsr clrch    ; clear I/O channels after
                 ; closing the file
    \end{asmcode}
\end{description}


% ****
% CLOSE\_ALL
% ****

\newpage
\subsection{CLOSE{\_}ALL}
\index{KERNAL Jump Table!CLOSE\_ALL}
\label{KERNAL Jump Table!CLOSE_ALL}
\begin{description}[leftmargin=2cm,style=nextline]
    \item [Address:] JSR \$FF50
    \item [Description:] Close all files on a device
    \item [Inputs:]
        \textbf{A} The device number (0-31)
    \item [Remarks:]
        This calls CLOSE for each open file for the given device (FA).

        If one of the closed files is the current input or output channel, the default is restored.
    \item [Example:]
        \begin{asmcode}
close_all = $ff50

    lda #$08
    jsr close_all
        \end{asmcode}
\end{description}


% ****
% CLRCH
% ****

\newpage
\subsection{CLRCH}
\index{KERNAL Jump Table!CLRCH}
\label{KERNAL Jump Table!CLRCH}
\begin{description}[leftmargin=2cm,style=nextline]
    \item [Address:] JSR \$FFCC
    \item [Description:] Restore default channels
    \item [Remarks:]
        This clears all open channels and restores the default system I/O channels, including the keyboard as the input channel and the screen terminal as the output channel.

        If the current input or output channels are to serial devices, CLRCH sends UNTLK or UNLSN, as appropriate.

        This routine can be overridden or extended using the ICLRCH vector.

        CLRCH is comparable to ``CLRCHN'' on the Commodore 64.
    \item [Example:]
        \begin{asmcode}
clrch = $ffcc

    jsr clrch
        \end{asmcode}
\end{description}


% ****
% CMP\_FAR
% ****

\newpage
\subsection{CMP{\_}FAR}
\index{KERNAL Jump Table!CMP\_FAR}
\label{KERNAL Jump Table!CMP_FAR}
\begin{description}[leftmargin=2cm,style=nextline]
    \item [Address:] JSR \$FF7A
    \item [Description:] Compare a byte with an address in any bank
    \item [Inputs:]
        \textbf{A} Data to compare \\
        \textbf{X} Base page pointer \\
        \textbf{Y} Index from pointed address \\
        \textbf{Z} Bank
    \item [Outputs:]
        CPU status flags based on the comparison
    \item [Remarks:]
        Conceptually, this is the equivalent of \texttt{cmp (X),y} in bank Z.

        This can access any address in the first 1MB of memory. Unlike JMPFAR, this does not change the memory map. It uses a DMA job to access the long address.

        For the MEGA65, the 45GS02 supports an instruction that can do this with a 32-bit address stored on the base page, without the need for a KERNAL routine: \texttt{cmp [zp4],z}
    \item [Example:]
        \begin{asmcode}
cmp_far = $ff7a

    ; Compare memory at address 4.4502
    ; to the accumulator
    lda #$00
    sta $fe
    lda #$45
    sta $ff
    ldx #$fe  ; base page pointer -> $4500
    ldy #$02  ; Y index
    ldz #$04  ; bank
    lda #$99  ; data to compare
    jsr cmp_far
    beq they_are_equal  ; ...
        \end{asmcode}

\end{description}

FF35

% ****
% CURSOR
% ****

\newpage
\subsection{CURSOR}
\index{KERNAL Jump Table!CURSOR}
\label{KERNAL Jump Table!CURSOR}
\begin{description}[leftmargin=2cm,style=nextline]
    \item [Address:] JSR \$FF35
    \item [Description:] Enable or disable the blinking cursor
    \item [Inputs:]
        \textbf{C flag} 0 to enable, 1 to disable
    \item [Remarks:]
        In normal operation, the editor enables the blinking cursor at the \texttt{READY.} prompt or when executing the {\bf INPUT} command at the terminal, and otherwise keeps it disabled when running other commands, including machine code invoked by {\bf SYS}. A program can re-enable the cursor blink when providing similar user experiences, such as waiting for user keyboard input with {\bf GETIN}.

        Enabling the cursor only causes the screen editor's cursor position to be indicated by animating the character at that position. It is the program's responsibility to listen for keyboard input and emit characters to the screen.

        In BASIC, the {\bf CURSOR OFF} and {\bf CURSOR ON} commands have the same effect as this routine.
    \item [Example:]
        \begin{asmcode}
bsout = $ffd2
cursor = $ff35
getin = $ffe4

    ; Enable the cursor.
    clc
    jsr cursor

    ; Wait for a keypress.
-   jsr getin
    beq -
    pha  ; Stash the typed character.

    ; Disable the cursor.
    clc
    jsr cursor

    ; Emit the typed character.
    pla
    jsr bsout
    \end{asmcode}

\end{description}


% ****
% GETIN
% ****

\newpage
\subsection{GETIN}
\index{KERNAL Jump Table!GETIN}
\label{KERNAL Jump Table!GETIN}
\begin{description}[leftmargin=2cm,style=nextline]
    \item [Address:] JSR \$FFE4
    \item [Description:] Read a character from input device, without waiting
    \item [Outputs:]
        \textbf{A} The character read \\
        \textbf{C flag} Set on error
    \item [Remarks:]
        The default input device is the keyboard. A program can change the input device with OPEN and CHKIN.

        GETIN treats the input device as buffered, but does not wait for input. If the input buffer for the device is empty, GETIN returns \$00.

        For the MEGA65, if the input device is the keyboard, GETIN uses several keyboard input buffers: the KERNAL keyboard buffer, the function key macro buffer, and the MEGA65 hardware keyboard buffer. Unlike the C65, the MEGA65 KERNAL does not populate the KERNAL keyboard buffer during the IRQ handler. Instead, GETIN consumes one typing event from the MEGA65 hardware keyboard buffer directly. The KERNAL keyboard buffer is still honored to support software features that inject keystrokes.

        If the input device is serial or the screen, it behaves like BASIN.

        This routine can be overridden or extended using the IGETIN vector.
    \item [Example:]
        \begin{asmcode}
getin = $ffe4

get_a_key:
    jsr getin
    beq get_a_key
    sta data
        \end{asmcode}

\end{description}


% ****
% GETIO
% ****

\newpage
\subsection{GETIO}
\index{KERNAL Jump Table!GETIO}
\label{KERNAL Jump Table!GETIO}
\begin{description}[leftmargin=2cm,style=nextline]
    \item [Address:] JSR \$FF41
    \item [Description:] Read the current input and output devices
    \item [Outputs:]
        \textbf{X} current input device (FA) \\
        \textbf{Y} current output device (FA)
    \item [Remarks:]
        An input device of 0 represents the keyboard.

        An output device of 3 represents the screen.

        These settings are modified by CHKIN and CKOUT.
    \item [Example:]
        \begin{asmcode}
getio = $ff41

    jsr getio
    cpy #3
    beq is_screen_output
        \end{asmcode}

\end{description}


% ****
% GETLFS
% ****

\newpage
\subsection{GETLFS}
\index{KERNAL Jump Table!GETLFS}
\label{KERNAL Jump Table!GETLFS}
\begin{description}[leftmargin=2cm,style=nextline]
    \item [Address:] JSR \$FF44
    \item [Description:] Read file, device, secondary address
    \item [Outputs:]
        \textbf{A} The logical file number (LA) \\
        \textbf{X} The device number (FA) \\
        \textbf{Y} The secondary address, or \$FF (SA)
    \item [Remarks:]
        These settings are modified by SETLFS, as well as other KERNAL calls and BASIC commands that look up information about open files via the logical file number.

        A program can use this to determine from which disk drive the program was loaded by calling it just after starting, before performing other disk I/O.
    \item [Example:]
        \begin{asmcode}
getlfs = $ff44

    jsr getlfs
        \end{asmcode}

\end{description}


% ****
% IOBASE
% ****

\newpage
\subsection{IOBASE}
\index{KERNAL Jump Table!IOBASE}
\label{KERNAL Jump Table!IOBASE}
\begin{description}[leftmargin=2cm,style=nextline]
    \item [Address:] JSR \$FFF3
    \item [Description:] RESERVED, DO NOT USE
    \item [Remarks:]
        Historically, this routine returns the start address of the I/O registers. On the C65, this is always \$D000.
\end{description}


% ****
% IOINIT
% ****

\newpage
\subsection{IOINIT}
\index{KERNAL Jump Table!IOINIT}
\label{KERNAL Jump Table!IOINIT}
\begin{description}[leftmargin=2cm,style=nextline]
    \item [Address:] JSR \$FF84
    \item [Description:] Initialise I/O devices
    \item [Remarks:]
        IOINIT resets the CIA chips, the 45GS02 port, the VIC chip, the UART, and the DOS.

        It must be called with IRQs disabled.
    \item [Example:]
    \begin{asmcode}
ioinit = $ff84

    sei
    jsr ioinit
    cli
    \end{asmcode}
\end{description}


% ****
% JMPFAR
% ****

\newpage
\subsection{JMPFAR}
\index{KERNAL Jump Table!JMPFAR}
\label{KERNAL Jump Table!JMPFAR}
\begin{description}[leftmargin=2cm,style=nextline]
    \item [Address:] JMP \$FF71
    \item [Description:] Jump to address in any bank
    \item [Inputs:]
        \textbf{\$02} Address bank (0-5) \\
        \textbf{\$03} Address high \\
        \textbf{\$04} Address low \\
        \textbf{\$05} CPU status register \\
        \textbf{\$06} CPU A register \\
        \textbf{\$07} CPU X register \\
        \textbf{\$08} CPU Y register \\
        \textbf{\$09} CPU Z register
    \item [Remarks:]
        This updates the CPU memory map to make a requested address visible as a 16-bit address, then JMPs to it. It does not return.

        This routine has similar restrictions as the \textbf{SYS} BASIC command, and cannot reach every address. It can only jump to addresses in non-zero banks from X.2000 -- X.7FFF.

        The revised memory map keeps \$0000--\$1FFF and \$8000--\$DFFF un-mapped (so they refer to bank 0 and I/O registers), and \$E000--\$FFFF mapped to KERNAL ROM.

        A program sets parameters to JMPFAR by writing to the KERNAL's base page. Parameters include the three address bytes, and the state of the CPU registers expected on entry to the new location.

        This routine does not return. Only use JMP to call this routine.
    \item [Example:]
        \begin{asmcode}
jmpfar = $ff71

    ; MAP $2000-$7FFF to bank 4, then JMP
    ; to 4.4500
    lda #$04
    sta $02
    lda #$45
    sta $03
    lda #$00
    sta $04
    lda #$04  ; Start routine with
              ; interrupts disabled
    sta $05
    jmp jmpfar
        \end{asmcode}
\end{description}


% ****
% JSRFAR
% ****

\newpage
\subsection{JSRFAR}
\index{KERNAL Jump Table!JSRFAR}
\label{KERNAL Jump Table!JSRFAR}
\begin{description}[leftmargin=2cm,style=nextline]
    \item [Address:] JSR \$FF6E
    \item [Description:] Call subroutine in any bank
    \item [Inputs:]
        \textbf{\$02} Address bank (0-5) \\
        \textbf{\$03} Address high \\
        \textbf{\$04} Address low \\
        \textbf{\$05} CPU status register \\
        \textbf{\$06} CPU A register \\
        \textbf{\$07} CPU X register \\
        \textbf{\$08} CPU Y register \\
        \textbf{\$09} CPU Z register
    \item [Outputs:]
        CPU status, A, X, Y, and Z registers are as they were at the end of the subroutine.
    \item [Remarks:]
        This updates the CPU memory map to make a requested address visible as a 16-bit address, then JSRs to it. On return, it restores the previous KERNAL-managed memory map (e.g. the SYS map).

        This routine has similar restrictions as the \textbf{SYS} BASIC command, and cannot reach every address. It can only jump to addresses in non-zero banks from X.2000 -- X.7FFF.

        The revised memory map keeps \$0000--\$1FFF and \$8000--\$DFFF un-mapped (so they refer to bank 0 and I/O registers), and \$E000--\$FFFF mapped to KERNAL ROM.

        A program sets parameters to JSRFAR by writing to the KERNAL's base page. Parameters include the three address bytes, and the state of the CPU registers expected on entry to the new location.

        Only use JSR to call this routine.
    \item [Example:]
        \begin{asmcode}
jsrfar = $ff6e

    ; MAP $2000-$7FFF to bank 4, then
    ; JSR to 4.4500
    lda #$04
    sta $02
    lda #$45
    sta $03
    lda #$00
    sta $04
    lda #$04  ; Start routine with
              ; interrupts disabled
    sta $05
    jsr jsrfar

    ; Program continues...
        \end{asmcode}
\end{description}


% ****
% KEY
% ****

\newpage
\subsection{KEY}
\index{KERNAL Jump Table!KEY}
\label{KERNAL Jump Table!KEY}
\begin{description}[leftmargin=2cm,style=nextline]
    \item [Address:] JSR \$FF9F
    \item [Description:] Perform a keyboard scan.
    \item [Remarks:]
        For the MEGA65, typing events are managed by a hardware buffer mechanism and not the KERNAL. A program that wants to consume a typing event can call GETIN, or access the typing event buffer registers directly, without calling KEY.

        KEY manages special keyboard behaviors, including Mega + Shift, No Scroll, and Ctrl-S. The MEGA65 KERNAL calls KEY in the KERNAL IRQ handler many times per second, similar to other Commodores. A program that disables the KERNAL IRQ handler but wants to maintain these behaviors can call KEY from a custom handler.

        \underline{NOTE}: For the MEGA65, KEY diverges from the C65 and does not offer vector hooks. Keyboard intercept vectors are a work in progress.

        KEY has the same address as the Commodore 64 routine ``SCNKEY,'' but behaves differently.
    \item [Example:]
        \begin{asmcode}
key = $ff9f

    jsr key
        \end{asmcode}
\end{description}


% ****
% LDA\_FAR
% ****

\newpage
\subsection{LDA{\_}FAR}
\index{KERNAL Jump Table!LDA\_FAR}
\label{KERNAL Jump Table!LDA_FAR}
\begin{description}[leftmargin=2cm,style=nextline]
    \item [Address:] JSR \$FF74
    \item [Description:] Read a byte from an address in any bank
    \item [Inputs:]
        \textbf{X} Base page pointer \\
        \textbf{Y} Index from pointed address \\
        \textbf{Z} Bank
    \item [Outputs:]
        \textbf{A} The value at the address
    \item [Remarks:]
        Conceptually, this is the equivalent of \texttt{lda (X),y} in bank Z.

        This can access any address in the first 1MB of memory. Unlike JMPFAR, this does not change the memory map. It uses a DMA job to access the long address.

        For the MEGA65, the 45GS02 supports an instruction that can do this with a 32-bit address stored on the base page, without the need for a KERNAL routine: \texttt{lda [zp4],z}
    \item [Example:]
        \begin{asmcode}
lda_far = $ff74

    ; Load memory at address 4.4502
    ; into the accumulator
    lda #$00
    sta $fe
    lda #$45
    sta $ff
    ldx #$fe  ; base page pointer -> $4500
    ldy #$02  ; Y index
    ldz #$04  ; bank
    jsr lda_far
        \end{asmcode}

\end{description}


% ****
% LISTN
% ****

\newpage
\subsection{LISTN}
\index{KERNAL Jump Table!LISTN}
\label{KERNAL Jump Table!LISTN}
\begin{description}[leftmargin=2cm,style=nextline]
    \item [Address:] JSR \$FFB1
    \item [Description:] Send ``listen'' command
    \item [Inputs:]
        \textbf{A} The device number (FA, 0-31)
    \item [Remarks:]
        This is a low-level serial routine. Most programs will prefer the higher level I/O routines (OPEN et al.).

        With the low-level serial protocol, the computer is expected to coordinate all device communication on the serial bus. Each device has a hard-coded device number; some devices, like disk drives, may have switches for changing its recognised device number. All devices see all traffic on the bus, and are expected to change their mode when they see a command with their device number.

        When the computer wants to send data to a device, the computer sends the device the ``listen'' command. Typically, the computer is only sending data to one device at a time. To change listeners, the computer broadcasts an ``untalk'' command to all devices, then sends another ``listen'' command with a different device number.

        The computer gives one device at a time permission to ``talk'' on the serial bus. To change talkers, the computer broadcasts an ``untalk'' command to all devices, then sends another ``talk'' command.

        Use UNLSN to tell all devices to stop talking.

        This behavior can be overridden or extended with the ILISTEN vector.

        LISTN is comparable to ``LISTEN'' on the Commodore 64.

    \item [Example:]
        \begin{asmcode}
listn = $ffb1

    lda #8
    jsr listn
        \end{asmcode}
\end{description}


% ****
% LKUPLA
% ****

\newpage
\subsection{LKUPLA}
\index{KERNAL Jump Table!LKUPLA}
\label{KERNAL Jump Table!LKUPLA}
\begin{description}[leftmargin=2cm,style=nextline]
    \item [Address:] JSR \$FF5F
    \item [Description:] Search for logical file number in use
    \item [Inputs:]
        \textbf{A} The logical file number (LA)
    \item [Outputs:]
        If found: \\
        \textbf{C flag} = 0 \\
        \textbf{A} The logical file number (LA) \\
        \textbf{X} The device number (FA) \\
        \textbf{Y} The secondary address (SA), or \$FF if not applicable

        If not found: \\
        \textbf{C flag} = 1
    \item [Remarks:]
        This is used primarily by BASIC to locate an unused logical file number. In practice, most programs have complete control over the logical file numbers they use, and no look-up is necessary.

        This can also be used to identify the device for an open file by the logical file number.
    \item [Example:]
        \begin{asmcode}
lkupla = $ff5f

    lda #$01
    jsr lkupla
    bcs not_in_use
        \end{asmcode}
\end{description}


% ****
% LKUPSA
% ****

\newpage
\subsection{LKUPSA}
\index{KERNAL Jump Table!LKUPSA}
\label{KERNAL Jump Table!LKUPSA}
\begin{description}[leftmargin=2cm,style=nextline]
    \item [Address:] JSR \$FF62
    \item [Description:] Search for secondary address in use
    \item [Inputs:]
        \textbf{Y} The secondary address (SA)
    \item [Outputs:]
        If found: \\
        \textbf{C flag} = 0 \\
        \textbf{A} The logical file number (LA) \\
        \textbf{X} The device number (FA) \\
        \textbf{Y} The secondary address (SA), or \$FF if not applicable

        If not found: \\
        \textbf{C flag} = 1
    \item [Remarks:]
        This is used primarily by BASIC to locate an unused secondary address. In practice, most programs have complete control over the secondary addresses they use, and no look-up is necessary.
    \item [Example:]
        \begin{asmcode}
lkupsa = $ff62

    ldy #$60
    jsr lkupsa
    bcs not_in_use
        \end{asmcode}
\end{description}


% ****
% LOAD
% ****

\newpage
\subsection{LOAD}
\index{KERNAL Jump Table!LOAD}
\label{KERNAL Jump Table!LOAD}
\begin{description}[leftmargin=2cm,style=nextline]
    \item [Address:] JSR \$FFD5
    \item [Description:] Load/verify from file
    \item [Inputs:]
        \textbf{A} = 0 to load, 1 to verify; set bit 6 for ``raw'' mode \\
        \textbf{X} Load address, low \\
        \textbf{Y} Load address, high
    \item [Outputs:]
        \textbf{C flag} Set on error \\
        \textbf{A} Error code, if error \\
        \textbf{X} Ending address, low \\
        \textbf{Y} Ending address, high
    \item [Remarks:]
        Before calling LOAD, the program must call SETBNK, SETLFS, and SETNAM to set parameters for the file to load. LOAD cannot be used with RS-232 devices, the screen, or the keyboard. The logical file number is not used.

        When used with a disk drive, LOAD only loads a file of type PRG. To load a file of another type, use SETNAM with the type mentioned in the name string (such as \texttt{"MYFILE,S"} for a SEQ file), along with OPEN, CHKIN, BASIN, and CLOSE. See OPEN for an example.

        If A bit 0 is clear, LOAD loads the file from the input device and writes it to memory.

        If A bit 0 is set, LOAD verifies the file: it loads the file and compares it to the contents of memory. If the file does not match memory, LOAD returns an error status. The memory is not changed.

        If A bit 6 is set (``raw'' mode), LOAD treats the first two bytes of the file as data. When clear, the first two bytes of the file are expected to be the stored load address, as is typical for a PRG file. A PRG loaded with raw mode must still contain at least two bytes, or a File Not Found error is reported.

        If the secondary address (SA) is \$00, the provided X/Y is used as the load address, and the first two bytes of the file are skipped if not in ``raw'' mode. If SA is not \$00, X/Y are ignored, and the actual load address is taken from the stored load address in the first two bytes of the file. This is the same behavior as this BASIC command: \texttt{LOAD "FILE",8,1}

        The complete LOAD operation must fit within a single bank. It cannot cross bank boundaries.

        LOAD performs all of the tasks needed to open the file, read it, then close it. It does not leave any logical files open (as OPEN does).

        On success, the address of the byte last byte loaded (or verified) \emph{plus one} is returned in the X and Y registers.

        LOAD sets the I/O status as appropriate. This can be read with READSS.

        For the list of possible error codes, see OPEN.

        This behavior can be overridden or extended with the ILOAD vector.
    \item [Example:]
        \begin{asmcode}
load = $ffd5
setbnk = $ff6b
setlfs = $ffba
setnam = $ffbd

    lda #$00    ; memory in bank 0
    ldx #$00    ; filename in bank 0
    jsr setbnk

    lda #$00    ; (not used by load)
    ldx #$08    ; device number
    ldy #$00    ; load
    jsr setlfs

    lda #filename_end-filename
    ldx #<filename
    ldy #>filename
    jsr setnam

    lda #0      ; load
    ldx #$00    ; to 0.4500
    ldy #$45
    jsr load

    rts

filename:
    !pet "myfile"
filename_end:
        \end{asmcode}
\end{description}


% ****
% MEMBOT
% ****

\newpage
\subsection{MEMBOT}
\index{KERNAL Jump Table!MEMBOT}
\label{KERNAL Jump Table!MEMBOT}
\begin{description}[leftmargin=2cm,style=nextline]
    \item [Address:] JSR \$FF9C
    \item [Description:] RESERVED, DO NOT USE
    \item [Remarks:]
        Historically, MEMBOT and MEMTOP refer to memory allocation parameters used by the KERNAL. The C65 KERNAL does not use this facility. These routines are mentioned in C65 documentation but do not do anything.
\end{description}


% ****
% MEMTOP
% ****

\newpage
\subsection{MEMTOP}
\index{KERNAL Jump Table!MEMTOP}
\label{KERNAL Jump Table!MEMTOP}
\begin{description}[leftmargin=2cm,style=nextline]
    \item [Address:] JSR \$FF99
    \item [Description:] RESERVED, DO NOT USE
    \item [Remarks:]
        Historically, MEMBOT and MEMTOP refer to memory allocation parameters used by the KERNAL. The C65 KERNAL does not use this facility. These routines are mentioned in C65 documentation but do not do anything.
\end{description}


% ****
% MONEXIT
% ****

\newpage
\subsection{MONEXIT}
\index{KERNAL Jump Table!MONEXIT}
\label{KERNAL Jump Table!MONEXIT}
\begin{description}[leftmargin=2cm,style=nextline]
    \item [Address:] JMP \$FFA2
    \item [Description:] Monitor's exit to BASIC
    \item [Remarks:]
        This is only used by the monitor, and is not useful to programs.

        MONEXIT has the same address as the Commodore 64 KERNAL routine ``SETTMO.'' SETTMO is not available on the MEGA65.
    \item [Example:]
        \begin{asmcode}
monexit = $ffa2

    jmp monexit
        \end{asmcode}

\end{description}


% ****
% MonitorCall
% ****

\newpage
\subsection{MonitorCall}
\index{KERNAL Jump Table!MonitorCall}
\label{KERNAL Jump Table!MonitorCall}
\begin{description}[leftmargin=2cm,style=nextline]
    \item [Address:] JMP \$FF56
    \item [Description:] Invoke monitor
    \item [Remarks:]
        This stops the program, then invokes the monitor.

        This routine does not return. Only use JMP to call this routine.

        When the monitor exits, it returns to the \texttt{READY.} prompt.
    \item [Example:]
        \begin{asmcode}
monitor_call = $ff56

    jmp monitor_call
        \end{asmcode}
\end{description}


% ****
% OPEN
% ****

\newpage
\subsection{OPEN}
\index{KERNAL Jump Table!OPEN}
\label{KERNAL Jump Table!OPEN}
\begin{description}[leftmargin=2cm,style=nextline]
    \item [Address:] JSR \$FFC0
    \item [Description:] Open logical file
    \item [Outputs:]
        \textbf{C flag} Set on error \\
        \textbf{A} Error code, if error
    \item [Remarks:]
        Before calling OPEN, the program must call SETBNK, SETLFS, and SETNAM to set parameters for the file to open.

        The requested logical file number (LA) must be currently un-opened. Use LKUPLA to test for unused logical file numbers, if necessary. There can be up to ten logical files opened simultaneously.

        OPEN performs device-specific opening tasks for serial, RS-232, and keyboard and screen devices, including clearing the previous status and transmitting any given filename or command string supplied to SETNAM and SETBNK.

        OPEN sets the I/O status as appropriate. This can be read with READSS.

        An open file can be made the current input device with CHKIN, or the current output device with CKOUT. BASIN and GETIN read from the input device, and BSOUT writes to the output device.

        A program must close an open file when it is done using it. To close a specific file, use CLOSE. To close all files for a device, use CLOSE\_ALL. To close all open files, use CLALL. When closing a disk command channel, set the Carry flag before calling CLOSE; see CLOSE for more information.

        The possible error codes returned by BASIN, BSOUT, CHKIN, CKOUT, CLOSE, LOAD, OPEN, SAVE, and SAVEFL are as follows:

        \begin{center}
        \begin{tabular}{|c|l|}
        \hline
        \textbf{Error code} & \textbf{Meaning} \\
        \hline
        1 & Too many files \\
        \hline
        2 & File open \\
        \hline
        3 & File not open \\
        \hline
        4 & File not found \\
        \hline
        5 & Device not present \\
        \hline
        6 & Not input file \\
        \hline
        7 & Not output file \\
        \hline
        8 & Missing file name \\
        \hline
        9 & Bad device number \\
        \hline
        \end{tabular}
        \end{center}

        \underline{NOTE}: Not all routines are capable of detecting all errors. In particular, OPEN does not detect a ``File not found'' error. This condition is not detected until the program tries to read from the file. Some error conditions are only reported by READSS after an operation. Disk drives may report other error conditions over the command channel.

        The behavior of OPEN can be overridden or extended with the IOPEN vector.

    \item [Examples:]
        Disk drives accept arguments in the filename string set with SETNAM, following the filename and delimited with commas. The following example opens a file of type SEQ for writing (\texttt{"...,s,w"}), then writes the bytes 0 to 255. (Error checking is elided for brevity.)

        \begin{asmcode}
open = $ffc0
close = $ffc3
ckout = $ffc9
bsout = $ffd2
setbnk = $ff6b
setlfs = $ffba
setnam = $ffbd

    lda #$00    ; memory in bank 0
    ldx #$00    ; filename in bank 0
    jsr setbnk

    lda #$01    ; logical file number
    ldx #$08    ; device number
    ldy #$02    ; secondary address
                ; (not 0, 1, or 15)
    jsr setlfs

    lda #filename_end-filename
    ldx #<filename
    ldy #>filename
    jsr setnam

    jsr open
    ldx #$01
    jsr ckout

    lda #0
-   jsr bsout
    inc
    bne -

    clc
    lda #$01
    jsr close

    rts

filename:
    !pet "myseqfile,s,w"
filename_end:
        \end{asmcode}

        Disk drives have a \emph{command channel} that can be accessed with the secondary address 15. When a logical file is opened to the drive's SA of 15, the filename is the text of the command to perform. The command is performed immediately. The following example is similar to the BASIC command \texttt{OPEN 1,8,15,"I0"}, which sends the initialization command to disk drive device 8.

        \begin{asmcode}
open = $ffc0
close = $ffc3
setbnk = $ff6b
setlfs = $ffba
setnam = $ffbd

    lda #cmd_initialise_end-cmd_initialise
    ldx #<cmd_initialise
    ldy #>cmd_initialise
    jsr setnam

    ldx #0
    jsr setbnk

    lda #1
    ldx #8
    ldy #15
    jsr setlfs

    jsr open    ; perform I0 command
    bcs error

    lda #1
    sec         ; "special close" (see CLOSE)
    jsr close

    ; ...

cmd_initialise:
    !pet "i0"
cmd_initialise_end:
        \end{asmcode}
\end{description}


% ****
% PFKEY
% ****

\newpage
\subsection{PFKEY}
\index{KERNAL Jump Table!PFKEY}
\label{KERNAL Jump Table!PFKEY}
\begin{description}[leftmargin=2cm,style=nextline]
    \item [Address:] JSR \$FF68
    \item [Description:] Program an editor function key
    \item [Inputs:]
        \textbf{A} Base page pointer to string address (lo/hi/bank) \\
        \textbf{X} Function key number (1-16) \\
        \textbf{Y} String length
    \item [Outputs:]
        \textbf{C flag} Set if the update failed
    \item [Remarks:]
        The screen editor supports user-defined macros associated with function keys. The PFKEY call updates a macro's definition, similar to the \textbf{KEY} BASIC command.

        As input, the program sets the accumulator to a base page address that contains the 24-bit address of the macro string, in little-endian format. The Y register is the string's length, up to 240 characters.

        The X register is the key to define: 1 -- 14 for the corresponding numbered function key, 15 for the Help key, and 16 for the Stop (Shift + Run/Stop) key.

        The string contains the PETSCII codes that the macro will type when executed. This is unlike the BASIC string expression that you would use with the \textbf{KEY} command: special characters like double-quotes must appear as PETSCII codes in the string used with PFKEY. (See the example below.)

        If successful, PFKEY copies the string into macro memory, and does not need the string at the provided address to persist.
    \item [Example:]
        \begin{asmcode}
pfkey = $ff68

    ; Store string address in base page $fd-$ff
    lda #<macro
    sta $fd
    lda #>macro
    sta $fe
    lda #0
    sta $ff

    ; Install macro string for F6 key
    lda #$fd
    ldy #macro_end-macro
    ldx #6
    jsr pfkey

    rts

macro:
    !pet "print ",34,"it works!",34,13
macro_end:
        \end{asmcode}
\end{description}


% ****
% PHOENIX
% ****

\newpage
\subsection{PHOENIX}
\index{KERNAL Jump Table!PHOENIX}
\label{KERNAL Jump Table!PHOENIX}
\begin{description}[leftmargin=2cm,style=nextline]
    \item [Address:] JMP \$FF5C
    \item [Description:] Call disk boot loader
    \item [Remarks:]
        See BOOT\_SYS for a description of boot disks.

        Some early Commodore 65 prototype ROMs replaced this routine with one that performed diagnostics and printed results. Later versions replaced this with a call to the boot sequence.

        This routine does not return. Only use JMP with this routine.

        \underline{NOTE}: For the MEGA65, this feature is untested.

    \item [Example:]
        \begin{asmcode}
phoenix = $ff5c

    jmp phoenix
        \end{asmcode}

\end{description}


% ****
% PLOT
% ****

\newpage
\subsection{PLOT}
\index{KERNAL Jump Table!PLOT}
\label{KERNAL Jump Table!PLOT}
\begin{description}[leftmargin=2cm,style=nextline]
    \item [Address:] JSR \$FFF0
    \item [Description:] Read/set cursor position
    \item [Inputs:]
        To set: \\
        \textbf{C flag} = 0 \\
        \textbf{X} Cursor line \\
        \textbf{Y} Cursor column

        To read: \\
        \textbf{C flag} = 1
    \item [Outputs:]
        When setting: \\
        \textbf{C flag} Set on error

        When reading: \\
        \textbf{X} Cursor line \\
        \textbf{Y} Cursor column
    \item [Remarks:]
        The cursor position is relative to the active window. Use SCRORG to read the current window size.

        When setting, if the requested position is outside the active window, the routine returns with the Carry flag set.

        No guarantees are made about the state of X and Y after setting the cursor position.
    \item [Example:]
        \begin{asmcode}
plot = $fff0
primm = $ff7d

    ; Move the cursor to the 78th column
    ; of the 3rd row
    clc
    ldy #77
    ldx #2
    jsr plot

    jsr primm
    !pet "message",0

    ; Read the new cursor position
    sec
    jsr plot
    stx $1600
    sty $1601
        \end{asmcode}
\end{description}


% ****
% PRIMM
% ****

\newpage
\subsection{PRIMM}
\index{KERNAL Jump Table!PRIMM}
\label{KERNAL Jump Table!PRIMM}
\begin{description}[leftmargin=2cm,style=nextline]
    \item [Address:] JSR \$FF7D
    \item [Description:] Print an inline null-terminated short string.
    \item [Remarks:]
        The string to print immediately follows the JSR instruction in memory.

        The string must end in a null (\$00) byte, and must be at most 255 bytes in length including the null byte.

        Characters are written to the output stream, as with BSOUT.

        Only use JSR to call PRIMM.

        PRIMM preserves registers. This makes it suitable for dropping in the middle of a program temporarily for troubleshooting purposes.
    \item [Example:]
        \begin{asmcode}
primm = $ff7d

    jsr primm
    !pet "this is a petscii string to print.",0

    ; Program continues...
        \end{asmcode}
\end{description}


% ****
% RAMTAS
% ****

\newpage
\subsection{RAMTAS}
\index{KERNAL Jump Table!RAMTAS}
\label{KERNAL Jump Table!RAMTAS}
\begin{description}[leftmargin=2cm,style=nextline]
    \item [Address:] JSR \$FF87
    \item [Description:] Initialise RAM and buffers
    \item [Remarks:]
        RAMTAS resets KERNAL base page variables, and resets pointers to the top and bottom of system RAM.

        This is primarily for use by the system start-up and reset sequences. A machine code program executed from a BASIC bootstrap program cannot cleanly return to BASIC after calling RAMTAS, nor can RAMTAS be called from the monitor.
    \item [Example:]
        \begin{asmcode}
ramtas = $ff87

    jsr ramtas
        \end{asmcode}
\end{description}


% ****
% RDTIM
% ****

\newpage
\subsection{RDTIM}
\index{KERNAL Jump Table!RDTIM}
\label{KERNAL Jump Table!RDTIM}
\begin{description}[leftmargin=2cm,style=nextline]
    \item [Address:] JSR \$FFDE
    \item [Description:] Read CIA1 24-hour clock
    \item [Outputs:]
        \textbf{Y} Hours, 0-23, in BCD \\
        \textbf{X} Minutes, 0-59, in BCD \\
        \textbf{A} Seconds, 0-59, in BCD \\
        \textbf{Z} Tenths of a second, 0-9
    \item [Remarks:]
        This reads the time-of-day (TOD) clock feature of the CIA1 device. This clock counts tenths of seconds, and maintains hours, minutes, seconds, and tenths of seconds.

        Use SETTIM to set the value of the TOD clock.

        For the MEGA65, the CIA1 TOD clock has no relationship to the battery-backed Real-Time Clock (RTC). The value returned by RDTIM is the CIA1 TOD value, not the RTC value. BASIC 65's TI\$ and DT\$ special variables use the RTC, and not the TOD.

        Values are in Binary Coded Decimal format, where each nibble represents a decimal digit. For example, when RDTIM returns the hexadecimal value \$59 in the accumulator, this represents a value of 59 (decimal) seconds.
    \item [Example:]
        \begin{asmcode}
rdtim = $ffde
bsout = $ffd2

    ; Print hours and minutes, as HH:MM
    jsr rdtim
    tya
    jsr print_bcd
    lda #':'
    jsr bsout
    txa
    jsr print_bcd
    rts

print_bcd:
    pha
    lsr
    lsr
    lsr
    lsr
    adc #'0'
    jsr bsout
    pla
    and #$0f
    adc #'0'
    jsr bsout
    rts
        \end{asmcode}
\end{description}


% ****
% READSS
% ****

\newpage
\subsection{READSS}
\index{KERNAL Jump Table!READSS}
\label{KERNAL Jump Table!READSS}
\begin{description}[leftmargin=2cm,style=nextline]
    \item [Address:] JSR \$FFB7
    \item [Description:] Get status of last I/O operation
    \item [Outputs:]
        \textbf{A} The status byte
    \item [Remarks:]
        This returns the status of the last I/O operation performed for a serial or RS-232 device. This only works with high-level I/O operations such as OPEN, LOAD, and SAVE, as well as reading and writing to channels opened with OPEN.

        A common use of READSS is to determine the end of a file or stream.

        In the case of an RS-232 device, READSS clears the status byte and returns its previous value.

        Status bits are set for serial devices as follows:

        \begin{center}
        \begin{tabular}{|c|l|}
        \hline
        \textbf{Bit} & \textbf{Meaning} \\
        \hline
        0 & Write time out \\
        \hline
        1 & Read time out \\
        \hline
        6 & End of input \\
        \hline
        7 & Device not present \\
        \hline
        \end{tabular}
        \end{center}

        READSS is comparable to ``READST'' on the Commodore 64.
    \item [Example:]
        \begin{asmcode}
readss = $ffb7

    ldx #0
loop:
    jsr basin
    sta buffer,x
    inx

    jsr readss
    beq loop    ; read successful, continue
    and #$40
    bne eof     ; status is end of file

    ; handle other error...

eof:
    ; ...
    \end{asmcode}
\end{description}


% ****
% RESET_RUN
% ****

\newpage
\subsection{RESET{\_}RUN}
\index{KERNAL Jump Table!RESET\_RUN}
\label{KERNAL Jump Table!RESET_RUN}
\begin{description}[leftmargin=2cm,style=nextline]
    \item [Address:] JSR \$FF32
    \item [Description:] Reset the computer, with the option of running a program
    \item [Inputs:]
        \textbf{A} The reset mode

    \item [Remarks:]
        Select a reset mode by setting the accumulator {\bf A} to the appropriate value:
        \begin{itemize}
            \item {\bf A=0} KERNAL warm boot.
            \item {\bf A=1} Hypervisor reset.
            \item {\bf A=2} KERNAL warm boot, then {\bf RUN} the BASIC program in memory.
        \end{itemize}

        A KERNAL warm boot restarts from the MEGA65 KERNAL boot process, including resetting KERNAL state, and displaying the BASIC banner. Reset mode 0 also attempts to auto-boot the mounted disk.

        A Hypervisor reset triggers the Hypervisor boot process, similar to pressing the reset button on the side of the computer. This performs a few additional boot steps, including displaying the MEGA65 start-up screen and re-loading the ROM from the SD card. It does not reset all memory, but it may reset more registers than a KERNAL warm boot.

        Reset mode 2 modifies the KERNAL boot process to {\bf RUN} the BASIC program stored in memory (starting at address \$2001). This is useful for programs that launch other arbitrary programs, such as kiosk menus or file selectors. In these modes, the KERNAL does not perform the {\bf NEW} command, and does not auto-boot the mounted disk.

        Reset modes 0 through 2 ignore the values in CPU registers {\bf X}, {\bf Y}, and {\bf Z}. Additional modes that accept parameters in these registers may be added in the future. All values of {\bf A} greater than 2 are reserved for future expansion.

        This routine does not return.
    \item [Example:]
    \begin{asmcode}
reset_run = $ff32

    lda #1         ; Hypervisor reset
    jmp reset_run
    \end{asmcode}
\end{description}


% ****
% RESTOR
% ****

\newpage
\subsection{RESTOR}
\index{KERNAL Jump Table!RESTOR}
\label{KERNAL Jump Table!RESTOR}
\begin{description}[leftmargin=2cm,style=nextline]
    \item [Address:] JSR \$FF8A
    \item [Description:] Initialise KERNAL vector table
    \item [Remarks:]
        This resets all KERNAL-related vectors to their default values, pointing to routines in ROM.

        It does not reset vectors used by the screen editor and BASIC.

        It must be called with interrupts disabled.

        See VECTOR.
    \item [Example:]
    \begin{asmcode}
restor = $ff8a

    sei
    jsr restor
    cli
    \end{asmcode}
\end{description}


% ****
% SAVE
% ****

\newpage
\subsection{SAVE}
\index{KERNAL Jump Table!SAVE}
\label{KERNAL Jump Table!SAVE}
\begin{description}[leftmargin=2cm,style=nextline]
    \item [Address:] JSR \$FFD8
    \item [Description:] Save to file
    \item [Inputs:]
        \textbf{A} Base page pointer to start address \\
        \textbf{X} End address + 1, low \\
        \textbf{Y} End address + 1, high
    \item [Outputs:]
        \textbf{C flag} Set on error \\
        \textbf{A} Error code, if error
    \item [Remarks:]
        Before calling SAVE, the program must call SETBNK, SETLFS, and SETNAM to set parameters for the file to load. SAVE cannot be used with RS-232 devices, the screen, or the keyboard. The logical file number is not used.

        When used with a disk drive, SAVE saves a file of type PRG by default. To save a file of another type, use SETNAM with the type mentioned in the name string (such as \texttt{"MYFILE,S"} for a SEQ file), along with OPEN, CKOUT, BSOUT, and CLOSE. See OPEN for an example.

        SAVE always writes the starting address as the first two bytes, regardless of the file type. If the data is in a bank other than zero, that information is not stored. To omit the address bytes from the file, use SAVEFL with the appropriate flag.

        By default, saving a file with the name of an existing file is an error, and does not change the file. To overwrite the file if it exists, prefix the filename string with: \texttt{@:} For example: \texttt{"@:MYFILE"}

        SAVE writes a region of memory to a file on a serial device. The start address of the memory region must be stored as a 16-bit address, little-endian, on the base page. The program provides the base page address of this pointer in the A register.

        The program provides the ending address of the memory region as a 16-bit value \emph{plus one} in the X and Y registers.

        The complete memory region must fit within a single bank. It cannot cross bank boundaries.

        SAVE performs all of the tasks needed to open the file, write it, then close it. It does not leave any logical files open (as OPEN does).

        SAVE sets the I/O status as appropriate. This can be read with READSS.

        For the list of possible error codes, see OPEN.

        This behavior can be overridden or extended with the ISAVE vector.
    \item [Example:]
        \begin{asmcode}
save = $ffd8
setbnk = $ff6b
setlfs = $ffba
setnam = $ffbd

    ; fill memory with data to test saving
    ldx #0
-   txa
    sta $4500,x
    inx
    bne -

    lda #$00    ; memory in bank 0
    ldx #$00    ; filename in bank 0
    jsr setbnk

    lda #$00    ; (not used by save)
    ldx #$08    ; device number
    ldy #$01    ; save
    jsr setlfs

    lda #filename_end-filename
    ldx #<filename
    ldy #>filename
    jsr setnam

    ; Save 0.4500 to 0.45FF
    lda #$00
    sta $fe
    lda #$45
    sta $ff
    lda #$fe
    ldx #$00
    ldy #$46

    jsr save
    bcs error

    ; ...

filename:
    !pet "myfile"
filename_end:
        \end{asmcode}
\end{description}

% ****
% SAVEFL
% ****

\newpage
\subsection{SAVEFL}
\index{KERNAL Jump Table!SAVEFL}
\label{KERNAL Jump Table!SAVEFL}
\begin{description}[leftmargin=2cm,style=nextline]
    \item [Address:] JSR \$FF3B
    \item [Description:] Save to file, with flags
    \item [Inputs:]
        \textbf{A} Base page pointer to start address \\
        \textbf{X} End address + 1, low \\
        \textbf{Y} End address + 1, high \\
        \textbf{Z} Flags: bit 6 = raw mode
    \item [Outputs:]
        \textbf{C flag} Set on error \\
        \textbf{A} Error code, if error
    \item [Remarks:]
        SAVEFL is identical to SAVE, with the additional flags argument in the Z register. See SAVE for more information.

        This currently supports one flag: if bit 6 is set, SAVEFL omits the two-byte address header from the saved file. This is symmetric with the ``raw'' mode of the LOAD routine.
\end{description}


% ****
% ScanStopKey
% ****

\newpage
\subsection{ScanStopKey}
\index{KERNAL Jump Table!ScanStopKey}
\label{KERNAL Jump Table!ScanStopKey}
\begin{description}[leftmargin=2cm,style=nextline]
    \item [Address:] JSR \$FFEA
    \item [Description:] Scan Stop key
    \item [Outputs:]
        \textbf{A} \$7F if the Stop key is pressed, \$FF otherwise \\
        \textbf{N flag} Clear if the Stop key is pressed, set otherwise
    \item [Remarks:]
        Unlike STOP, ScanStopKey does not invoke the ISTOP vector, and does not close channels or flush the keyboard buffer.

        ScanStopKey has the same address as the Commodore 64 KERNAL routine ``UDTIM.'' UDTIM is not available on the MEGA65.
    \item [Example:]
        \begin{asmcode}
scan_stop_key = $ffea

    ; Wait for Stop key
-   jsr scan_stop_key
    bmi -
        \end{asmcode}
\end{description}


% ****
% SCRORG
% ****

\newpage
\subsection{SCRORG}
\index{KERNAL Jump Table!SCRORG}
\label{KERNAL Jump Table!SCRORG}
\begin{description}[leftmargin=2cm,style=nextline]
    \item [Address:] JSR \$FFED
    \item [Description:] Get current screen window size
    \item [Outputs:]
        \textbf{C flag} Text screen width: 0=80, 1=40 \\
        \textbf{X} Window width \\
        \textbf{Y} Window height \\
        \textbf{A} Window top-left screen memory address, low \\
        \textbf{Z} Window top-left screen memory address, high
    \item [Remarks:]
        The screen editor maintains a ``window,'' a rectangle on the text screen, where terminal output is written and input is accepted. The window size and position defaults to the full screen, and can be adjusted with the \textbf{WINDOW} BASIC command.

        SCRORG returns properties of this window that can be used to plot characters in screen memory.

        SCRORG is comparable to ``SCREEN'' on the Commodore 64.

    \item [Example:]
        \begin{asmcode}
scrorg = $ffed

    ; Plot a spade glyph in the top left corner
    ; of the active window.
    jsr scrorg
    sta $fe
    stz $ff
    lda #65
    ldy #0
    sta ($fe),y
        \end{asmcode}

\end{description}


% ****
% SECND
% ****

\newpage
\subsection{SECND}
\index{KERNAL Jump Table!SECND}
\label{KERNAL Jump Table!SECND}
\begin{description}[leftmargin=2cm,style=nextline]
    \item [Address:] JSR \$FF93
    \item [Description:] Send secondary address to listener
    \item [Inputs:]
        \textbf{A} Secondary address (SA)
    \item [Remarks:]
        This is a low-level serial routine. Most programs will prefer the higher level I/O routines (OPEN et al.).

        After sending the LISTN command to a serial device to tell it to accept input, SECND sends a secondary address to the device. This is typically used to configure the device prior to sending it data. The behavior depends on the device.

        To send a secondary address to a TALKing device, use TKSA.

        This behavior can be overridden or extended with the ISECOND vector.

        SECND is comparable to ``SECOND'' on the Commodore 64.

    \item [Example:]
        \begin{asmcode}
listn = $ffb1
secnd = $ff93

    lda #8
    jsr listn
    lda #15
    jsr secnd
        \end{asmcode}
\end{description}


% ****
% SETBNK
% ****

\newpage
\subsection{SETBNK}
\index{KERNAL Jump Table!SETBNK}
\label{KERNAL Jump Table!SETBNK}
\begin{description}[leftmargin=2cm,style=nextline]
    \item [Address:] JSR \$FF6B
    \item [Description:] Set megabyte and bank for I/O and filename memory address
    \item [Inputs:]
        For banks 0 -- 5: \\
        \textbf{A} The memory bank \\
        \textbf{X} The filename bank

        For 28-bit addresses: \\
        \textbf{A} Bit 7 set, bits 0 -- 3 = bits 24 -- 27 of memory address \\
        \textbf{Y} Bits 16 -- 23 of memory address \\
        \textbf{X} Bit 7 set, bits 0 -- 3 = bits 24 -- 27 of filename address \\
        \textbf{Z} Bits 16 -- 23 of filename address
    \item [Remarks:]
        SETBNK must be used for any memory I/O operation, along with SETLFS and SETNAM for opening files. The lower two bytes of each address are arguments to other calls.

        For backwards compatibility with the C65 KERNAL, SETBNK has two separate modes, one for banks 0 -- 5 in the first megabyte of memory, and another for 28-bit addresses.

        If the first register's bit 7 is clear, it represents an address in the first megabyte, and the value is the bank number. For example, if A = \$04, then the memory address is in bank 4 of the first megabyte (\$004.xxxx).

        If the first register's bit 7 is set, it represents a 28-bit address, and the lower nibble of the first register is the top-most nibble of the address. The second register represents the next two nibbles of the address. For example, if A = \$88 and Y = \$03, then the memory address is a 28-bit address of the form \$803.xxxx.

        The memory address is set with input registers A and Y. The filename address is set with input registers X and Z. The memory bank and filename bank can be different, and can use bank mode or 28-bit mode independently.

        See OPEN for a complete example.
    \item [Example:]
        \begin{asmcode}
setbnk = $ff6b

    ; Memory address:     $4.xxxx
    ; Filename address:   $0.xxxx
    lda #$04
    ldx #$00
    jsr setbnk

    ; Memory address:   $803.xxxx
    ; Filename address:   $0.xxxx
    lda #$88
    ldy #$03
    ldx #$00
    jsr setbnk

    ; Memory address:   $803.xxxx
    ; Filename address: $805.xxxx
    lda #$88
    ldy #$03
    ldx #$88
    ldz #$05
    jsr setbnk
        \end{asmcode}
\end{description}


% ****
% SETLFS
% ****

\newpage
\subsection{SETLFS}
\index{KERNAL Jump Table!SETLFS}
\label{KERNAL Jump Table!SETLFS}
\begin{description}[leftmargin=2cm,style=nextline]
    \item [Address:] JSR \$FFBA
    \item [Description:] Set file, device, secondary address
    \item [Inputs:]
        \textbf{A} The logical file number (LA) \\
        \textbf{X} The device number (FA) \\
        \textbf{Y} The secondary address, or \$FF (SA)
    \item [Remarks:]
        SETLFS must be used for high-level file operations (OPEN, LOAD, SAVE), along with SETBNK and SETNAM.

        The logical file number must be unique among opened files. If used with OPEN, it must not yet be open.

        If the secondary address does not apply, set Y to \$FF. For disk drives, secondary addresses 0 and 1 are reserved for LOAD and SAVE, and 15 is the command channel.

        See OPEN for a complete example.
    \item [Example:]
        \begin{asmcode}
setlfs = $ffba

    lda #$01
    ldx #$08
    ldy #$02
    jsr setlfs
        \end{asmcode}
\end{description}


% ****
% SETMSG
% ****

\newpage
\subsection{SETMSG}
\index{KERNAL Jump Table!SETMSG}
\label{KERNAL Jump Table!SETMSG}
\begin{description}[leftmargin=2cm,style=nextline]
    \item [Address:] JSR \$FF90
    \item [Description:] Enable/disable KERNAL messages
    \item [Inputs:]
        \textbf{A} bit 7 = control messages, bit 6 = error messages
    \item [Remarks:]
        The KERNAL can print messages to the screen when performing disk operations, such as ``LOADING,'' ``SAVING,'' ``VERIFYING,'' and ``I/O ERROR.'' These are disabled by default.

        To enable KERNAL control messages, set bit 7. To enable KERNAL I/O error messages, set bit 6.

        KERNAL messages are not the same as those output by BASIC.
    \item [Example:]
        \begin{asmcode}
setmsg = $ff90

    ; Enable all KERNAL messages
    lda #$c0
    jsr setmsg
        \end{asmcode}

\end{description}


% ****
% SETNAM
% ****

\newpage
\subsection{SETNAM}
\index{KERNAL Jump Table!SETNAM}
\label{KERNAL Jump Table!SETNAM}
\begin{description}[leftmargin=2cm,style=nextline]
    \item [Address:] JSR \$FFBD
    \item [Description:] Set filename pointers
    \item [Inputs:]
        \textbf{A} The string length, or \$00 \\
        \textbf{X} The string address, low \\
        \textbf{Y} The string address, high
    \item [Remarks:]
        SETNAM must be used for high-level file operations (OPEN, LOAD, SAVE), along with SETBNK and SETLFS.

        If a filename is not appropriate for the I/O device, SETNAM must be called with A = \$00 (and any values for X and Y).

        The bank for the address is set by SETBNK.

        See OPEN for a complete example.
    \item [Example:]
        \begin{asmcode}
setnam = $ffbd

    lda #filename_end-filename
    ldx #<filename
    ldy #>fileanme
    jsr setnam

    ; ...

filename:
    !pet "myfile"
filename_end:
        \end{asmcode}
\end{description}


% ****
% SETTIM
% ****

\newpage
\subsection{SETTIM}
\index{KERNAL Jump Table!SETTIM}
\label{KERNAL Jump Table!SETTIM}
\begin{description}[leftmargin=2cm,style=nextline]
    \item [Address:] JSR \$FFDB
    \item [Description:] Set CIA1 24-hour clock
    \item [Inputs:]
        \textbf{Y} Hours, 0-23, in BCD \\
        \textbf{X} Minutes, 0-59, in BCD \\
        \textbf{A} Seconds, 0-59, in BCD \\
        \textbf{Z} Tenths of a second, 0-9
    \item [Remarks:]
        This sets the time-of-day (TOD) clock feature of the CIA1 device. This clock counts tenths of seconds, and maintains hours, minutes, seconds, and tenths of seconds.

        Use RDTIM to read the current value of the TOD clock.

        For the MEGA65, the CIA1 TOD clock has no relationship to the battery-backed Real-Time Clock (RTC). Setting the TOD clock does not affect the RTC. BASIC 65's TI\$ and DT\$ special variables use the RTC, and not the TOD.

        Values are in Binary Coded Decimal format, where each nibble represents a decimal digit. For example, to set the seconds value to 59 seconds, use the hexadecimal value \$59.
    \item [Example:]
        \begin{asmcode}
settim = $ffdb

    ; Set the TOD clock to 14:30:57.5
    ; (aka 2:30:57.5 PM)
    ldy #$14
    ldx #$30
    lda #$57
    ldz #$05
    jsr settim
        \end{asmcode}
\end{description}


% ****
% SPIN\_SPOUT
% ****

\newpage
\subsection{SPIN{\_}SPOUT}
\index{KERNAL Jump Table!SPIN\_SPOUT}
\label{KERNAL Jump Table!SPIN_SPOUT}
\begin{description}[leftmargin=2cm,style=nextline]
    \item [Address:] JSR \$FF4D
    \item [Description:] Set up fast serial ports
    \item [Inputs:]
        \textbf{C flag} Clear for SPINP, set for SPOUT
    \item [Remarks:]
    This routine relates to low-level driving of the C1571/1581 ``fast serial'' protocol.

    From the C65 specification:

    The  fast serial protocol utilises CIA\_1 (6526 at \$DC00) and a special
    driver  circuit  controlled  in  part by the FSDIR register. SPINP and
    SPOUT  are  routines  used  by  the  system to set up the CIA and fast
    serial  driver  circuit  for input or output. SPINP sets up CRA (CIA\_1
    register  14)  and  clears the FSDIR bit for input. SPOUT sets up CRA,
    ICR (CIA\_1 register 13), timer A (CIA\_l registers 4 \& 5), and sets the
    FSDIR  bit  for  output.  Note  the  state of the TOD\_IN bit of CRA is
    always  preserved.  These  routines  are required only by applications
    driving the fast serial bus themselves from the lowest level.
    \item [Example:]
        \begin{asmcode}
spin_spout = $ff4d

    clc
    jsr $ff4d
        \end{asmcode}
\end{description}


% ****
% STA\_FAR
% ****

\newpage
\subsection{STA{\_}FAR}
\index{KERNAL Jump Table!STA\_FAR}
\label{KERNAL Jump Table!STA_FAR}
\begin{description}[leftmargin=2cm,style=nextline]
    \item [Address:] JSR \$FF77
    \item [Description:] Store a byte to an address in any bank
    \item [Inputs:]
        \textbf{A} Data to store \\
        \textbf{X} Base page pointer \\
        \textbf{Y} Index from pointed address \\
        \textbf{Z} Bank
    \item [Remarks:]
        Conceptually, this is the equivalent of \texttt{sta (X),y} in bank Z.

        This can access any address in the first 1MB of memory. Unlike JMPFAR, this does not change the memory map. It uses a DMA job to access the long address.

        For the MEGA65, the 45GS02 supports an instruction that can do this with a 32-bit address stored on the base page, without the need for a KERNAL routine: \texttt{sta [zp4],z}
    \item [Example:]
            \begin{asmcode}
sta_far = $ff77

    ; Store value to address 4.4502
    lda #$00
    sta $fe
    lda #$45
    sta $ff
    ldx #$fe  ; base page pointer -> $4500
    ldy #$02  ; Y index
    ldz #$04  ; bank
    lda #$99  ; the value to store
    jsr sta_far
            \end{asmcode}

\end{description}


% ****
% STOP
% ****

\newpage
\subsection{STOP}
\index{KERNAL Jump Table!STOP}
\label{KERNAL Jump Table!STOP}
\begin{description}[leftmargin=2cm,style=nextline]
    \item [Address:] JSR \$FFE1
    \item [Description:] Report Stop key, and reset I/O if pressed
    \item [Outputs:]
        \textbf{Z flag} Set if Stop is pressed
    \item [Remarks:]
        If the Stop key (Shift + Run/Stop) was pressed in the most recent call to ScanStopKey, STOP sets the Zero (Z) flag, closes all active channels, and flushes the keyboard queue. ScanStopKey is called routinely by the KERNAL, or can be called explicitly by the program if the KERNAL is disabled.

        This behavior can be overridden or extended with the ISTOP vector.

        To test for the Stop key without side effects, see ScanStopKey.
    \item [Example:]
        \begin{asmcode}
stop = $ffe1

    ; Wait for the Stop key
-   jsr stop
    bne -
        \end{asmcode}
\end{description}


% ****
% SWAPPER
% ****

\newpage
\subsection{SWAPPER}
\index{KERNAL Jump Table!SWAPPER}
\label{KERNAL Jump Table!SWAPPER}
\begin{description}[leftmargin=2cm,style=nextline]
    \item [Address:] JSR \$FF65
    \item [Description:] Toggle between 40x25 and 80x25 text modes
    \item [Remarks:]
        This does not yet support 80x50 text mode.
    \item [Example:]
        \begin{asmcode}
swapper = \$ff65

    jsr swapper
        \end{asmcode}
\end{description}


% ****
% SYSFLAGS
% ****

\newpage
\subsection{SYSFLAGS}
\index{KERNAL Jump Table!SYSFLAGS}
\label{KERNAL Jump Table!SYSFLAGS}
\begin{description}[leftmargin=2cm,style=nextline]
    \item [Address:] JSR \$FF47
    \item [Description:] Read/set system flags
    \item [Inputs:]
        To set: \\
        \textbf{C flag} = 0 \\
        \textbf{A} The new flags value

        To read: \\
        \textbf{C flag} = 1
    \item [Outputs:]
        When reading:
        \textbf{A} The flags value
    \item [Remarks:]
        Locks bits: \\
        \textbf{Bit 4:} GO64 mode (0=unlocked, 1=locked) \\
        \textbf{Bit 5:} Function keys (0=macros, 1=raw keys) \\
        \textbf{Bit 6:} No Scroll locked (0=unlocked, 1=locked) \\
        \textbf{Bit 7:} Mega+Shift locked (0=unlocked, 1=locked)

        Bits 0--3 are reserved for future use. To maintain future compatibility, always read the flags value and preserve the values of these bits.

        Formerly known as ``KEYLOCKS.''

    \item [Example:]
        \begin{asmcode}
sysflags = $ff47

    sec
    jsr sysflags
    ora #%00100000  ; Disable function key macros
    clc
    jsr sysflags
        \end{asmcode}

\end{description}


% ****
% TALK
% ****

\newpage
\subsection{TALK}
\index{KERNAL Jump Table!TALK}
\label{KERNAL Jump Table!TALK}
\begin{description}[leftmargin=2cm,style=nextline]
    \item [Address:] JSR \$FFB4
    \item [Description:] Send ``talk'' command to a device
    \item [Inputs:]
        \textbf{A} The device number (FA, 0-31)
    \item [Remarks:]
        This is a low-level serial routine. Most programs will prefer the higher level I/O routines (OPEN et al.).

        With the low-level serial protocol, the computer is expected to coordinate all device communication on the serial bus. Each device has a hard-coded device number; some devices, like disk drives, may have switches for changing its recognised device number. All devices see all traffic on the bus, and are expected to change their mode when they see a command with their device number.

        When the computer wants to read data from a device, the computer sends the device the ``talk'' command. Only one device at a time has permission to ``talk'' on the serial bus. To change talkers, the computer broadcasts an ``untalk'' command to all devices, then sends another ``talk'' command with a different device number.

        Use UNTLK to tell all devices to stop talking.

        This behavior can be overridden or extended with the ITALK vector.
    \item [Example:]
        \begin{asmcode}
talk = $ffb4

    lda #8
    jsr talk
        \end{asmcode}
\end{description}


% ****
% TKSA
% ****

\newpage
\subsection{TKSA}
\index{KERNAL Jump Table!TKSA}
\label{KERNAL Jump Table!TKSA}
\begin{description}[leftmargin=2cm,style=nextline]
    \item [Address:] JSR \$FF96
    \item [Description:] Send secondary address to talker
    \item [Inputs:]
        \textbf{A} Secondary address (SA)
    \item [Remarks:]
        This is a low-level serial routine. Most programs will prefer the higher level I/O routines (OPEN et al.).

        After sending the TALK command to a serial device to tell it to provide output, TKSA sends a secondary address to the device. This is typically used to configure the device prior to reading data. The behavior depends on the device.

        To send a secondary address to a LISTNing device, use SECND.

        This behavior can be overridden or extended with the ITALKSA vector.
    \item [Example:]
        \begin{asmcode}
talk = $ffb4
tksa = $ff96

    lda #8
    jsr talk
    lda #15
    jsr tksa
        \end{asmcode}
\end{description}


% ****
% UNLSN
% ****

\newpage
\subsection{UNLSN}
\index{KERNAL Jump Table!UNLSN}
\label{KERNAL Jump Table!UNLSN}
\begin{description}[leftmargin=2cm,style=nextline]
    \item [Address:] JSR \$FFAE
    \item [Description:] Send ``unlisten'' command
    \item [Remarks:]
        This is a low-level serial routine. Most programs will prefer the higher level I/O routines (OPEN et al.).

        UNLSN sends the ``unlisten'' command to all serial devices. See LISTN.

        This behavior can be overridden or extended with the IUNLISTEN vector.
    \item [Example:]
        \begin{asmcode}
unlsn = $ffae

    jsr unlsn
        \end{asmcode}
\end{description}


% ****
% UNTLK
% ****

\newpage
\subsection{UNTLK}
\index{KERNAL Jump Table!UNTLK}
\label{KERNAL Jump Table!UNTLK}
\begin{description}[leftmargin=2cm,style=nextline]
    \item [Address:] JSR \$FFAB
    \item [Description:] Send ``untalk'' command
    \item [Remarks:]
        This is a low-level serial routine. Most programs will prefer the higher level I/O routines (OPEN et al.).

        UNTLK sends the ``untalk'' command to all serial devices. See TALK.

        This behavior can be overridden or extended with the IUNTALK vector.
    \item [Example:]
        \begin{asmcode}
untlk = $ffab

    jsr untlk
        \end{asmcode}
\end{description}


% ****
% VECTOR
% ****

\newpage
\subsection{VECTOR}
\index{KERNAL Jump Table!VECTOR}
\label{KERNAL Jump Table!VECTOR}
\begin{description}[leftmargin=2cm,style=nextline]
    \item [Address:] JSR \$FF8D
    \item [Description:] Read/set KERNAL vector table
    \item [Inputs:]
        To read: \\
        \textbf{C flag} = 1 \\
        \textbf{X} Destination address, low \\
        \textbf{Y} Destination address, high

        To set: \\
        \textbf{C flag} = 0 \\
        \textbf{X} Source address, low \\
        \textbf{Y} Source address, high

    \item [Remarks:]
        A program can override or extend some behaviors of the KERNAL, screen editor, and BASIC by installing custom routines. The KERNAL has a fixed set of extension points, known as \emph{vectors}, where KERNAL behaviors jump to vector addresses under certain circumstances. The default values of these vectors point to routines in KERNAL ROM. A program installs a routine by replacing the original vector address with a custom routine's address, and typically (but optionally) rewriting the custom routine to call the original KERNAL vector address.

        To install new vectors:

        \begin{enumerate}
            \item Use VECTOR to copy the vector table to a location in the program's memory.
            \item As needed, copy the original vector address into the custom routine's code, such as to return control to the original KERNAL routine after performing a custom action.
            \item Write the custom routine's address into the vector table in memory.
            \item Use VECTOR to install the updated vector addresses.
        \end{enumerate}

        Some Commodore programmers are accustomed to reading and writing custom vector addresses directly from internal KERNAL memory, without using the VECTOR routine. For the MEGA65, it is important for programs to use the VECTOR accessor routine, and to not depend on the internal KERNAL memory location, to assure compatibility with future versions of the ROM.

        \underline{NOTE}: The KERNAL uses a memory map for BASIC that hides most program memory from the CPU. Custom routines must live in the range \$1600 -- \$1EFF to be visible when BASIC and the screen editor are running. This only applies to cases when the program returns control to BASIC or the screen editor, a common use of vectors. For longer custom routines, you can use a small dispatch routine that disables interrupts, changes the CPU memory map, then calls a routine somewhere else in memory. When that routine returns, the dispatch routine restores the KERNAL memory map and enables interrupts, then jumps to the KERNAL's original vector.

        When updating vectors related to interrupt handling, disable interrupts before calling VECTOR.

        When installing vectors that return control to the original KERNAL routines, it's a good practice to call RESTOR before reading the table with VECTOR. This ensures that the installation routine sees the original KERNAL addresses, and not the addresses from a previous installation.

        Use RESTOR to reset the vector table to the KERNAL defaults.

    \item [Vector table:]
        The vector table is 56 bytes, two bytes for each 16-bit vector address.

        Nearly all vectors are for KERNAL jump table routines. Replacing one of these vectors will cause all calls to the KERNAL jump table entry to instead call the custom routine. The custom routine is expected to support the documented preconditions (such as CPU register arguments) of the KERNAL routine. If it subsequently calls the KERNAL routine, it must also meet those preconditions on exit. If it doesn't call the KERNAL routine, it must meet the routine's postconditions (such as CPU register return values) on exit.

        The IIRQ, IBRK, and INMI vectors refer to the KERNAL's interrupt handlers. The KERNAL uses a once-per-frame raster IRQ. The KERNAL's interrupt handlers preserve all CPU registers on the stack, pushed in this order: A, X, Y, Z, B. For an IRQ, the handler then tests whether the IRQ is caused by the raster frame or by a BRK instruction, and JMPs to either the IIRQ or IBRK handler accordingly. For an NMI, the handler disables IRQs, then calls the INMI handler. If a custom routine calls the original KERNAL vector when it is finished, the KERNAL will restore the CPU registers, then exit the interrupt handler cleanly. If the custom routine opts to not invoke the original KERNAL vector, it must handle the five bytes on the stack.

        CTLVEC, SHFVEC, and ESCVEC allow a custom routine an opportunity to intercept printed and typed characters that involve the Control, Shift, or Escape keys. The accumulator contains the unmodified PETSCII code. A custom routine can cancel further processing by returning with RTS, or it can yield back to the KERNAL by calling the original KERNAL vector. While this is intended primarily to add interactive behaviors to the screen editor when these characters are typed, all such codes can be printed by programs to achieve these effects as well.

        KEYSCAN allows a custom routine to intercept typing events as they are read from the keyboard. This vector is called during the KERNAL's implementation of GETIN when the input device is the keyboard. The accumulator contains the unmodified PETSCII code, or \$FF to indicate no keypress. Y contains a bitmask of modifier keys; see a description of MODKEY in \bookref{cha:keyboard}. If the C flag is clear, the typing event is eligible to trigger a function key macro; if set, the event will not trigger a macro. The vector routine can read or change any of these attributes. It must return control to the original KERNAL address in the vector.

        \begin{longtable}{|L{2cm}|p{2cm}|p{4cm}|}
        \hline
        \textbf{Offset} & \textbf{Name} & \textbf{Description} \\
        \hline
        \endfirsthead
        \multicolumn{3}{l@{}}{\ldots continued}\\
        \hline
        \textbf{Offset} & \textbf{Name} & \textbf{Description} \\
        \hline
        \endhead
        \multicolumn{3}{l@{}}{continued \ldots}\\
        \endfoot
        \hline
        \endlastfoot
        \hline
        0 & IIRQ & Raster IRQ handler \\
        \hline
        2 & IBRK & BRK handler \\
        \hline
        4 & INMI & NMI handler \\
        \hline
        6 & IOPEN & OPEN \\
        \hline
        8 & ICLOSE & CLOSE \\
        \hline
        10 & ICHKIN & CHKIN \\
        \hline
        12 & ICKOUT & CKOUT \\
        \hline
        14 & ICLRCH & CLRCH \\
        \hline
        16 & IBASIN & BASIN \\
        \hline
        18 & IBSOUT & BSOUT \\
        \hline
        20 & ISTOP & STOP \\
        \hline
        22 & IGETIN & GETIN \\
        \hline
        24 & ICLALL & CLALL \\
        \hline
        26 & EXMON & Reserved, do not modify \\
        \hline
        28 & ILOAD & LOAD \\
        \hline
        30 & ISAVE & SAVE \\
        \hline
        32 & ITALK & TALK \\
        \hline
        34 & ILISTEN & LISTN \\
        \hline
        36 & ITALKSA & TKSA \\
        \hline
        38 & ISECOND & SECND \\
        \hline
        40 & IACPTR & ACPTR \\
        \hline
        42 & ICIOUT & CIOUT \\
        \hline
        44 & IUNTALK & UNTLK \\
        \hline
        46 & IUNLISTEN & UNLSN \\
        \hline
        48 & CTLVEC & Control code \\
        \hline
        50 & SHFVEC & Shift code \\
        \hline
        52 & ESCVEC & Escape code \\
        \hline
        54 & KEYSCAN & Handling keyboard input \\
        \hline
        \end{longtable}

    \item [Examples:]
        The following example installs a short routine at \$1600 that increments the screen code in the top left corner of the screen, then sets the IIRQ vector such that the KERNAL calls this routine once per raster frame. After this program returns to BASIC, the screen code updates continuously while BASIC and the screen editor remain operational.

        \begin{asmcode}
vector = $ff8d

    ; Install custom_irq routine to $1600,
    ; so that it will be visible when BASIC
    ; is active.
    ldx #custom_irq_end-custom_irq+1
-   lda custom_irq-1,x
    sta $1600-1,x
    dex
    bne -

    ; Read vector table into memory
    sei
    jsr restor
    cli
    sec
    ldx #<vectable
    ldy #>vectable
    jsr vector

    ; Copy the iirq vector to custom_irq's
    ; jmp instruction
    lda vectable
    sta $1600+custom_irq_return-custom_irq+1
    lda vectable+1
    sta $1600+custom_irq_return-custom_irq+2

    ; Write custom_irq address to iirq (the first table entry)
    lda #$00
    sta vectable
    lda #$16
    sta vectable+1

    ; Install updated vector table
    clc
    ldx #<vectable
    ldy #>vectable
    sei
    jsr vector
    cli

    ; Return to BASIC
    rts

custom_irq:
    ; Rotate the top left character
    inc $0800
custom_irq_return:
    jmp $0000
custom_irq_end:

vectable:
    !fill $38
        \end{asmcode}

        The following example uses the KEYSCAN vector to intercept the Help key. Instead of invoking the F15 function key macro, pressing the Help key changes the border colour. The PHP and PLP instructions ensure that the C flag is preserved. Only the vector routine is shown; the installation code is similar to the previous example, using \texttt{vectable+54} as the address of the KEYSCAN vector table entry.

        \begin{asmcode}
custom_keyscan:
    php
    cmp #132  ; Help key
    bne +
    lda #$ff  ; cancel "Help"
    inc $d020
+   plp
custom_keyscan_return:
    jmp $0000
custom_keyscan_end:
        \end{asmcode}

\end{description}


% ****
% VERSIONQ
% ****

\newpage
\subsection{VERSIONQ}
\index{KERNAL Jump Table!VERSIONQ}
\label{KERNAL Jump Table!VERSIONQ}
\begin{description}[leftmargin=2cm,style=nextline]
    \item [Address:] JSR \$FF2F
    \item [Description:] Load the ROM version number into Q.
    \item [Remarks:]
        This loads the ROM version number into the Q register (A, X, Y, Z), as a 32-bit integer, little endian.
    \item [Example:]
        \begin{asmcode}
versionq = $ff2f

    jsr versionq
    stq $1700
        \end{asmcode}
\end{description}


%%%%%%%%%%%%%%%%%%%%%%%%%%%%%%%%%%%%%%%%%%%%%%%%%%%%%%%%%%%%%%%%%
