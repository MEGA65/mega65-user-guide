\chapter{Machine Language Monitor}

\section{Introduction}

The machine language monitor is a debugging tool for machine language
programs. It includes a mini-assembler, a disassembler and many useful commands.
When the program execution encounters the code 00 (zero) alias BRK,
the default action of the operating system is, to call the monitor.
This features allows the debugging of programs by setting breakpoints.

\section{Table of monitor commands}

\ttfamily
{\setlength{\tabcolsep}{1mm}
\begin{tabular}{|l|l|l|}
\hline
C & mnemonic & description \\
\hline
A &     ASSEMBLE        & Assemble a line of 4502 code\\
C &     COMPARE         & Compare two sections of memory\\
D &     DISASSEMBLE     & Disassemble a line of 4502 code\\
E &     EXIT            & Exit Monitor mode\\
F &     FILL            & Fill a section of memory with a value \\
G &     GO              & Start execution at specified address\\
H &     HUNT            & Find specified data in a section of memory\\
L &     LOAD            & Load a file from disk\\
M &     MEMORY          & Dump a section of memory\\
R &     REGISTERS       & Display the contents of the 4502 registers\\
S &     SAVE            & Save a section of memory to a disk file\\
T &     TRANSFER        & Transfer memory to another location\\
V &     VERIFY          & Compare a section of memory with a disk file\\
\hline
 . &     <period>        & Assembles a line of 4502 code\\
 > &     <greater>       & Modifies memory\\
 ; &     <semicolon>     & Modifies register contents\\
 @ &     <at sign>       & Display disk status\\
\hline
\$ &     <hex>           & Display hex, decimal, octal, and binary value \\
 + &     <decimal>       & Display hex, decimal, octal, and binary value\\
\& &     <octal>         & Display hex, decimal, octal, and binary value\\
\% &     <binary>        & Display hex, decimal, octal, and binary value\\
\hline
\end{tabular}
}

\section {calling the monitor}

To enter the monitor from BASIC, type:
\screentext{MONITOR}

The monitor responds with a display of register contents and waits for a command:

\begin{tcolorbox}[colback=black,coltext=white]
\verbatimfont{\codefont}
\begin{verbatim}
    PC   SR AC XR YR ZR SP
; 000000 00 00 00 00 00 f8
\end{verbatim}
\end{tcolorbox}


