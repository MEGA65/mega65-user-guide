\chapter{PETSCII Codes}


\section{PETSCII Codes and CHR\$}

\label{appendix:asciicodes}

In BASIC,  \screentext{PRINT CHR\$(X)} can be used to print a character from a PETSCII code.
Below is the full table of PETSCII codes you can print by index.  For example, while in the default uppercase/graphics mode, by
using index 65 from the table below as: \screentext{PRINT CHR\$(65)} you will
print the letter \screentext{A}. You can read more about {\bf CHR\$} on page \pageref{BASIC 65 Functions!CHR}.

You can also do the reverse with the ASC statement.  For example:
\screentext{PRINT ASC("A")} will output \screentext{65}, which matches the
code in the table.

\underline{NOTE}: Function key (F1-F14 + HELP) values in this table are not intended to be printed via \screentext{CHR\$()},
but rather to allow function-key input to be assessed in BASIC programs via the GET / GETKEY commands.

\index{Keyboard!PETSCII Codes and CHR\$}
\begin{adjustwidth}{}{-2cm}
\begin{multicols}{4}
\begin{description}[align=left,labelwidth=0.2cm]
    \item [0]
    \item [1]   \small{ALTERNATE PALETTE}
    \item [2]   \small{UNDERLINE ON}
    \item [3]
    \item [4]   \small{DEFAULT PALETTE}
    \item [5]   \small{WHITE}
    \item [6]
    \item [7]   \small{BELL}
    \item [8]
    \item [9]   \specialkey{TAB}
    \item [10]  \small{LINEFEED}
%   \item [11]  \footnotesize{DISABLE \specialkey{SHIFT}\megasymbolkey}
    \item [11]  DISABLE \\ \specialkey{SHIFT}\megasymbolkey
%   \item [12]  \footnotesize{ENABLE \specialkey{SHIFT}\megasymbolkey}
    \item [12]  ENABLE \\ \specialkey{SHIFT}\megasymbolkey
    \item [13]  \specialkey{RETURN}
    \item [14]  \small{LOWER CASE}
    \item [15]  \small{BLINK/FLASH ON}
    \item [16]  F9
    \item [17]  \megakey{$\downarrow$}
    \item [18]  \specialkey{RVS ON}
    \item [19]  \specialkey{CLR\\HOME}
    \item [20]  \specialkey{INST\\DEL}
    \item [21]  F10 / BACK WORD
    \item [22]  F11
    \item [23]  F12 / NEXT WORD
    \item [24]  SET/CLEAR TAB
    \item [25]  F13
    \item [26]  F14 / BACK TAB
    \item [27]  \small{ESCAPE}
    \item [28]  \small{RED}
    \item [29]  \megakey{$\rightarrow$}
    \item [30]  \small{GREEN}
    \item [31]  \small{BLUE}
    \item [32]  \megakey{SPACE}
    \item [33]  !
    \item [34]  "
    \item [35]  \#
    \item [36]  \$
    \item [37]  \%
    \item [38]  \&
    \item [39]  '
    \item [40]  (
    \item [41]  )
    \item [42]  *
    \item [43]  +
    \item [44]  ,
    \item [45]  -
    \item [46]  .
    \item [47]  /
    \item [48]  0
    \item [49]  1
    \item [50]  2
    \item [51]  3
    \item [52]  4
    \item [53]  5
    \item [54]  6
    \item [55]  7
    \item [56]  8
    \item [57]  9
    \item [58]  :
    \item [59]  ;
    \item [60]  <
    \item [61]  =
    \item [62]  >
    \item [63]  ?
    \item [64]  @
    \item [65]  A
    \item [66]  B
    \item [67]  C
    \item [68]  D
    \item [69]  E
    \item [70]  F
    \item [71]  G
    \item [72]  H
    \item [73]  I
    \item [74]  J
    \item [75]  K
    \item [76]  L
    \item [77]  M
    \item [78]  N
    \item [79]  O
    \item [80]  P
    \item [81]  Q
    \item [82]  R
    \item [83]  S
    \item [84]  T
    \item [85]  U
    \item [86]  V
    \item [87]  W
    \item [88]  X
    \item [89]  Y
    \item [90]  Z
    \item [91]  [
    \item [92]  \pounds
    \item [93]  ]
    \item [94]  \megakeywhite{$\uparrow$}
    \item [95]  \megakeywhite{$\leftarrow$}
    \item [128]
    \item [129] \small{ORANGE}
    \item [130] \small{UNDERLINE OFF}
    \item [131] \specialkey{SHIFT}\specialkey{RUN\\STOP}
    \item [132] HELP
    \item [133] F1
    \item [134] F3
    \item [135] F5
    \item [136] F7
    \item [137] F2
    \item [138] F4
    \item [139] F6
    \item [140] F8
    \item [141] \specialkey{SHIFT}\specialkey{RETURN}
    \item [142] \small{UPPERCASE}
    \item [143] \small{BLINK/FLASH OFF}
    \item [144] \small{BLACK}
    \item [145] \megakey{$\uparrow$}
    \item [146] \specialkey{RVS OFF}
    \item [147] \specialkey{SHIFT}\specialkey{CLR\\HOME}
    \item [148] \specialkey{SHIFT}\specialkey{INST\\DEL}
    \item [149] \small{BROWN}
    \item [150] \small{LT. RED (PINK)}
    \item [151] \small{DK. GREY}
    \item [152] \small{GREY}
    \item [153] \small{LT. GREEN}
    \item [154] \small{LT. BLUE}
    \item [155] \small{LT. GREY}
    \item [156] \small{PURPLE}
    \item [157] \megakey{$\leftarrow$}
    \item [158] \small{YELLOW}
    \item [159] \small{CYAN}
    \item [160] \megakey{SPACE}
    \item [161] \graphicsymbol{k}
    \item [162] \graphicsymbol{i}
    \item [163] \graphicsymbol{t}
    \item [164] \graphicsymbol{[}
    \item [165] \graphicsymbol{g}
    \item [166] \graphicsymbol{=}
    \item [167] \graphicsymbol{m}
    \item [168] \graphicsymbol{/}
    \item [169] \graphicsymbol{?}
    \item [170] \graphicsymbol{v}
    \item [171] \graphicsymbol{q}
    \item [172] \graphicsymbol{d}
    \item [173] \graphicsymbol{z}
    \item [174] \graphicsymbol{s}
    \item [175] \graphicsymbol{n}
    \item [176] \graphicsymbol{a}
    \item [177] \graphicsymbol{e}
    \item [178] \graphicsymbol{r}
    \item [179] \graphicsymbol{w}
    \item [180] \graphicsymbol{h}
    \item [181] \graphicsymbol{j}
    \item [182] \graphicsymbol{l}
    \item [183] \graphicsymbol{y}
    \item [184] \graphicsymbol{u}
    \item [185] \graphicsymbol{p}
    \item [186] \graphicsymbol{\{}
    \item [187] \graphicsymbol{f}
    \item [188] \graphicsymbol{c}
    \item [189] \graphicsymbol{x}
    \item [190] \graphicsymbol{v}
    \item [191] \graphicsymbol{b}
    \item [192] \graphicsymbol{C}	% actually: @ + 128
    \item [193] \graphicsymbol{A}
    \item [194] \graphicsymbol{B}
    \item [195] \graphicsymbol{C}
    \item [196] \graphicsymbol{D}
    \item [197] \graphicsymbol{E}
    \item [198] \graphicsymbol{F}
    \item [199] \graphicsymbol{G}
    \item [200] \graphicsymbol{H}
    \item [201] \graphicsymbol{I}
    \item [202] \graphicsymbol{J}
    \item [203] \graphicsymbol{K}
    \item [204] \graphicsymbol{L}
    \item [205] \graphicsymbol{M}
    \item [206] \graphicsymbol{N}
    \item [207] \graphicsymbol{O}
    \item [208] \graphicsymbol{P}
    \item [209] \graphicsymbol{Q}
    \item [210] \graphicsymbol{R}
    \item [211] \graphicsymbol{S}
    \item [212] \graphicsymbol{T}
    \item [213] \graphicsymbol{U}
    \item [214] \graphicsymbol{V}
    \item [215] \graphicsymbol{W}
    \item [216] \graphicsymbol{X}
    \item [217] \graphicsymbol{Y}
    \item [218] \graphicsymbol{Z}
    \item [219] \graphicsymbol{+}	% [ + 128
    \item [220] \graphicsymbol{-}	% \pounds + 128
    \item [221] \graphicsymbol{B}	% ] + 128
    \item [222] \graphicsymbol{\textbackslash}	% pi is a special case 
    \item [223] \graphicsymbol{]}	% leftarrow + 128
    \item [255] \graphicsymbol{\textbackslash}	% pi is a special case 
\end{description}
\end{multicols}
\end{adjustwidth}

\underline{Note 1}: Codes from 96 to 127 are undefined but on output appear the same as 192 to 223. Likewise, codes from 224 to 254 are undefined but appear equal to 160 to 190.
Pi (\graphicsymbol{\textbackslash}) is a very special case: the keyboard produces code 222 but \screentext{PRINT ASC("\graphicsymbol{\textbackslash}")} prints 255 (the screencode to PETSCII conversion makes this exception).
The undefined codes never appear on input from the keyboard.

\underline{Note 2}: While using lowercase/uppercase mode (by pressing \megasymbolkey + \specialkey{SHIFT}), be aware that:
\begin{itemize}
  \item The uppercase letters in region 65-90 of the above table are replaced with lowercase letters.
  \item The graphical characters in region 193-218 of the above table are replaced with uppercase letters.
  \item PETSCII is based on the older uppercase-only ASCII-1963 (which has the backarrow and uparrow characters). Because of that, the codes in the region where ASCII-1967 has lowercase letters, PETSCII is undefined.
\end{itemize}
