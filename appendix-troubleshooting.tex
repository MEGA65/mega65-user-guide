\chapter{Trouble shooting}
    \section{vivado}
    \subsection{RAM requirements}
    \begin{tcolourbox}{colback=black,coltext=white}
    \verbatimfont{\codefont}
    \begin{verbatim}
INFO: [Synth 8-256] done synthesizing module 'ram32x1024' [/home/....]
INFO: [Synth 8-256] synthesizing module 'charrom' [/home/....]
    /opt/Xilinx/Vivado/2019.2/bin/loader: line 280: 2317 killed
    WARNING: [Vivado 12-8222] Failed run(s) : 'synth_1'
        ERROR: Application Exception: failed to launch run 'impl_1' due to failures in the following run(s):
        synth_1
        These failed run(s) need to be reset prior to launching 'impl_1' again.
    \end{verbatim}
    \end{tcolourbox}
This error is due to vivado crashing because the machine doesn't have enough RAM for vivado to run.
Vivado requires at least 4GB to synthesise the MEGA65 target, but 8GB is better.

