\chapter{Configuring your MEGA65}
\label{cha:configuring}

\section{Preparing for First Use: Formatting SD Cards}

The MEGA65 has two SD card slots: A full-size SD card slot inside, next to
the trap-door, and a microSD size slot on the rear.  The current version
of the MEGA65 firmware only supports the use of one SD card at a time.
If you have cards in both slots, the external MEGA65 will default to the external slot. The exception to this, is that the MEGA65's FDISK/FORMAT
utility can access both, allowing you to select which SD card to format or
repair.

Depending on the model, your MEGA65 may or may not come with a pre-configured SD card. If it doesn't, or if you wish to use a different SD card, e.g., with a
larger capacity, you must first format it for use in the MEGA65.

{\em This must be done in the MEGA65, not in a PC or other computer.}

{\em Only use SDHC cards. Older SD cards (typically with
  a capacity of <4GB) will not work. Newer SDXC cards with
  capacities greater than 32GB may or may not work. We would
  appreciate hearing your experience with such cards. It is unimportant
  what file-system is currently on the card, as the MEGA65
  FDISK/FORMAT utility will completely reformat the card.}

There are several reasons for this: First, in order to fit the most
features into the MEGA65's small operating system, it is
particular about the FAT32 file system it uses. Second, only the
MEGA65 FDISK/FORMAT utility can create a MEGA65 System Partition. The
MEGA65 System Partition holds non-volatile configuration settings for
your MEGA65, and also contains the freeze slots that make it easy to
switch between MEGA65 programmes or games.

Fortunately, formatting an SD card on the MEGA65 is easy.

First, power the MEGA65 on while holding down the \megakey{ALT} key.
This will present the MEGA65 Utility Menu, which contains a
selection of built-in utilities, similar to the following:

\includegraphics[width=\linewidth]{images/ss-utilmenu.png}

The exact set of utilities depends on the model of your MEGA65 and the version of the MEGA65 factory core which it is running. However, all versions include both the MEGA65 FDISK/FORMAT utility, and the MEGA65 Configure utility. Most models include a keyboard test utility. Use this utility to test keyboard functionality. This is the same utility used in the factory when testing brand
new keyboards.

Select the number that corresponds to the FDISK/FORMAT utility; typically 2. The FDISK utility will start, and detect the size of all SD cards installed.  If this is both an internal and external SD card, the utility will allow you to
choose which one to format. The internal SD card is bus 1 and the external card is bus 0. Note that the MEGA65 will always attempt to boot using an external microSD card, if one is installed.

For safety when formatting, we {\em strongly} recommend that you remove any SD
or microSD card that you do not intend to format. This will ensure you do
not accidentally destroy data. Formatting an SD card in the MEGA65 cannot be undone. Once a card is formatted, all data {\em is lost}. If you have files or data on the SD card that you wish to retain, make a backup using a PC or other computer. A backup of the MEGA65 system partition, including the contents of freeze slots, requires the use of specialised software.

% What is this specialized software?

You should backup data from your MEGA65 on a regular basis, especially while the computer remains under development.  While we take every care to avoid data corruption, we cannot guarantee that the MEGA65 is free of bugs.

If you have only an internal SD card, you might see a display similar to the following:

\includegraphics[width=\linewidth]{images/ss-m65fdisk-busselect.png}

Once you have selected the bus, the FDISK/FORMAT utility asks you to confirm that you wish to delete everything:

\includegraphics[width=\linewidth]{images/ss-m65fdisk-typesomething.png}

To avoid accidentally loss of data, you must type ``DELETE EVERYTHING'' in capitals and press \megakey{RETURN}.  Alternatively, turn the MEGA65 off and on to abort this process without causing damage to your data.

It is possible to attempt a recovery from a lost Master Boot Record error (``Boot Sector'') by instead typing ``FIX MBR,''.

\section{Installing ROM and Other Support Files}

The MEGA65 FDISK/FORMAT utility will install a version of the open-source OpenROM project's C64-compatible ROMs as part of the format process. However, you may have other ROMs that you wish to use on the MEGA65. The 911001 version of the C65 ROM in particular is known to work well with the MEGA65. You can copy as many of these as you wish onto the SD card.  Make sure that they have the .ROM extension. The default ROM should be named MEGA65.ROM. These files
are 128KB in size, and use the same internal format as ROMs intended for the C65. This means that the C64-mode KERNAL will be placed at offset \$E000, a C65-mode BASIC at \$A000, and a suitable character set at \$D000.

Other important files include FREEZER.M65 and AUDIOMIX.M65, which
allow you to use the MEGA65's integrated freezer. You can download
the full set of support files for the MEGA65 from:

\url{https://github.com/mega65/mega65-files}

\section{Configuring your MEGA65}

The configuration utility for the MEGA65 fills a similar purpose to the BIOS on a PC, and allows you to control certain default behaviours of your MEGA65; however, rather than storing the configuration data in a
battery-backed RAM, the MEGA65 stores this data on sector 1 of the SD card. If you switch SD cards, you will change the configuration data.

To enter the configuration utility, turn the MEGA65 on while holding the \megakey{ALT} key. A utility menu, similar to the one below, will be displayed:

\includegraphics[width=\linewidth]{images/ss-utilmenu.png}

Press the number corresponding to the Utility Menu. The MEGA65 Configuration Utility will launch, and a menu similar to the one below will be displayed.

\includegraphics[width=\linewidth]{images/ss-m65config-1.png}

If your MEGA65's System Partition has become corrupt, a prompt similar to the one below may be displayed:

\includegraphics[width=\linewidth]{images/ss-m65config-corrupt.png}

To correct this error, press \megakey{F13}. Then press \megakey{F7} to save the reset configuration, or the reset data will not be saved to the MEGA65 System
Partition.

Once you have dismissed that display, or if your MEGA65 System Partition was not corrupted, you can begin exploring and adjusting various settings. The programme can be controlled using the keyboard, or optionally, an Amiga(tm) or C1351 mouse.

You can advance screens by pressing \megakey{F1}, or use \megakey{F2} to navigate in the opposite direction. Use the \megakey{$\leftarrow$} and \megakey{$\rightarrow$} keys to navigate between screens.

Use the \megakey{$\uparrow$} and \megakey{$\downarrow$} keys to select an item.

Press \megakey{RETURN} or \megakey{SPACE} to toggle a setting, or to change a text or numeric value. The black circle next to an option indicates the current selection.

When finished, press \megakey{F7} to provide the following options:

\includegraphics[width=\linewidth]{images/ss-m65config-save.png}

\begin{itemize}
  \item{\em Exit Without Saving} Abandon any changes made in the MEGA65 Configure utility and exit the utility.
  \item{\em Apply and Test Settings Now} Use the current settings immediately but do not exit. This is helpful to test compatibility of your TV or monitor with PAL or NTSC video modes. If you still see your display after applying a change, it is safe to save those settings.
  \item{\em Restore Factory Defaults} Resets the MEGA65 configuration settings to the factory defaults. It wukk randomly select a new MAC address for models that include an internal Ethernet adaptor. If you wish to commit these changes, you must still save them.
  \item{\em Save as Default and Exit} Commits changes made to the SD card. These changes will be used when the MEGA65 is turned on.
\end{itemize}

\subsection{Input Devices}

\includegraphics[width=\linewidth]{images/ss-m65config-1.png}

\begin{itemize}
  \item{\em Joystick 1 Amiga Mouse Mode} allows either {\bf normal} operation,
    where software will see it as an Amiga mouse, or {\bf 1351
      emulation} mode, where the MEGA65 translates the Amiga mouse's
    movements into 1351 compatible signals. This allows you to use an
    Amiga mouse with existing C64/C65 software that expect a 1351
    mouse.
  \item{\em Joystick 1 Amiga Mouse Detection} can be set to conservative
    or aggressive.  If you use an Amiga mouse, and it fails to move
    smoothly in all directions, you may set it to {\bf
      aggressive}. Conversely, if you regularly use joysticks in the
    port, and have difficulties with the joystick input
    mis-behaving, you may select the {\bf conservative}
    option.
  \item{\em Joystick 2 Amiga Mouse Mode} is the same as the first
    option, but for the second joystick port. This allows you to
    have different policies for each port.
  \item{\em Joystick 2 Amiga Mouse Detection} similarly provides the
    ability to separately control the Amiga mouse detection
    algorithm for the second joystick port.
\end{itemize}


\subsection{Chipset}

\includegraphics[width=\linewidth]{images/ss-m65config-2.png}

\begin{itemize}
  \item{em Real-Time Clock} allows setting the MEGA65's Real-Time
    Clock for those models that include one.  To set the clock or
    calendar, simply edit the field and press the \specialkey{RETURN}
    key.  The display does not change while viewing this page, but if
    you use the cursor left and right keys to select another page and
    return to this page, the values will update if a Real-Time Clock
    is fitted and functioning.
  \item{\em DMAgic Revision} allows selecting the default mode of
    operation for the C65 DMAgic DMA controller.  This option is only
    required for ROMs not detected by the MEGA65's HYPPO Hypervisor.
    If you see screen corruption in BASIC,
    try toggling this option.
  \item{\em F011 Disk Controller}
    This option allows you to select whether the internal 3.5'' floppy
    drive functions using real diskettes, or whether it simply makes
    noises to add atmosphere when using D81 disk images from the SD
    card.  This merely sets the default option, and you can change
    this setting, or select a different disk image for use as either
    or both of the C65 3.5'' DOS based drives.
  \item{\em Default Disk Image} allows you to choose the D81 disk image
    used with the internal drive, if the F011 Disk
    Controller option above is set to use an SD card disk image.
\end{itemize}

\subsection{Video}

\includegraphics[width=\linewidth]{images/ss-m65config-3.png}

\begin{itemize}
  \item{\em Video Mode} selects whether the MEGA65 starts in PAL or NTSC.
    The MEGA65 supports true 480p NTSC and 576p PAL double-scan modes,
    with exact 60Hz / 50Hz frame-rates.  This setting sets the
    default value, and the system can be switched between PAL and NTSC
    via the Freeze Menu, or under software control by MEGA65-enabled
    programmes.
\end{itemize}

\subsection{Audio}

\includegraphics[width=\linewidth]{images/ss-m65config-4.png}

\begin{itemize}
  \item{\em Audio Output} selects whether the SIDs and digital audio
    channels are combined to provide a mono-aural signal, or whether
    the left and right tagged audio sources are separated to provide a
    stereo signal. Again, this setting can be varied from in the Audio
    Mixer of the Freeze Menu, or under the control of MEGA65-enabled
    software.
  \item{\em Swap Stereo Channels} allows switching the left and right
    sides of the stereo audio output. This is primarily useful for
    software that expects left and right SIDs to be at swapped
    addresses compared with the MEGA65.
  \item{\em DAC Algorithm} allows selecting between two different
    digital to analog conversion algorithms.  Both are very good,
    but you may have a preference for one or the other.
  \item{\em Audio Amplifier} allows enabling or disabling the audio
    amplifier contained in some models of the MEGA65 for
    certain audio outputs, e.g., internal speaker or loud speaker.
\end{itemize}

\subsection{Network}

\includegraphics[width=\linewidth]{images/ss-m65config-5.png}

\begin{itemize}
  \item{\em MAC Address} allows you to set the default MAC address of your
    MEGA65.  This can be changed at run-time by MEGA65-enabled
    software
\end{itemize}
