\chapter{Configuring your MEGA65}
\label{cha:configuring}

\section{Preparing for First Use: Formatting SD Cards}

The MEGA65 has two SD card slots: A full-size SD card slot inside, next to
the trap-door, and a microSD size slot on the rear.  The current version
of the MEGA65 firmware only supports the use of one SD card at a time.
If you have cards in both slots, the external MEGA65 will default to the external slot. The exception to this, is that the MEGA65's FDISK/FORMAT
utility can access both, allowing you to select which SD card to format or
repair.

Depending on the model, your MEGA65 may or may not come with a pre-configured SD card. If it doesn't, or if you wish to use a different SD card, e.g., with a
larger capacity, you must first format it for use in the MEGA65.

{\em This must be done in the MEGA65, not in a PC or other computer.}

{\em Only use SDHC cards. Older SD cards (typically with
  a capacity of <4GB) will not work. Newer SDXC cards with
  capacities greater than 32GB may or may not work. We would
  appreciate hearing your experience with such cards. It is unimportant
  what file-system is currently on the card, as the MEGA65
  FDISK/FORMAT utility will completely reformat the card.}

There are several reasons for this: First, to fit the most
features into the MEGA65's small operating system, it is
particular about the FAT32 file system it uses. Second, only the
MEGA65 FDISK/FORMAT utility can create a MEGA65 System Partition. The
MEGA65 System Partition holds non-volatile configuration settings for
your MEGA65 and also contains the freeze slots that make it easy to
switch between MEGA65 programmes or games.

Formatting an SD card on the MEGA65 is easy.

Power the MEGA65 on while holding down the \megakey{ALT} key.
This will present the MEGA65 Utility Menu, which contains a
selection of built-in utilities, similar to the following:

\includegraphics[width=\linewidth]{images/ss-utilmenu.png}

{\em Note that Utility Menu is always accessible, even if no SD card is present in both internal and external slots.}

The exact set of utilities
depends on the model of your MEGA65 and the version of the MEGA65
factory core which it is running. However, all versions include both
the MEGA65 FDISK/FORMAT utility, and the MEGA65 Configure utility.
Most models also include a keyboard test utility, that you can use
to test that your keyboard is functioning correctly.  This is
the same utility used in the factory when testing brand
new keyboards.

Select the number that corresponds to the FDISK/FORMAT utility.  This
will typically be 2.  The FDISK utility will start, and attempt to
detect the size of all SD cards you have installed.  If you have both
an internal and external SD card installed, it will allow you to
choose which one you wish to format. The internal SD card is bus 1,
and the external card is bus 0.  Note that the MEGA65 will
always attempt to boot using an external microSD card, if one is
installed.

For safety when formatting we {\em strongly} recommend
that you remove any SD card or microSD card that you do not intend to
format, so that you do not accidentally destroy any data.  This is
because formatting an SD card in the MEGA65 cannot be undone, and
all data currently on the SD card {\em will be lost}.  If you
have files or data on the SD card that you wish to retain, you
should first back this up.  The contents of the FAT32
partition can be easily backed up by inserting the SD card into
another computer.  The contents of the MEGA65 System Partition,
including the contents of freeze slots requires the use of specialised
software.

You should aim to backup valuable data from your
MEGA65 on a regular basis, especially while the computer remains under
development.  While we take every care to avoid data corruption or
other mishaps, we cannot guarantee that the MEGA65 is free of bugs in
this regard.

If you have only an internal SD card, you might see a
display similar to the following:

\includegraphics[width=\linewidth]{images/ss-m65fdisk-busselect.png}

Once you have selected the bus, the FDISK/FORMAT utility asks you to confirm that you wish to delete everything:

\includegraphics[width=\linewidth]{images/ss-m65fdisk-typesomething.png}

To avoid accidental loss of data, you must type ``DELETE EVERYTHING'' in capitals and press \megakey{RETURN}. Alternatively, turn the MEGA65 off and on to abort this process without causing damage to your data.

It is possible to attempt a recovery from a lost Master Boot Record error (``Boot Sector'') by instead typing ``FIX MBR,''.


As an end result, we want to have a correctly formatted SD card with the essential files stored on it for MEGA65 to boot from.

This is how it works: When powering on, MEGA65 will search for- and boot these files:
\begin{itemize}
\item {\tt FREEZER.M65}
\item {\tt AUDIOMIX.M65}
\item {\tt C64THUMB.M65}
\item {\tt C65THUMB.M65}
\item {\tt MEGA65.ROM}
\item {\tt MEGA65.D81 (default disk image, automatically mounted)}
\end{itemize}

Straight out-of-the-box, MEGA65 will only have one SD card installed, located inside behind the trap-door. This SD card contains the essential files to properly boot from.
When an external microSD card is inserted, MEGA65 consider the external one as higher priority and will try to boot from it.
That means that the microSD card needs to have the essential files on it, otherwise MEGA65 cannot boot properly and will fall back into loading the OpenROM which does not support all the MEGA65 features.
In general, if MEGA65 cannot boot properly and fall back to OpenROM, it will be shown in the power-up screen, similar to this:

\includegraphics[width=\linewidth]{images/mega65_OpenROM_boot_noSD.png}


\section{Installing ROM and Other Support Files}
\label{sec:installingrometc}

The MEGA65 FDISK/FORMAT utility will install a version of the
open-source OpenROM project's C64-compatible ROMs as part of the
formatting process. However, you may have other ROMs that you wish to
use on the MEGA65. The 911001 version of the C65 ROM in
particular is known to work well with the MEGA65.
You can copy as many of these as you wish onto the
SD card.  Make sure that they have the .ROM extension.  The default ROM
should be called MEGA65.ROM.  These files
should be 128KB in size, and use the same internal format as ROMs
intended for the C65.  This means that the C64-mode KERNAL should be
placed at offset \$E000, a C65-mode BASIC at \$A000, and a suitable
character set at \$D000.

Other important support files include FREEZER.M65 and AUDIOMIX.M65, which
allow you to use the MEGA65's integrated freezer. More details provided below.

\subsection{ROM File}

\textbf{Original C65 ROMs}

You may want to source your own separate C65 ROM via other means.  There were many different versions created during the development of the Commodore 65,
and the MEGA65 can use any of them.  However, out of this set, we recommend you use 911001.bin, as this has the most complete BASIC and DOS implementations.

\textbf{MEGA65 ROMs}

There are newer versions of the \textbf{MEGA65 Closed ROM} actively under development also. These ROMs improve upon the original C65 ROMs and make better use of the extra hardware capabilities that the MEGA65 has over the original C65 hardware. These ROMs are available via the filehost also, but only to owners of the MEGA65, who will need to log into the filehost with their credentials in order to download it. It can be located by visiting the "\textbf{Files}" tab and searching for "\textbf{kernel rom}":

\includegraphics[width=\linewidth]{images/latest_closed_rom.png}

\textbf{MEGA65 ROM diff files}

If you have sourced your own 911001.bin C65 ROM and would like to patch it to the latest MEGA65 ROM, we do provide patches, as the additional improvements we have made to the closed rom are open source. Those diff files are available here:

\url{https://files.mega65.org?id=fd2c40b9-f337-41f7-8a81-0254b1e09fb5}

\textbf{MEGA65 OpenROMs}

Another available option is to make use of \textbf{MEGA65 OpenROMs}. The latest version of this is always downloadable from either of the following urls:

\begin{itemize}
  \item \url{https://files.mega65.org?id=8aec2fba-3b0a-4677-80ae-7a7f5f4f0cb8}
  \item \url{https://github.com/MEGA65/open-roms/raw/master/bin/mega65.rom}
\end{itemize}


\subsection{Support Files}

For official owners of the MEGA65 (both devkit and final product), visit the following url and log in with the user credentials you have been provided. This will take you to the MEGA65 Filehost location where the "\textbf{MEGA65 SD card essentials}" download page is located. Then click the "\textbf{Download}" link to retrieve the latest "\textbf{SD essentials.rar}" file.

\url{https://files.mega65.org?id=a809e0ae-30ac-42f5-ab9c-766d72fd6331}

\includegraphics[width=\linewidth]{images/latest_support_files_with_closedrom.png}

Note that this link is only available to official owners of the MEGA65 product, as the fileset also contains the licensed closed-source MEGA65.ROM file.

For Nexys board owners in search of a similar fileset (without the ROM), visit the following url instead:

\url{https://files.mega65.org/?id=0fb941fe-5c5f-4608-b0f1-32849d4a8dff}

This will take you to the MEGA65 Filehost location where the "\textbf{MEGA65 SD card essentials - No ROM}" download page is located. Click the "\textbf{Download}" link to retrieve the latest "\textbf{SD essentialsNoROM.rar}" file.

Note that while this fileset does not contain a ROM, there are future plans for it to include the freely available OpenROM.

\includegraphics[width=\linewidth]{images/latest_support_files.png}

\section{On-boarding}

On first launch of your MEGA65, you will see the on-boarding screen.

\begin{center}
  \includegraphics[width=\linewidth]{images/img011_final_boot_01.jpg}
\end{center}

Here you can select and test you screen configuration.

For example, type \megakey{TAB} to switch to PAL 50HZ

\begin{center}
  \includegraphics[width=\linewidth]{images/img011_final_boot_02.png}
\end{center}

Then press \megakey{RETURN} , followed by \megakey{Y} to test the new video mode.

\begin{center}
  \includegraphics[width=\linewidth]{images/img011_final_boot_03.jpg}
\end{center}

Press \megakey{K} to keep the new video mode.

\begin{center}
  \includegraphics[width=\linewidth]{images/img011_final_boot_04.jpg}
\end{center}

Press \megakey{RETURN} to complete the configuration.

\begin{center}
  \includegraphics[width=\linewidth]{images/img011_final_boot_05.png}
\end{center}

\textcolor{red}{\underline{Note for Nexys4 board users}: \\
\\
  At this very specific step, the board is supposed to reboot and display the main MEGA65 screen. If the board does not reboot and the screen remains black, then switch power to the board off then on again.}

After the reboot you will get the main MEGA65 screen:

\begin{center}
  \includegraphics[width=\linewidth]{images/img011_final_boot_06.jpg}
\end{center}

\section{Configuration Utility}

The configuration utility for the MEGA65 fills a similar purpose to the BIOS on a PC, and allows you to control certain default behaviours of your MEGA65; however, rather than storing the configuration data in a
battery-backed RAM, the MEGA65 stores this data on sector 1 of the SD card. If you switch SD cards, you will change the configuration data.


\begin{minipage}{\linewidth}
  To enter the configuration utility, turn the MEGA65 on while
  holding the \megakey{ALT} key.  This will show the utility menu,
  similar to the following: \\
  \\
  \includegraphics[width=\linewidth]{images/ss-utilmenu.png}
\end{minipage}

\begin{minipage}{\linewidth}
  Now press the number corresponding to the Utility Menu.  The MEGA65
  Configuration Utility will launch, showing a display similar to
  the following: \\
  \\
  \includegraphics[width=\linewidth]{images/ss-m65config-1.png}
\end{minipage}

\begin{minipage}{\linewidth}
  If your MEGA65's System Partition has become corrupt, you may be
  prompted to press \megakey{F14} to correct this, i.e., hold \megakey{SHIFT} and tap
  the \megakey{F13} key, with a display like the following: \\
  \\
  \includegraphics[width=\linewidth]{images/ss-m65config-corrupt.png}
\end{minipage}

To correct this error, press \megakey{F13}. Then press \megakey{F7} to save the reset configuration, or the reset data will not be saved to the MEGA65 System
Partition.

Once you have dismissed that display, or if your MEGA65 System Partition was not corrupted, you can begin exploring and adjusting various settings. The program can be controlled using the keyboard, or optionally, an Amiga(tm) or C1351 mouse.

You can advance screens by pressing \megakey{F1}, or use \megakey{F2} to navigate in the opposite direction. Use the \megakey{$\leftarrow$} and \megakey{$\rightarrow$} keys to navigate between screens.

Use the \megakey{$\uparrow$} and \megakey{$\downarrow$} keys to select an item.

Press \megakey{RETURN} or \megakey{SPACE} to toggle a setting, or to change a text or numeric value. The black circle next to an option indicates the current selection.

\begin{minipage}{\linewidth}
  When finished, you can press \megakey{F7} to receive the
  option to save the changes. This will give you four options: \\
  \\
  \includegraphics[width=\linewidth]{images/ss-m65config-save.png}
\end{minipage}

\begin{itemize}
  \item{\em Exit Without Saving} abandons any changes made in the MEGA65 Configure utility and exit the utility.
  \item{\em Apply and Test Settings Now} uses the current settings immediately but do not exit. This is helpful to test compatibility of your TV or monitor with PAL or NTSC video modes. If you still see your display after applying a change, it is safe to save those settings.
  \item{\em Restore Factory Defaults} resets the MEGA65 configuration settings to the factory defaults. It wukk randomly select a new MAC address for models that include an internal Ethernet adaptor. If you wish to commit these changes, you must still save them.
  \item{\em Save as Default and Exit} commits changes made to the SD card. These changes will be used when the MEGA65 is turned on.
\end{itemize}

\subsection{Input Devices}

\includegraphics[width=\linewidth]{images/ss-m65config-1.png}

\begin{itemize}
  \item{\em Joystick 1 Amiga Mouse Mode} allows either {\bf normal} operation,
  where software will see it as an Amiga mouse or {\bf 1351 emulation} mode, where the MEGA65 translates the Amiga mouse's movements into 1351 compatible  signals. This allows you to use an Amiga mouse with existing C64/C65 software compatible with a 1351 mouse.
  \item{\em Joystick 1 Amiga Mouse Detection} can be set to conservative or aggressive. If you use an Amiga mouse, and it fails to move smoothly in all directions, you may set it to {\bf aggressive}. Conversely, if you regularly use joysticks in the port, and have difficulties with the joystick input mis-behaving, you may select the {\bf conservative} option.
  \item{\em Joystick 2 Amiga Mouse Mode} is identical to the first option, but for the second joystick port. This allows you to have different policies for each port.
  \item{\em Joystick 2 Amiga Mouse Detection} similarly provides the ability to separately control the Amiga mouse detection algorithm for the second joystick port.
\end{itemize}


\subsection{Chipset}

\includegraphics[width=\linewidth]{images/ss-m65config-2.png}

\begin{itemize}
  \item{\em Real-Time Clock} allows setting the MEGA65's Real-Time
    Clock for those models that include one.  To set the clock or
    calendar, simply edit the field and press the \specialkey{RETURN}
    key.  The display does not change while viewing this page, but if
    you use the cursor left and right keys to select another page and
    return to this page, the values will update if a Real-Time Clock
    is fitted and functioning.
  \item{\em DMAgic Revision} allows selecting the default mode of
    operation for the C65 DMAgic DMA controller.  This option is only
    required for ROMs not detected by the MEGA65's HYPPO Hypervisor.
    If you see screen corruption in BASIC,
    try toggling this option.
  \item{\em F011 Disk Controller}
    This option allows you to select whether the internal 3.5'' floppy
    drive functions using real diskettes, or whether it simply makes
    noises to add atmosphere when using D81 disk images from the SD
    card.  This merely sets the default option, and you can change
    this setting, or select a different disk image for use as either
    or both of the C65 3.5'' DOS based drives.
  \item{\em Default Disk Image} allows you to choose the D81 disk image
    used with the internal drive, if the F011 Disk
    Controller option above is set to use an SD card disk image.
\end{itemize}

\subsection{Video}

\includegraphics[width=\linewidth]{images/ss-m65config-3.png}

\begin{itemize}
  \item{\em Video Mode} selects whether the MEGA65 starts in PAL or NTSC.    The MEGA65 supports true 480p NTSC and 576p PAL double-scan modes, with exact 60Hz / 50Hz frame-rates. This setting sets the default value, and the system can be switched between PAL and NTSC via the Freeze Menu or under software control by MEGA65-enabled programmes.
\end{itemize}

\subsection{Audio}

\includegraphics[width=\linewidth]{images/ss-m65config-4.png}

\begin{itemize}
  \item{\em Audio Output} selects whether the SIDs and digital audio channels are combined to provide a mono-aural signal or whether the left and right tagged audio sources are separated to provide a stereo signal. This setting can be varied from in the Audio Mixer of the Freeze Menu or under the control of MEGA65-enabled software.
  \item{\em Swap Stereo Channels} allows switching the left and right sides of the stereo audio output. This is useful for software that expects left and right SIDs to be at swapped addresses compared with the MEGA65 defaults.
  \item{\em DAC Algorithm} allows selecting between two different digital to analog conversion algorithms. Both options sound good and the selection is a personal preference.
  \item{\em Audio Amplifier} allows enabling or disabling the audio amplifier contained in some models of the MEGA65. This option works for audio outputs, e.g., internal speaker or loud speaker.
\end{itemize}

\subsection{Network}

\includegraphics[width=\linewidth]{images/ss-m65config-5.png}

\begin{itemize}
  \item{\em MAC Address} allows you to set the default MAC address of your MEGA65. This can be changed at run-time by MEGA65-enabled software.
\end{itemize}

% 2021-03-17 edits by SBC
