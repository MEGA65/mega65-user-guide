\chapter{Cores and Flashing}

\phantomsection
\section{What are cores?}

The MEGA65 computer uses a versatile chip called an FPGA as its heart.
FPGAs are ``Field Programmable Gate Arrays''. This is a fancy way of
saying that FPGAs are chips that can be programmed to behave impersonate
other chips.  They do this by configuring their arrays of logic gates to
reproduce the circuits of other chips. In this way, FPGAs are not emulation
but re-creation of other chips.

FPGAs forget what chip they were pretending
to be whenever the power is turned off, or when they are ``reconfigured''.
This might sound annoying, but it's actually really powerful. It means that
we can tell the FPGA in the MEGA65 to impersonate not just the MEGA65 design
as it currently stands, but to impersonate any improvements we make to the design.
In other words, we can upgrade the MEGA65 hardware just by providing a new
set of instructions to the FPGA.  These sets of instructions are called ``cores''
or ``bitstreams''.  For the purpose of the MEGA65, these two terms can usually be
considered to be interchangeable.
