\chapter{Cores and Flashing}

\phantomsection
\section{What are cores, and why does it matter?}

The MEGA65 computer uses a versatile chip called an FPGA as its heart.
FPGAs are ``Field Programmable Gate Arrays''. This is a fancy way of
saying that FPGAs are chips that can be programmed to behave impersonate
other chips.  They do this by configuring their arrays of logic gates to
reproduce the circuits of other chips. In this way, FPGAs are not emulation
but re-creation of other chips.

FPGAs forget what chip they were pretending
to be whenever the power is turned off, or when they are ``reconfigured''.
This might sound annoying, but it's actually really powerful. It means that
we can tell the FPGA in the MEGA65 to impersonate not just the MEGA65 design
as it currently stands, but to impersonate any improvements we make to the design.
In other words, we can upgrade the MEGA65 hardware just by providing a new
set of instructions to the FPGA.  These sets of instructions are called ``cores''
or ``bitstreams''.  For the purpose of the MEGA65, these two terms can usually be
considered to be interchangeable.

FPGAs are so flexible, that not only is it possible to teach the MEGA65 to be a better
MEGA65, but it is also possible to teach the MEGA65's' FPGA to be other interesting
home computers.  We believe that the FPGA is powerful enough that it could pretend to be
a VIC-20 (tm), Commodore PET (tm), Apple II (tm), Spectrum (tm), BBC Micro (tm), or even
an Amiga (tm) or one of the 16-bit era game consoles.  Unlike some previous FPGA-based
retro-computers, the MEGA65, its FPGA instructions, board layout and other information is
all available for free under open-source licenses. This means that anyone is free to
create other cores for the MEGA65 hardware.

To top it all off, the MEGA65 has enough storage for 7 different sets of FPGA instructions,
so that you can easily switch the MEGA65's ``personality'' from being a MEGA65 to another
of these systems (once the cores are available) and back again.

The remainder of this chapter describes how to select which core to run on the MEGA65, and
how to store a core into one of the seven slots in the flash memory storage.

\phantomsection
\section{Selecting a core}

To operate the MEGA65 using an alternative core, turn off the power to the MEGA65, and then hold the
\specialkey{TAB} key down while turning the power back on.  This instructs the MEGA65 to enter the
flash and core menu, instead of booting normally.  You should see a display like the following:

XXX - Screen shot of flash menu with 7 empty slots.

To select a core and start it, use the cursor keys to highlight the desired core, and then press the
\specialkey{RETURN} key.  Alternatively, you can press the number corresponding to the core you would
like to use. The MEGA65 immediately reconfigures the FPGA, and launches the core.  If for some reason
the core is faultly, the MEGA65 may instead restart normally after a few seconds, and depending on the
circumstances, take you automatically back into the menu.

The MEGA65 will keep running the new core until you physically power it off.  Pressing the reset button
will not reset which core is being run.

\phantomsection
\section{Installing an upgrade core for the MEGA65}

To install an upgrade core for the MEGA65, there are few easy steps.

First, copy the core file onto the MEGA65's SD card.  You can do this by removing the SD card and inserting
it into another computer that has internet access, and downloading the core from that computer. Alternatively
you can insert an SD card that already contains the upgrade core. Finally, you can use the MEGA65 TFTP Server
program and the MEGA65's ethernet port to copy the core upgrade file onto the SD card from another computer
on your local network.

Second, once you have the upgrade core on the MEGA65's SD card, enter the flash and core menu as above,
i.e., turn off the power, hold the \specialkey{TAB} key down while turning the power on.  When the flash
and core menu appears, hold the \specialkey{CTRL} key down and press the \specialkey{1} key.  The MEGA65
will present you a list of core files that are on the SD card.  Select the upgrade core file you wish to
install using the cursor keys, and then pressing the \specialkey{RETURN} key.  The MEGA65 will then erase
the flash slot, before writing the upgraded core.  This process is quite slow, and can take around 15
minutes.

It is important to not turn the power off during this process. If you do, the core file will be
only partially installed, and the MEGA65 may not start properly. If this happens, enter the flash and core
menu as described above, and follow the instructions again.  When the process completes, you will see a message
like this, indicating that the process is complete:

XXX

When this happens, simply turn off the power to the MEGA65 and turn it back on for it to start using the
upgraded core.  This is because the MEGA65 will always try to automatically start the core in slot 1 when
it is turned on.

\phantomsection
\section{Installing other core for the MEGA65}

Installing other cores works very similarly to installing upgrade cores. The only difference is that you
press \special{CTRL} and \specialkey{2} to \specialkey{7} from the flash and core menu, so that the core
gets installed in another slot.

Of course, there is nothing stopping you installing a different core
in slot 1, so that the MEGA65 behaves as a different type of computer when you turn it on.  If you do this,
you can always choose to run the MEGA65 core by entering the flash and core menu,  and selecting the MEGA65
core.




