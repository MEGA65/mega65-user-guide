\chapter{Developing System Programmes}

\section{Introduction}

The MEGA65 has a number of system programmes and utilities that are used at various times to perform various functions.
This includes the utilities accessible via the Utility Menu \index{Utility Menu}, the Freeze Menu \index{Freeze Menu} and
its own helper programmes, as well as the Flash Menu \index{Flash Menu}\index{MEGA Flash}.

A number of these system programmes are pre-loaded into the MEGA65 bitstream, while others live on the SD card.
For those that are pre-loaded into the MEGA65 bitstream, this works by having areas of pre-initialised memory, that
contain the appropriate programme.  For example, the utilities accessible via the Utility Menu are all located in
the colour RAM, while the Flash Menu is located at \$50000 -- \$57FFF.

In one sense, the easiest way to test new versions of these utilities is to generate a new bitstream with the updated versions.
However, synthesising a new bitstream is very time consuming, typically taking an hour on a reasonably fast computer.
Therefore this chapter explains the procedure for loading an alternate version of each of these system programmes, as well as
providing some useful information about these programmes, how the operate, and the environment in which they operate compared
with normal C64 or C65 mode programmes.

\section{Flash Menu}

The flash menu is located in pre-initialised RAM at \$50000 -- \$57FFFF.  It is executed during the first boot each time the
MEGA65 is powered on.  It is unusual in that it executes in the hypervisor context. This is so that it has access to the QSPI
flash, which is not available outside of Hypervisor Mode, so that user programmes cannot corrupt the cores stored in the flash.

It is also important to note that the flash menu programme must fit {\em entirely} below \$8000 when loaded {\em and} executing, as the Hypervisor is still mapped at \$8000 -- \$BFFF, and can easily be corrupted by an ill behaved flash menu programme.  In this regard, the flash menu
can be regarded as an extension of the hypervisor that is discarded after the first boot.
This is unlike all other system programmes, that operate in a dedicated memory context, from where the Hypervisor is safe from corruption. It also means that you can't crunch the flash menu to make it fit, as it would overwrite the Hypervisor during decrunching.

Also, as the flash menu is executed very early in the boot process, only the pre-included OpenROM ROM image is available.  Thus you must ensure that your flash menu programme is compatible with that ROM.  

The Hypervisor maintains a flag that indicates whether the flash menu has been executed or not. This flag is updated at the point
where the Hypervisor exits to user mode for the first time, since after that point, the contents of \$50000 -- \$57FFF can no longer
be trusted to contain the flash menu.  This means that if you wish to have the Hypervisor run a new version of the flash menu that
you have loaded, you must prevent the Hypervisor from exiting to user mode first.

The easiest way to achieve this is to hold the ALT key down while powering on the MEGA65.  This will cause the Hypervisor to display the Utility Menu, rather than exiting to user mode.  It is safe at this time to use the {\tt m65} utility to load the replacement flash menu programme using a command similar to the following:

\begin{tcolorbox}[colback=black,coltext=white]
\verbatimfont{\codefont}
\begin{verbatim}
m65 -@ newflashmenu.prg@50000
\end{verbatim}
\end{tcolorbox}

That command would load the file {\tt newflashmenu.prg} at memory location \$50000.

After that, you can simply press the reset button the side of the MEGA65, and it will boot again, and because it never left Hypervisor Mode during the previous boot cycle, it will run your updated flash menu programme. 

It should also be possible to completely automate this process, by first using {\tt m65 -b} to load a new bitstream, thus simulating a cold boot, and then quickly calling {\tt m65} again to simulate depressing the ALT key (or herhaps simply halting the processor), then {\tt m65 -@ ...} and finally {\tt m65 -F} to reset the machine.  Writing a script or utility that correctly implements this automation is left as an exercise for the reader.

\section{Format/FDISK Utility}

\section{Keyboard Test Utility}

\section{MEGA65 Configuration Utility}

\section{Freeze Menu}

\section{Freeze Menu Helper Programmes}

\section{Hypervisor}

\section{OpenROM}
