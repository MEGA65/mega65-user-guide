\chapter{Developing System Programmes}

\section{Introduction}

The MEGA65 has a number of system programmes and utilities that are used at various times to perform various functions.
This includes the utilities accessible via the Utility Menu \index{Utility Menu}, the Freeze Menu \index{Freeze Menu} and
its own helper programmes, as well as the Flash Menu \index{Flash Menu}\index{MEGA Flash}.

A number of these system programmes are pre-loaded into the MEGA65 bitstream, while others live on the SD card.
For those that are pre-loaded into the MEGA65 bitstream, this works by having areas of pre-initialised memory, that
contain the appropriate programme.  For example, the utilities accessible via the Utility Menu are all located in
the colour RAM, while the Flash Menu is located at \$50000 -- \$57FFF.

In one sense, the easiest way to test new versions of these utilities is to generate a new bitstream with the updated versions.
However, synthesising a new bitstream is very time consuming, typically taking an hour on a reasonably fast computer.
Therefore this chapter explains the procedure for loading an alternate version of each of these system programmes, as well as
providing some useful information about these programmes, how the operate, and the environment in which they operate compared
with normal C64 or C65 mode programmes.

\section{Hypervisor}

\section{Flash Menu}

\section{Format/FDISK Utility}

\section{Keyboard Test Utility}

\section{MEGA65 Configuration Utility}

\section{Freeze Menu}

\section{Freeze Menu Helper Programmes}
