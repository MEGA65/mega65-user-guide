\chapter{Getting Started}
\phantomsection
\section{Keyboard}
\label{cha:getting-started}

Now that everything is connected, it's time to get familiar with the MEGA65 keyboard.

You may notice that the keyboard is a little different from the keyboards used on computers today. While most keys will be in familiar positions, there are some specialised keys, and some with special graphic symbols marked on the front.

Here's a brief description of how some of these special keys function.

\subsection{Command Keys}

The Command Keys are: \specialkey{RETURN}, \specialkey{SHIFT}, \specialkey{CTRL}, \megasymbolkey, and \widekey{RESTORE}.

\subsubsection{RETURN}
\index{Keyboard!RETURN}
Pressing \specialkey{RETURN} enters the information you have typed into the MEGA65's memory. The computer will either act on a command, store some information, or display an error message if you made a mistake.

\subsubsection{SHIFT}
\index{Keyboard!Shift Keys}
The two \specialkey{SHIFT} keys are located on the left and the right. They work very much like the Shift key on a regular keyboard, however they also perform some special functions as well.

In upper case mode, holding down \specialkey{SHIFT} and pressing any key with two graphic symbols on the front produces the right-hand symbol on that key. For example, \specialkey{SHIFT} and \megakey{J} prints the \graphicsymbol{J} character.

In lower case mode, pressing \specialkey{SHIFT} and a letter key prints the upper case letter on that key.

Finally, holding \specialkey{SHIFT} down and pressing a Function key accesses the function shown on the front of that key. For example: \specialkey{SHIFT} and \megakey{F1} activates \megakey{F2}.


\subsubsection{SHIFT LOCK}
\index{Keyboard!SHIFT LOCK}
In addition to \specialkey{SHIFT} is \specialkey{SHIFT\\LOCK}. Press this key to lock down the Shift function. Now any key you press while \specialkey{SHIFT\\LOCK} is illuminated prints the character to the screen as if you were holding down \specialkey{SHIFT}. This includes special graphic characters.

\subsubsection{CTRL}
\index{Keyboard!CTRL}
\specialkey{CTRL} is the Control key. Holding down \specialkey{CTRL} and pressing another key allows you to perform Control Functions. For example, holding down \specialkey{CTRL} and one of the number keys allows you to change text colours. The colour that is printed at the top row on the front of the number key will be used.

There are some examples of this on page \pageref{sec:screen-editor}, and all of the Control Functions are listed on page \pageref{appendix:controlcodes}.

If a program is being {\bf LIST}ed to the screen, holding down \specialkey{CTRL} slows down the display of each line. You can read
more about the {\bf LIST} command on page \pageref{basic65-list}.

Holding \specialkey{CTRL} and pressing \megakey{*} enters the Matrix Mode Debugger (refer to the {\bf MEGA65 Book} for more details).

\subsubsection{RUN STOP}
\index{Keyboard!RUN STOP}
Normally, pressing \specialkey{RUN STOP} stops the execution of a program. Holding \specialkey{SHIFT} while pressing \specialkey{RUN STOP} {\bf LOAD}s the first program from disk.

Programs are able to disable \specialkey{RUN STOP}.

You can boot your MEGA65 into the {\bf Machine Code Monitor} by holding down \specialkey{RUN STOP} and pressing reset on the left-hand side.

\subsubsection{RESTORE}
\index{Keyboard!RESTORE}
The computer screen can be restored to a clean state without clearing the memory by holding down \specialkey{RUN STOP} and pressing \widekey{RESTORE}. This combination also resets operating system vectors and re-initialises the screen editor, which makes it a handy combination if the computer has become a little confused.

Programs are able to disable this key combination.

You can also enter the {\bf Freezer} by long pressing \widekey{RESTORE}. From there you can access the Machine Code Monitor. You can
read more about the Freezer on page \pageref{sec:freezer}.

\newpage

\subsubsection{THE CURSOR KEYS}
\index{Keyboard!Cursor Keys}
At the bottom right-hand side of the keyboard are the cursor keys. These four directional keys allow you move the cursor to any position for on-screen editing.

The cursor moves in the direction indicated on the keys: \megakey{$\leftarrow$} \megakey{$\uparrow$} \megakey{$\rightarrow$} \megakey{$\downarrow$}.

However, it is also possible to move the cursor up by using \specialkey{SHIFT} and \megakey{$\downarrow$}. In the same way you can move the cursor left by using \specialkey{SHIFT} and \megakey{$\rightarrow$}.

You don't have to keep pressing a cursor key over and over. If you need to move the cursor a long way, you can keep the key pressed down. When you are finished, simply release the key.

\subsubsection{ARROW KEYS}
\index{Keyboard!Arrow Keys}
These keys are different to the cursor keys! They are \megakeywhite{$\leftarrow$} (next to \megakey{1}), and \megakeywhite{$\uparrow$} (next to \widekey{RESTORE}).
Both arrow keys are used in various BASIC functions and escape sequences.

For example, \megakeywhite{$\leftarrow$} can be used as a shortcut for {\bf SAVE}, and \megakeywhite{$\uparrow$}
is used to raise a number to a power (which is the same as multiplying a number by itself a specified number of times).

You can read more about the available escape sequences on page \pageref{escape-sequences}. These two PETSCII specific keys will always be shown in MEGA65 literature with a white background.


\subsubsection{INSerT/DELete}
\index{Keyboard!INST DEL}
This is the INSERT / DELETE key. When pressing \specialkey{INST\\DEL}, the character to the left is deleted, and all characters to the right are shifted one position to the left.

To insert a character, hold \specialkey{SHIFT} and press \specialkey{INST\\DEL}. All the characters to the right of the cursor are shifted to the right. This allows you to type a letter, number or any other character at the newly inserted space.


\subsubsection{CLeaR/HOME}
\index{Keyboard!CLR HOME}
Pressing \specialkey{CLR\\HOME} places the cursor at the top left-most position of the screen.

Holding down \specialkey{SHIFT} and pressing \specialkey{CLR\\HOME} clears the entire screen {\it and} places the cursor at the top left-most position of the screen.

\subsubsection{MEGA KEY}
\index{Keyboard!MEGA Key}
\megasymbolkey or the MEGA key provides a number of different functions and can be used to launch special utilities.

Holding \specialkey{SHIFT} and pressing \megasymbolkey switches between lower and uppercase character modes.

Holding \megasymbolkey and pressing any key with two graphic symbols on the front prints the left-most graphic symbol to the screen.

Holding \megasymbolkey and pressing any key that shows a single graphic symbol on the front prints that graphic symbol to the screen.

Holding \megasymbolkey and pressing a number key switches to one of the colours in the second range, i.e., the colour that is printed at the bottom row on the front of the number key will be used.

Holding \megasymbolkey and pressing \specialkey{TAB} enters the Matrix Mode Debugger (refer to the {\bf MEGA65 Book} for more details).

Switching on the MEGA65 or pressing the reset button on the left-hand side while holding down \megasymbolkey switches the MEGA65 into C64-mode.

\subsubsection{NO SCROLL}
\index{Keyboard!NO SCROLL}
If a program is being {\bf LIST}ed to the screen, pressing \specialkey{NO\\SCROLL} freezes the screen output. This feature is not available in C64-mode.

\subsection{Function Keys}
\index{Keyboard!Function Keys}
There are seven Function keys available for use by software applications, \megakey{F1} \megakey{F3} \megakey{F5} \megakey{F7} \megakey{F9} \megakey{F11} and \megakey{F13} can be used to perform specific functions with a single press.

Hold \specialkey{SHIFT} to access \megakey{F2} through to \megakey{F14} as shown on the front of each Function key.
\index{Keyboard!Shift Keys}
Only Function keys \megakey{F1} to \megakey{F8} are available in C64-mode.

\subsubsection{HELP}
\index{Keyboard!HELP}
\specialkey{HELP} can be used by software and also acts as \megakey{F15} / \megakey{F16}.

\subsubsection{ALT}
\index{Keyboard!ALT}
Holding \specialkey{ALT} down while pressing other keys can be used by software to perform specific functions. Not available in C64-mode.

Holding \specialkey{ALT} down while switching the MEGA65 on activates the Utility Menu. You can format an SD card, or enter the MEGA65 Configuration Utility to select the default video mode and change other settings, or to test your keyboard.

\subsubsection{CAPS LOCK}
\index{Keyboard!CAPS LOCK}
\specialkey{CAPS\\LOCK} works similarly to \specialkey{SHIFT\\LOCK} in C65 and MEGA65-modes, but only modifies the letter keys.
Also, holding \specialkey{CAPS\\LOCK} down forces the processor to run at the maximum speed. This can be used for things such as
speeding up loading from the internal disk drive or SD card, or to greatly speed up the de-packing process after a program is run.
This can reduce the loading and de-packing time from many seconds to as little as a fraction of a second.
